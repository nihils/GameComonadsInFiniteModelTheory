\documentclass{ociamthesis}
\usepackage[utf8]{inputenc}
\usepackage{amsmath}
\usepackage{amssymb}
\let\amssquare\square
\usepackage{amsfonts}
\usepackage{amsthm}
\usepackage{fullpage}
\usepackage{enumitem}
\usepackage[square,numbers]{natbib}
\usepackage{extarrows}
%\usepackage{parskip}
\input diagxy

\setlength{\parindent}{0pt}
%\setlength{\parskip}{\baselineskip}

\theoremstyle{plain}
\newtheorem{thm}{Theorem}[section]
\newtheorem{lem}[thm]{Lemma}
\newtheorem{prop}[thm]{Proposition}
\newtheorem{cor}{Corollary}[thm]
\newtheorem{clm}{Claim}[thm]

\theoremstyle{definition}
\newtheorem{defn}{Definition}[section]

\theoremstyle{remark}
\newtheorem*{fact}{Fact}
\newtheorem{rmrk}{Remark}[section]
\newtheorem{exmpl}{Example}[section]

\newcommand{\pcomonad}[2]{{\mathbb{T}_{#1}#2}}
\newcommand{\efcomonad}[2]{{\mathbb{E}_{#1}#2}}
\newcommand{\mcomonad}[2]{{\mathbb{M}_{#1}#2}}
\newcommand{\gcomonad}[2]{{\mathbb{G}_{#1}#2}}
\newcommand{\bcomonad}[3]{{\mathbb{O}_{#1,#2}#3}}

\newcommand{\pgame}[2]{{\mathsf{P}(#1 \rightarrow #2)}}
\newcommand{\efgame}[2]{{\mathsf{EF}(#1 \rightarrow #2)}}
\newcommand{\mgame}[2]{{\mathsf{M}(#1 \rightarrow #2)}}
\newcommand{\ggame}[2]{{\mathsf{G}(#1 \rightarrow #2)}}

\newcommand{\pgameSym}[2]{{\mathsf{P}(#1,#2)}}
\newcommand{\efgameSym}[2]{{\mathsf{EF}(#1,#2)}}
\newcommand{\mgameSym}[2]{{\mathsf{M}(#1,#2)}}
\newcommand{\ggameSym}[2]{{\mathsf{G}(#1,#2)}}

\newcommand{\pgameBij}[2]{{\mathsf{P}^{bij}(#1,#2)}}
\newcommand{\efgameBij}[2]{{\mathsf{EF}^{bij}(#1,#2)}}
\newcommand{\mgameBij}[2]{{\mathsf{M}^{bij}(#1,#2)}}
\newcommand{\ggameBij}[2]{{\mathsf{G}^{bij}(#1,#2)}}

\newcommand{\equivL}[3]{{#1 \equiv #2 \ \mathsf{in}\: #3}} 

\newcommand{\Fraisse}{Fra{\"i}ss{\'e}}

\let\oldiota\iota
\renewcommand{\iota}{\dot{\oldiota}}

\title{Game Comonads in Finite Model Theory}
\author{Nihil Shah\\ Supervisor: Dr. Samson Abramsky}
\college{Lady Margaret Hall}
\degree{MSc. in Mathematics and Foundations of Computer Science}
\degreedate{September 2017}

\begin{document}
\maketitle
\begin{abstract}
Back-and-Forth or Ehrenfreucht-{\Fraisse} style games are an important tool in finite model theory, complexity theory, and database theory. Monads and comonads are constructions of category theory that appear often in the semantics of computation and functional programming languages. The paper \cite{Abramsky2017} showed that existential $k$-pebble games have a natural comonadic formulation. The purpose of this dissertation is to modify and generalize the results in \cite{Abramsky2017} to give a comonadic formulation of Ehrenfreuct-{\Fraisse} games, modal and guarded bisimulation games. The results presented here lay the basis for a new research programme that connect two, previously disjoint areas of logic in computer science: (1) finite and algorithmic model theory and (2) categorical structures of computation. This is a step towards the goal of giving purely categorical definitions of logics over finite structures and complexity classes.\\~\\ 
In Chapter 2, we provide the necessary notations and background for the concepts in finite model theory and category theory we will need. \\~\\
In Chapter 3, we construct a comonad $\efcomonad{k}{}$ that internalizes Ehrenfreucht-{\Fraisse} games. The Ehrenfreucht-{\Fraisse} game captures equivalence in finite quantifier rank fragments of first-order logic. The second section defines and proves the construction is indeed a comonad. The third section shows that $\efcomonad{k}{}$ indeed captures the game and equivalence in finite rank logic. The final section studies the category of coalgebras of $\efcomonad{k}{}$ and proves the suprising result that these coalgebras provide a definition of tree-depth.  \\~\\
In Chapter 4, we construct the comonad, discovered in \cite{Abramsky2017}, $\pcomonad{k}{}$ that internalizes pebbling games. The pebbling game captures equivalence in finite variable fragments of first-order and infinitary logic. We reproduce the proofs for the results analogous to the ones in chapter 2. A novel proof is given for the result \cite[]{Abramsky2017} that gives a definition of tree-width in terms of the category of coalgebras of $\pcomonad{k}{}$. \\~\\ 
In Chapter 5, we construct comonads $\mcomonad{k}{}$ and $\gcomonad{k}{}$ which internalize bisimulations for the modal and guarded framgents of infinitary logic. We prove results analogous to the ones in chapter 3 and 4 for $\mcomonad{k}{}$ and $\gcomonad{k}{}$ comonads. \\~\\ 
In Chapter 6, we explain some results that generalize and relate the $\efcomonad{k}{}$, $\pcomonad{k}{}$, and $\mcomonad{k}{}$ comonads. The first section gives a general arrow-theoretic development for transferring winning strategies in an asymmetric game to winning strategies in a symmetric game. The second section relates the $\efcomonad{k}{}$ and $\pcomonad{k}{}$. The third section demonstrates how these comonads yield new proofs of inclusions among fragments of first-order logic and inequalities among combinatorial parameters of relational structures. \\~\\ 
In Chapter 7, we summarize the main results. We also point towards further application of this work to functional programming with Idris code samples. We dicuss possible ways of creating comonads for extensions (e.g. generalized quantifiers, branching quantifiers) of first-order logic. \\~\\
Our novel contribution is:
\begin{itemize}
\item The construction of $\efcomonad{k}{}$ and results showing that it provides a comonadic formulation of asymmetric, symmetric, and bijective Ehrenfreucht-{\Fraisse} games. The charcterization of tree-depth of structures in terms of the coalgebras of $\efcomonad{k}{}$
\item Showing that the $\mcomonad{k}{}$ unravelling construction in \cite{Gradel2014} provides a comonadic formulation of simulation and bisimulation games in the modal fragment of first-order logic.
\item The general arrow-theoretic development relating asymmetric games and symmetric games.
\item A natural transformation from $\efcomonad{k}{}$ to $\pcomonad{k}{}$ gives a novel proof of the result that $k$-rank logic is contained within $k$-variable logic. The same natural transformation gives a novel proof that tree-depth of a structure is strictly greater than its tree-width. 
\item A natural transformation between $\mcomonad{\omega}{}$ and $\pcomonad{2}{}$ gives a novel proof that modal logic is contained within $2$-variable logic. 
\end{itemize}
\end{abstract}

\chapter{Introduction} 
Given a new mathematical theory $T$, model theory studies the relationship between the formal sentences needed to express $T$ and the mathematical structures that satisfy $T$. Category theory, on the other hand, studies the structure-preserving relationships between two or more structures that satisfy $T$. The fact that both of these fields abstract notions that are employed in every mathematical field, mean they occupy a high-level, even philosophical\cite{Mancosu2010}\cite{Landry2017}, place in mathematics. Despite their abstract nature, both model theory and category theory manifest concretely in the field of logic applied to computer science. In the case of model theory, restricting attention to finite mathematical structures, i.e. finite model theory, provides new avenues in complexity theory \cite{Immerman1998} and database theory \cite{Abiteboul1995}. In the case of category theory, many constuctions, in particular monads and comonads, are used in the semantics of programming languages \cite{BrookesGeva1992}. This dissertation draws a bridge between the applications of these two fields. The bridge consists of capturing games, e.g. Ehrenfreuct-{\Fraisse}, pebbling, bisimulation, which are used to prove when two structures are indistinguisable by a logic as comonadic constructions. This categorical reformulation ``internalizes'' the games between two structures as a constructions within the category of structures itself. Not only does this internalization of games provide easy proofs of common results, the category of coalgebras give a categorical definition of combinatorial parameters, like tree-depth and tree-width. The inspiration for this dissertation was the paper \cite{Abramsky2017} which constructed a family of comonads $\{\pcomonad{k}{}\}$ that internalizes pebble games. With this dissertation we construct and prove analogous results for a family of comonads $\{\efcomonad{k}{}\}$ that internalizes Ehrenfreucht-{\Fraisse} games. We also show that the unfolding comonad for modal $\{\mcomonad{k}{}\}$ logics detailed in \cite{Gradel2014}, share similar properties and internalize bisimulation games.  
\\~\\
\noindent All results are proven by author, unless explictly cited or stated otherwise. 
\section{History}
\subsection{Finite Model Theory and Games}
Model theory, unlike most mathematical fields which develop from other mathematics, developed from philosophical preoccupations with the langauge and concepts employed by mathematicians. Early results and constructions such as G{\"o}del's Completeness Theorem, L{\"o}wenheim-Skolem Theorems, and Robinson's hyperreals explicitly address the role of logic and infinity in the language of mathematics. Moreover, the definition of model-theoretic satisfaction used today is argued to have stemmed form Tarski's deflationary definitions of truth \cite[]{Mancosu2010}. Nevertheless, in the last several decades, model theory has grown into a mature mathematical field generalizing and providing insight into algebraic geometry,  number theory, and computer science. For a long time, model theory developed with no restriction on the cardinality of the structures in question and with no particular interest in the theory of finite structures. However, Fagin's Theorem discovered in Ronald Fagin's 1973 doctoral thesis demonstrated that existential second-order logic on finite structures captured the complexity class \textbf{NP}. This spawned a new interest in the model theory of finite and recursive structures.  Another reason why finite model theory has developed as its own independent study, rather than occupy a section in (unrestricted) model theory, is the different techniques used. Many of the classical results in model theory (e.g. Compactness Theorem, Lyndon's Theorem, L{\"o}wenheim-Skolem theorems, L{\'o}s-Vaught test) only work on or apply to infinite models. Moreover, the common constructions (e.g. Henkin models, ultraproducts) produce infinite structures. One technique that works both on finite and infinite structures are back-and-forth games. This technique, phrased as systems for partial isomorphisms, was developed in {\Fraisse}'s 1953 thesis. Arguably, this work originates from generalizing Cantor's back-and-forth argument which showed two countable dense linear orders are isomorphic. Ehrenfreucht, in 1961, phrased these back-and-forth systems, perspeciously, as a two-player game. Since then, there have been many such games developed such as pebble games \cite{Immerman1982} for finite-variable logics, bisimulation games \cite{Gradel2014} for modal and guarded logics, and even fragments of second-order logics like the bijection game \cite{Hella1996} for counting-quantifier logic and extended pebble game \cite{Libkin2004} for monadic second-order logic.
\subsection{Category Theory and Comonads}
Category theory, finds its origin in the 1945 paper \textit{General Theory of Natural Equivalences} by Eilenberg and MacLane. This paper develops the general theory of functors and natural transformations. The paper shows how notions of functor, isomorphism, and natural isomorphism are used for classifying and distinguishing structures through different notions of equivalence. Eilenberg and MacLane's paper had emerged out of generalizing work in homology and cohomology; group structures that capture the relationship between local and global data of a space. Comonads had first appeared in the work of Roger Godemont, in 1958, in a book on sheaf theory. Sheaves were integral to the modern turn of algebraic geometry and found their origins in studying manifolds. Just as with cohomology, sheaves are related to capturing the local data of a space. From this history, it becomes clear that category theory is a powerful language for (1) distingushing structures through notions of equivalence and (2) capturing local information about a mathematical object. Considering that Ehrenfreucht-{\Fraisse} style games capture equivalence of structures in a ``local' way (e.g. finite rank, finite variable fragments), it seems natural to reformulate these games in category theory. Moreover, the seminal paper \cite{Moggi1991} Eugenio Moggi, sparked the use of monads and comonads in computer science. In this paper, Moggi provides a categorical semantics of computation. Essentially, Moggi represents types in a program as objects in category and computation as an functor (in fact, a monad) on this category. This abstraction was an intentinally vague notion of computation that allowed the construction to capture and formalize many ideas in modern functional programming languages.  
\section{General research programme}
In finite model theory, through the discovery of Fagin's theorem, the field of descriptive complexity was born. The field is often praised for giving a more intrinsic definition of computational complexity classes. Namely, descriptive complexity measures complexity, not through steps or tapes on machine models, but rather through the amount of expressive resources needed \textit{describe} the problem. On the other hand, through Moggi's work on monads in computation, a very abstract definition of computation is given. By formulating games, which are integral to inexpressiblity proofs, in terms of comonads, we make step towards giving intrinsic categorical, rather than model-theoretic, definition of complexity classes. This step is part of a larger research programme. \\~\\
Namely, in the conclusion of \cite{Abramsky2017}, the authors mention questions related to a larger research programme linking finite model theory in the analysis of computational complexity, and the categorical semantics of programs. More specifically, the conclusion asks whether there are comonad formulations, similar to one given for the pebbling game, for the Ehrenfrecht-Frassie game. It is also asks whether the unravelling comonad in \cite{Gradel2014} plays a similar role. Moreover, in light of the result linking tree-width to coalgebras of the pebbling comonad, it is asks whether other combinatorial parementers have similar definitions. All of these questions are answered in the affirmative. One long range goal of the progamme is to analyze and generalize Rossman's seminal theorem on homomorphism preservation using category theory. Given that much of the work on the Ehrenfreucht-{\Fraisse} game in this dissertation was inspired by Rossman's paper, we provide further evidence for the success of such an endeavour. Furthermore, the colimit constructions we investigate may also shed light on the comments made in \cite{Abramsky2017} on the paper \cite{Nesetril2013} relating different limits of graphs with bounded tree-depth. 

\chapter{Background}
The primary purpose of this chapter is to recall the basic definitions, constructions and results that will be used in the subsequent chapters. This chapter also serves to fix the notation for the standard constructions employed in finite model theory and category theory. For a category-theorist, the first section is an introduction to finite model theory. For a finite model-theorist, the second section is an introduction to category theory and in particular, comonads.   
\section{Finite Model Theory}
Model theory studies the relationship between sentences in a logic $\mathcal{L}$ and classes of mathematical structures $\mathcal{C}$ that satisfy these sentences. Sentences are built from non-logical symbols (specified by a signature) and the familar logical symbols (e.g. $\neg,\vee,\wedge,\exists,\forall$). Structures are built from sets that specifiy a domain of objects and interpretations for the symbols in a signature.
\subsection{Definitions}
\begin{defn}
A \textit{signature} or \textit{vocabulary}, denoted usually with lowercase greek letter $\sigma$, is a set of constant symbols $c_{1},c_{2},\dots$, function symbols $f_{1},f_{2},\dots$, and relation symbols $R_{1},R_{2},\dots$ where every function symbol and relation symbol has an associated arity. 
\end{defn}
\begin{exmpl}
The signature for groups can be $\sigma = \{*,^{-1},e\}$ where $*$ is the binary group operation, $^{-1}$ is the unary inverse operation, and $e$ is the constant denoting the idenity in the group. 
\end{exmpl}
\begin{defn}
A signature $\sigma$ is a \textit{relational signature} if it only has relation symbols.
\end{defn}
\begin{exmpl}
A coloured graph has relational signature $\sigma = \{E,R,G,B\}$ where $E$ is the symbol for the binary edge relation. The $R$, $G$, and $B$ are unary relations symbols picking out the red, green, and blue (respectively) coloured vertices. 
\end{exmpl}
\begin{exmpl}
A $m$-uniform hypergraph has signature $\sigma = \{E\}$ where $E$ is an $m$-ary relation symbol for hyperedges.
\end{exmpl}
For every $\sigma$-structure in a relational signature $\sigma$, we can associate a graph that will be useful for some combinatorial parameters of the structure.
\begin{defn}
Given a $\sigma$-structure $A$ in a relational signature $\sigma$, the \textit{Gaifman Graph of $A$}, denoted $\mathcal{G}(A)$, is the undirected graph with vertices being elements of $A$. The vertices $a,a' \in A$ are connected by edge if there exists some $R \in \sigma$ with $a,a'$ appearing in the same tuple of $R^{A}$. 
\label{defn:gaifmanGraph}
\end{defn}
\begin{defn}
Given a signature $\sigma$, a \textit{$\sigma$-structure} 
$$A = \langle |A|, \{c_{i}^{A}\}, \{f_{i}^{A}\}, \{R_{i}^{A}\} \rangle$$ 
consists of a set $|A|$ called the \textit{universe} of $A$ together with an interpretation of:
\begin{itemize}
    \item each constant symbol $c_{i}$ from $\sigma$ is an element $c_{i}^{A} \in |A|$ 
    \item each function symbol $f_{i}$ with arity $k$ from $\sigma$ is a function $f_{i}^{A}:|A|^{k} \longrightarrow |A|$
    \item each relation symbol $R_{i}$ with arity $k$ from $\sigma$ is a $k$-ary relation $R_{i}^{A}$ on $|A|$ (i.e. $R_{i}^{A} \subseteq |A|^{k}$) 
\end{itemize}
\label{defn:structure}
\end{defn}
As observed in many mathematical fields, a class of structures comes with a corresponding notion of structure-preserving function, or homomorphism. 
\begin{defn}
Given a signature $\sigma$, and two $\sigma$-structures, $A,B$. function $g:|A| \longrightarrow |B|$ is a \textit{$\sigma$-morphism} if: 
\begin{itemize}
    \item for constant symbol $c_{i} \in \sigma$, $g(c_{i}^{A}) = c_{i}^{B}$
    \item for function symbol $f_{i} \in \sigma$ with arity $k$, $g(f_{i}^{A}(a_{1},\dots,a_{k})) = f_{i}^{B}(g(a_{1}),\dots,g(a_{k}))$
    \item for relation symbol $R_{i} \in \sigma$ with arity $k$, $(a_{1},\dots,a_{k}) \in R_{i}^{A} \Rightarrow (g(a_{1}),\dots,g(a_{k})) \in R_{i}^{B}$
\end{itemize}
\end{defn}
This also leads to a very fine notion of equivalence between two $\sigma$-structures. 
\begin{defn}
A bijective $\sigma$-morphism $f:A \longrightarrow B$ whose inverse $g:B \longrightarrow A$ is also a $\sigma$-morphism is called an \textit{isomorphism}. Two $\sigma$-structures $A,B$ are \textit{isomorphic}, denoted $A \cong B$, if there exists an isomorphism between them. 
\end{defn}
A common simplification in finite model theory texts is to only consider relational signatures. This is a resonable assumption since any function symbol $f$ of arity $k$ can be replaced with relation symbol $R_{f}$ of arity $k+1$ whose intrepretation $R^{A}_{f}$ is the graph of the intrepretation $f^{A}$. Any constant symbol $c$ can be replaced with relation symbol $P_{c}$ of arity $1$ whose interpretation is the singleton $\{c^{A}\}$. Hence, for the rest of this paper we will consider $\sigma$ to be a relational signature.  It should be noted that translations to a relational signature do not preserve measures of formula complexity, such as quantifier rank or number of variables. \\~\\
 The logics we will consider will have countably many variables. Namely, for every $j \in \omega$, there is symbol $x_{j}$ representing a variable. For clarity of presentation, we may use lowercase latin letters (possibly with subscripts) $z,y,w,\dots$ also as variables in our logic, but these can always be replaced with the more formal $x_{j}$ for some $j \in \omega$. Define $[n] := \{1,\dots,n\}$, i.e. the finite segment of the natural numbers.   
\begin{defn}
A \textit{formula} $\phi(\mathbf{x})$ with free variables among $\mathbf{x} = (x_{1},\dots,x_{n})$ is recursively defined as: 
$$ \phi(\mathbf{x}) ::= x_{i} = x_{j} \mid R_{z}(x_{i_{1}},\dots,x_{i_{k}}) \mid \neg \phi(\mathbf{x}) \mid \bigvee_{j \in J} \phi_{j}(\mathbf{x}) \mid \bigwedge_{j \in J} \phi_{j}(\mathbf{x}) \mid \exists y \phi(y,\mathbf{x}) \mid \forall y \phi(y,\mathbf{x})$$
with $\{R_{z}\}$ the relation symbols in signature $\sigma$, $i,j,i_{1},\dots,i_{k} \in [n]$, and $J$ a (possibly infinite) indexing set.  
\label{defn:formula}
\end{defn}
\begin{defn}
A \textit{sentence} $\phi$ is a formula with no free variables.
\end{defn}
With these defintions in place, we can define the central relationship studied in model theory. Namely, the relation $\vDash$, or `satisfies', between the structures as defined in (\ref{defn:structure}) and formulas as defined in (\ref{defn:formula}).
\begin{defn}
Given an $\sigma$-structure $A$ with $\mathbf{a} = (a_{1},\dots,a_{n}) \in A^{n}$ and formula $\phi(\mathbf{x})$ with free variables among $\mathbf{x} = (x_{1},\dots,x_{n})$, define $\vDash$ by induction on the complexity on formula $\phi$:
\begin{align*}
A,\mathbf{a} &\vDash x_{i} = x_{j} \Leftrightarrow a_{i} = a_{j} \\
 &\vDash R_{z}(x_{i_{1}},\dots,x_{i_{m}}) \Leftrightarrow (a_{i_{1}},\dots,a_{i_{m}}) \in R_{z}^{A} \\
 &\vDash \neg \phi(\mathbf{x}) \Leftrightarrow \text{it is not the case that } A,\mathbf{a} \vDash \phi(\mathbf{x}) \\
 &\vDash \bigvee_{j \in J} \phi_{j}(\mathbf{x}) \Leftrightarrow \text{for some } j \in J,  A,\mathbf{a} \vDash \phi_{j}(\mathbf{x}) \\
 &\vDash \bigwedge_{j \in J} \phi_{j}(\mathbf{x}) \Leftrightarrow \text{for every } j \in J,  A,\mathbf{a} \vDash \phi_{j}(\mathbf{x}) \\
 &\vDash \exists y \phi(y,\mathbf{x}) \Leftrightarrow \text{for some } a \in A,  A,a\mathbf{a} \vDash \phi(y,\mathbf{x}) \\
 &\vDash \forall y \phi(y,\mathbf{x}) \Leftrightarrow \text{for every } a \in A,  A,a\mathbf{a} \vDash \phi(y,\mathbf{x}) 
\end{align*}
\end{defn}
For every formula, $\text{var}(\phi)$ denotes the number of distinct variables, free and bound, in $\phi$. For every formula, $\text{qr}(\phi)$ denotes the quantifier rank of $\phi$. This is inductively defined as:
\begin{eqnarray*}
\text{qr}(x_{i} = x_{j}) = \text{qr}(R_{i}(x_{i_{1}},\dots,x_{i_{k}})) = 0 \\
\text{qr}(\bigvee_{j \in J} \phi_{j}(\mathbf{x})) = \text{qr}(\bigwedge_{j \in J} \phi_{j}(\mathbf{x}))  = \sup_{j \in J} \text{qr}(\phi_{j}(\mathbf{x})) \\
\text{qr}(\exists y \phi(y,\mathbf{x})) = \text{qr}(\forall y \phi(y,\mathbf{x})) = \text{qr}(\phi(y,\mathbf{x})) + 1 
\end{eqnarray*}
\begin{defn}
Given a signature $\sigma$, a \textit{language} is the collection of formulas as defined in (\ref{defn:formula}). A \textit{logic} is collection of languages which differ only in signature. 
\end{defn}
In particular, the collection of all formulas in definition (\ref{defn:formula}), ranging over every signature, gives the infinitary logic usually denoted $\mathcal{L}_{\infty,\omega}$. We will be studying expressibilty in fragments of $\mathcal{L}_{\infty,\omega}$. For this reason, we need notation to denote the different fragments.
\begin{defn}
For every $\alpha$ ordinal or $\alpha = \infty$, $z \in \omega+1$ and $y \in \omega+1$, $\mathcal{L}^{y}_{\alpha,z}$ denotes the logic with formulas $\phi$ where:
\begin{itemize}
    \item the indexing set $J$ (used in the $\bigvee, \bigwedge$ clauses in the definition (\ref{defn:formula})) has cardinality $< \alpha$ (or unrestricted if $\alpha = \infty$)
    \item $\text{qr}(\phi) \leq z$ if $z \in \omega$ or arbitrary if $z = \omega$
    \item $\text{var}(\phi) \leq y$ if $y \in \omega$ or arbitrary if $y = \omega$
\end{itemize}
If $y$ is not specified, then the $y = \omega$, i.e. the logic $\mathcal{L}_{\alpha,z} = \mathcal{L}^{\omega}_{\alpha,z}$. Moreover, for a logic $\mathcal{L}$, the fragment without $\neg$ or $\forall$ statements is called the existential positive fragment and is denoted, $\exists^{+}\mathcal{L}$.
\end{defn}
Hence, ordinary first-order logic is $\mathcal{L}^{\omega}_{\omega,\omega} = \mathcal{L}_{\omega,\omega}$. As an example, $\exists^{+}\mathcal{L}^{5}_{\omega,3}$ is the existential positive fragment of first-order logic with formulas using at most $5$ distinct variables and quantifier rank $\leq 3$. With this notation, we can define two important fragments, the finite rank and finite variable fragments. 
\begin{defn}
Given a $k \in \omega$, the \textit{$k$-rank logic} is $\mathcal{L}_{\omega,k}$. 
\end{defn}
Ostensibly, it may seem odd to study the finite rank fragments of first-order logic, rather than the more general infinitary logic. However, the $k$-rank fragment of first order logic is in fact equivalent to the $k$-rank fragment of infinitary logic (from the perspective of finite model theory). This clearly follows from a standard result in finite model theory:
\begin{prop}[{\cite[Lemma 3.13]{Libkin2004}}]
Given a finite signature $\sigma$, then up to logical equivalence, there are only finite many formulae $\phi$ over $\sigma$ with $\text{qr}(\phi)\leq k$ and $m$ free variables.
\end{prop}
\begin{defn}
Given a $k \in \omega$, the \textit{$k$-variable logic} is $\mathcal{L}^{k}_{\infty,\omega}$.
\end{defn}
Note that, unlike with the case of $k$-rank logic, there are in general, infinitely many formulas, up to logical equivalence, that use $k$ distinct variables. Two other fragments that we will be studying are the modal and guarded fragments of first-order logic. These fragments are obtained by altering the $\exists$ and $\forall$ clauses in definition (\ref{defn:formula}) to be guarded by a relation $R_{w} \in \sigma$ with arity $\not= 1$. We can define this more precisely: 
\begin{defn}
A formula $\phi(\mathbf{x})$ with free variables among $\mathbf{x} = (x_{1},\dots,x_{n})$ in the \textit{guarded fragment} is recursively defined just as in (\ref{defn:formula}) but with the $\exists$ and $\forall$ clauses altered to:  
$$ \exists \mathbf{y} (R_{w}(\mathbf{x},\mathbf{y}) \wedge \phi(\mathbf{y},\mathbf{x})) \mid \forall \mathbf{y}(R_{w}(\mathbf{x},\mathbf{y}) \rightarrow \phi(\mathbf{y},\mathbf{x}))$$
$\{R_{w}\}$ the relation symbols in signature $\sigma$ with arity $\not= 1$, $i_{1},\dots,i_{k} \in [n]$. The $\rightarrow$ is an abbreviation, i.e. $\phi \rightarrow \psi := \neg \phi \vee \psi$ and represents `implication'. The notation $\exists \mathbf{y}$ abbreviates $\exists y_{l} \exists y_{l-1} \dots \exists y_{1}$ for $\mathbf{y} = (y_{1},\dots,y_{l})$.
\end{defn}
For the guarded fragment, we will a the notion of complexity, similar to quantifier rank, called guarded depth. For a guarded formula $\phi(\mathbf{x})$, $\text{gd}(\phi)$ denotes the guarded depth of $\phi$. This is inductively defined as:   
\begin{eqnarray*}
\text{gd}(x_{i} = x_{j}) = \text{qd}(R_{i}(x_{i_{1}},\dots,x_{i_{k}})) = 0 \\
\text{gd}(\bigvee_{j \in J} \phi_{j}(\mathbf{x})) = \text{gd}(\bigwedge_{j \in J} \phi_{j}(\mathbf{x}))  = \sup_{j \in J} \text{gd}(\phi_{j}(\mathbf{x})) \\
\text{gd}(\exists \mathbf{y} (R_{w}(\mathbf{x},\mathbf{y}) \wedge \phi(\mathbf{y},\mathbf{x}))) = \text{gd}(\forall \mathbf{y}(R_{w}(\mathbf{x},\mathbf{y}) \rightarrow \phi(\mathbf{y},\mathbf{x}))) = \text{gd}(\phi(\mathbf{y},\mathbf{x})) + 1 
\end{eqnarray*}
Note that guarded depth $\phi$ is less than the quantifier rank of $\phi$ since a guarded quantification allows for quantification over full tuples $\mathbf{y} = (y_{1},\dots,y_{l})$. We will denote the guarded fragment of depth $k$ of $\mathcal{L}_{\infty,\omega}$ as $\mathcal{G}^{k}_{\infty,\omega}$.
\begin{defn}
The \textit{modal fragment} is the guarded fragment restricted to signatures $\sigma$ with relations that have arity at most $2$. 
\end{defn}
We will denote the modal fragment of guarded depth $k$ of $\mathcal{L}_{\infty,\omega}$ as $\mathcal{M}^{k}_{\infty,\omega}$.
The modal fragment is the image of translating modal logic (i.e. propositional logic enriched with modal operators $\amssquare_{i}$ and $\lozenge_{i}$). The semantics of modal logic is a transition system or Kripke structure, where each pair of modal operators (e.g. $\amssquare_{i}$ and $\lozenge_{i}$) has an associated binary accessibility relation and each node in the transition system satisfies a certain set of propositions. In the translation to the modal fragment, the binary relations in $\sigma$ are the accessibilty relations and the unary relations are the propositions. Modal logic, due to it's nice decidability properties and possible worlds interpretation, have found application in many fields of computer science and philosophy. The guarded fragment is the image of translating polymodal logic. Polymodal logic is an generalization of modal logic where accessibility relations are not just binary, but pick out accessibility regions represented by hyperredges. \\~\\    
In addition to fragments of $\mathcal{L}_{\infty,\omega}$, we will also consider fragments of a extended logic $\mathcal{L}_{\infty,\omega}(\mathsf{Cnt})$. This logic $\mathcal{L}_{\infty,\omega}(\mathsf{Cnt})$ adds counting quantifiers $\exists_{\leq n}x_{j}$, $\exists_{\geq n}x_{j}$ for every $n \in \mathbb{N}$ and variable $x_{j}$. The semantics of $\exists_{\leq n}\phi(x)$ is $A \vDash \exists_{\leq n}x \phi(x)$ if and only if there are at least $n$ elements $a \in A$ such that $A,a \vDash \phi(x)$. We extend the notion of quantifier rank to include counting quantifiers, $\mathsf{qr}(\exists_{\leq n}x\phi(x)) = \mathsf{qr}(\exists_{\geq n}x\phi(x)) = \mathsf{qr}(x) + 1$.  
\subsection{Inexpressiblity and Games}
The primary question, once a logic $\mathcal{L}$ is defined, is to understand what properties of structures can be expressed in $\mathcal{L}$ and what properties are impossible to express in $\mathcal{L}$. In particular, finite model theory is concerned with what properties of \textit{finite} structures can be expressed in a given logic. Although mathematicians have intuitive idea of what `property' means, we will be more explicit:
\begin{defn}
Given a signature $\sigma$, a \textit{property} $P$ of $\sigma$-structures, is a (possibly, proper) class function:
$$P:\mathcal{C} \longrightarrow \{0,1\}$$
such that for $A,B \in \mathcal{C}$:
$$A \cong B \Rightarrow P(A) = P(B)$$ 
where $\mathcal{C}$ is a class of all finite $\sigma$-structures.
\end{defn}
Intuitively, $P(A) = 1$ asserts that some property is true of the structure $A$. A logic can express a property when there exists a sentence in the logic that captures all of the structures where the propety holds. 
\begin{defn}
Given a logic $\mathcal{L}$, signature $\sigma$ and property $P$ \textit{$P$ is expressible in $\mathcal{L}$ with signature $\sigma$} if there exists a sentence $\phi_{P} \in \mathcal{L}$ such that for all $\sigma$-structures $A$,
$$A \vDash \phi_{P} \Leftrightarrow P(A) = 1$$
\end{defn}
To prove a property $P$ is not expressible in signature $\sigma$, we need to exhibit two $\sigma$-structures $A$,$B$ which from the perspective of the logic $\mathcal{L}$ are ``equivalent'', but differ by property $P$. Of course, by the definiton of property given, $\cong$ is too strong of an equivalence to show inexpressiblity. For a given logic, we define what it means to be equivalent from the perspective of $\mathcal{L}$.  
\begin{defn}
Two $\sigma$ structures $A,B$ are \textit{equivalent in logic $\mathcal{L}$}, denoted $\equivL{A}{B}{\mathcal{L}}$ if for all sentences $\phi \in \mathcal{L}$, $A \vDash \phi \Leftrightarrow B \vDash \phi$.  
\label{defn:equivLogic}
\end{defn}
Hence, for a property $\mathcal{P}$ to be inexpressible in signature $\sigma$ and logic $\mathcal{L}$, we must find two $\sigma$-structures $A,B$ such that $P(A) = 1$ and $P(B) = 0$ while $\equivL{A}{B}{\mathcal{L}}$. The purpose of ``back-and-forth'' style games in finite and classical model theory is to provide a methodology to prove equivalence in a logic $\mathcal{L}$.  
\section{Category Theory}
Category theory is the general theory of mathematical structures and structure-preserving relationships between theses structures. 
\subsection{Definitions}
\begin{defn}
A \textit{category} $\mathcal{C}$ is:
\begin{itemize}
    \item a class of objects, denoted (somewhat ambiguously) as $\mathcal{C}$
    \item a class of $\mathcal{C}$-morphisms (or arrows) $f:A \longrightarrow B$ for every object $A,B \in \mathcal{C}$, denoted $\text{Hom}_{\mathcal{C}}(A,B)$.
\end{itemize}
        such that: 
\begin{enumerate}[label=(\arabic*)]
    \item For morphisms $f:A \longrightarrow B$, $g:B \longrightarrow C$, there exists a morphism $g \circ f:A \longrightarrow C$.  
    \item For every object $A \in \mathcal{C}$, there exists a morphism $\mathsf{id}_{A}:A \longrightarrow A$ such that $\mathsf{id}_{A} \circ f = f$ and $g \circ \mathsf{id}_{A} = g$ for every $f:B \longrightarrow A$, $g:A \longrightarrow C$. 
    \item For morphisms $f:A \longrightarrow B$, $g:B \longrightarrow C$ and $h:C \longrightarrow D$, $h \circ (g \circ f) = (h \circ g) \circ f$. 
\end{enumerate}
We can rephrase the condition on $\mathsf{id}_{A}:A \longrightarrow A$ by stating that the following diagrams commute for all $A \in \mathcal{C}$:
\begin{equation*}
\bfig
    \ptriangle[B`A`A;f`f`\mathsf{id}_{A}]
\efig
\bfig
    \dtriangle[A`A`C;\mathsf{id}_{A}`g`g]
\efig
\end{equation*}
We will make use of \textit{commutative diagrams}, like the ones above, to express relationships between morphisms in a category. In fact, many of our proofs can be reduced to verifying a certain diagram commutes. This method is called \textit{diagram chasing}.
\end{defn}
\begin{exmpl}
The category of sets with morphisms ordinary functions, denoted $\textbf{Set}$.
\end{exmpl}
\begin{exmpl}
The category of groups with group homomorphism.  
\end{exmpl}
\begin{exmpl}
The category of preorders with monotone functions, denoted $\textbf{Pos}$. 
\end{exmpl}
The most important categories will be the ones that appear within finite model theory. Which are:
\begin{exmpl}
Given a signature $\sigma$, the category of $\sigma$-structures with $\sigma$-morphisms will be denoted $\mathcal{R}(\sigma)$. The subcategory of finite structures in $\mathcal{R}(\sigma)$ will be denoted $\mathcal{R}_{f}(\sigma)$.
\end{exmpl}
Modulo some set-theoretic issues, the collection of categories themselves $\textbf{Cat}$ can itself be considered a category with morphisms called functors.
\begin{defn}
A \textit{functor} $F:\mathcal{C} \longrightarrow \mathcal{D}$ is a map from the objects and morphisms of $\mathcal{C}$ to the objects and morphims of $\mathcal{D}$ such that:
\begin{enumerate}[label=(\arabic*)]
    \item  For every object $A \in \mathcal{C}$, $F(\mathsf{id}_{A}) = \mathsf{id}_{F(A)}$ 
    \item  For morphisms $f:A \longrightarrow B$ and $g:B \longrightarrow C$, $F(g \circ f) = F(g) \circ F(f)$.  
\end{enumerate}
\end{defn}
\begin{exmpl}
Forgetful functors, such as $F:\textbf{Grp} \longrightarrow \textbf{Set}$ sending a group to it's underlying set.
\end{exmpl}
\begin{exmpl}
Free functors in algebra, such as $F:\textbf{Set} \longrightarrow \textbf{Grp}$ sending a set $X$ to the free group on the letters in $X$. 
\end{exmpl}
\begin{defn}
A functor $F:\mathcal{C} \longrightarrow \mathcal{C}$ from a category to itself is called an \textit{endofunctor} 
\end{defn}
Given two categories $\mathcal{C}$ and $\mathcal{D}$, we can consider the collection of all functors $F:\mathcal{C} \longrightarrow \mathcal{D}$ as a category itself with morphisms called natural transformations. 
\begin{defn}
Suppose $\mathcal{C}$ and $\mathcal{D}$ with functors $F:\mathcal{C} \longrightarrow \mathcal{D}$ such that $G:\mathcal{D} \longrightarrow \mathcal{C}$, then a \textit{natural transformation} $\eta:F \longrightarrow G$ is a collection of $\mathcal{D}$-morphisms $\eta_{A}:F(A) \longrightarrow G(A)$ in $\mathcal{D}$ (called the components are $\eta$) for every $A \in \mathcal{C}$ such that the following diagram commutes for every $\mathcal{C}$-morphism $f:A \longrightarrow B$:
\begin{equation*}
\bfig \square[F(A)`G(A)`F(B)`G(B);\eta_{A}`Ff`Gf`\eta_{B}]\efig
\end{equation*}
\end{defn}
\subsection{Comonads}
The primary focus for this dissertation, as the name suggests, is a certain class of endofunctors with ``nice'' algebraic constructions called comonads. Comonads are dual to the notion of monads. Monads are analogous to the algebraic structure called a monoid; a set with an identity and associative binary operation.
\begin{defn}
An endofunctor $\mathbb{F}:\mathcal{C} \rightarrow \mathcal{C}$ is called \textit{comonad} if there exists natural transformations $\epsilon:\mathbb{F} \longrightarrow id_{\mathcal{C}}$ (called the counit) and $\delta:\mathbb{F} \longrightarrow \mathbb{F} \circ \mathbb{F}$ (called the comultiplication) such that the following diagrams commute for all objects $A \in \mathcal{C}$:
\begin{equation}
\bfig 
    \Square[\mathbb{F}A`\mathbb{F}\mathbb{F}A`\mathbb{F}\mathbb{F}A`\mathbb{F}\mathbb{F}\mathbb{F}A;\delta_{A}`\delta_{A}`\delta_{\mathbb{F}A}`\mathbb{F}{\delta_{A}}] 
\efig
\label{eq:comultiplication}
\end{equation}
\begin{equation}
\bfig 
    \square[\mathbb{F}A`\mathbb{F}{\mathbb{F}A}`\mathbb{F}{\mathbb{F}A}`\mathbb{F}A;\delta_{A}`\delta_{A}`\mathbb{F}\epsilon_{A}`\epsilon_{\mathbb{F}A}] 
    \morphism(0,500)/=/<500,-500>[\mathbb{F}A`\mathbb{F}A;]
\efig 
\label{eq:counit}
\end{equation}
The two diagrams are analogous to the axioms of a monoid. Namely, the diagram (\ref{eq:comultiplication}) expresses the comultiplication is ``coassociative'. The diagram (\ref{eq:counit}) is analogous to the left and right identity laws of monoids.
\end{defn}
\begin{exmpl}
For a fixed set $Z$, there is the comonad on $\textbf{Set}$ given by $A \mapsto Z \times A$ (on objects) and $f \mapsto \langle \mathsf{id}_{Z},f\rangle$. The counit $\epsilon_{A}:Z \times A \longrightarrow A$ is given by $(z,a) \mapsto a$ and the comultiplication $\delta_{A}:Z \times A \longrightarrow Z \times (Z \times A)$ is given by $(z,a) \mapsto (z,(z,a))$.
\label{exmpl:fixedSetComonad}
\end{exmpl}
\begin{exmpl}
There is the sequences, or list comonad, on $\textbf{Set}$ given by sending $A \mapsto A^{\omega}$ (on objects) where $A^{\omega}$ is the set of finite sequences of elements in $A$ (i.e $[a_{1},\dots,a_{n}]$ with $n \in \omega$ and $a_{i} \in A$) and $f \mapsto ([a_{1},\dots,a_{n}] \mapsto [f(a_{1}),\dots,f(a_{n})]$). The counit is the last operation given by $[a_{1},\dots,a_{n}] \mapsto a_{n}$ and the comultiplication is given by the prefixes operation $[a_{1},\dots,a_{n}] \mapsto [[a_{1}],[a_{1},a_{2}],\dots,[a_{1},\dots,a_{n}]]$.
\label{exmpl:listComonad}
\end{exmpl}
We will be studying infinite families of comonads with an additional ``grading'' relating the comonads in the family.
\begin{defn}
Given $(I,\leq)$ an indexing partially-ordered set, \textit{an $I$-graded family of comonads} is a set $\{\mathbb{F}_{i}\}_{i \in I}$ such that each $\mathbb{F}_{i}$ is a comonad and for every $i, j \in I$ with $i \leq j$, there exists a natural transformation $i^{i,j}:\mathbb{F}_{i} \longrightarrow \mathbb{F}_{j}$.  
\end{defn}
In most cases, the indexing set is the ordinal $\omega$ with its usual total order.  
\begin{exmpl}
For every $k \in \omega$, there is the fixed set comonad $F_{[k]}$ where $A \mapsto [k] \times A$, as in example (\ref{exmpl:fixedSetComonad}). Since $[l] \subseteq [k]$ for $l \leq k$, the components of the grading natural transformation are given by the inclusion maps. 
\end{exmpl}
\begin{exmpl}
For every $k \in \omega$, there is the sequences of length $\leq k$ comonad analogous to the example (\ref{exmpl:listComonad}). The components of the grading natural transformation are given by the inclusion maps. 
\label{exmpl:listComonadGraded}
\end{exmpl}
\begin{rmrk}
The comonads we will be studying are similar to the sequences comonad of example (\ref{exmpl:listComonad}) and graded sequences comonad of example (\ref{exmpl:listComonadGraded}). In fact, the forgetful functor from the category of structures $\mathcal{R}(\sigma)$ to sets takes the comonads we will discuss to the sequences or graded sequences comonad. For this reason, we will need some notation for discussing sequences. 
\end{rmrk}
\subsection*{Notation for Sequences}
\begin{defn}
Suppose $s,s'$ are two sequences such that $s = [s_{1},\dots,s_{n}]$ and $s' = [s'_{1},\dots,s'_{m}]$, then define $ss' = [s_{1},
\dots,s_{n},s'_{1},\dots,s'_{m}]$ (i.e. the concatenation of these two sequences).
\end{defn}
\begin{defn}
For two sequences $s,t$, say $s \sqsubseteq t$ if there exists an $s'$ such that $ss' = t$; such an $s'$ is called the suffix of $s$ in $t$. $\sqsubseteq$ preorders sets of sequences.  
\end{defn}
For some types of sequence, we will need notation for a successor or update operation. 
\begin{defn}
Given two sequences $s,s'$, $s'$ is the \textit{$a$-successor} of $s$ if $s' = s[a]$. We denote this, $s \xlongrightarrow{a} s'$. If $s'$ is the successor of $s$ for some element $a$, then $s \longrightarrow s'$
\end{defn}
\subsection{Co-Kleisli Category and Eilenberg-Moore Category}
For every comonad $\mathbb{F}:\mathcal{C} \longrightarrow \mathcal{C}$, there exist two important categories associated with $\mathbb{F}$. The coKleisli category of $\mathbb{F}$ and the category of coalgebras, or Eilenberg-Moore category, of $\mathbb{F}$. 
\begin{defn}
The \textit{coKleisli category of $\mathbb{F}$}, denoted $\mathcal{K}(\mathbb{F})$, is a category with:
\begin{itemize}
    \item objects are the same objects as $\mathcal{C}$
    \item morphisms from $A$ to $B$ given by $\mathcal{C}$-morphism $f:\mathbb{F}A \longrightarrow B$. That is, $\text{Hom}_{\mathcal{K}(\mathbb{F})}(A,B) = \text{Hom}_{\mathcal{C}}(\mathbb{F}A,B)$
\end{itemize}
such that 
\begin{enumerate}[label=(\arabic*)]
    \item For every object $A \in \mathcal{K}(\mathbb{F}) = \mathcal{C}$, the identity morphism is given by the $A$ component of the counit:  
    \begin{equation}
        \epsilon_{A}:\mathbb{F}A \longrightarrow A
    \end{equation}
    \item For two $\mathcal{C}$-morphisms $f:\mathbb{F}A \longrightarrow B$ and $g:\mathbb{F}B \longrightarrow C$ (i.e. two $\mathcal{K}(F)$ morphisms) we use the comonad structure to compose them to produce a morphism $g \circ_{\mathcal{K}} f:\mathbb{F}A \longrightarrow C$:
    \begin{equation} 
        \mathbb{F}A \xlongrightarrow{\delta_{A}} \mathbb{F}\mathbb{F}A \xlongrightarrow{\mathbb{F}f} \mathbb{F}B \xlongrightarrow{g} C 
    \label{eq:coextension}
    \end{equation}
\end{enumerate}
\end{defn}
The definition (\ref{eq:coextension}) makes use of the morphism $\mathbb{F}f \circ \delta_{A}$ to define the composition law in $\mathcal{K}(\mathbb{F})$. This operation is used often and even provides an alternative set of axioms of for the coKleisli category \cite{Moggi1991}. 
\begin{defn}
The operation $f \mapsto \mathbb{F}f \circ \delta_{A}$ which sends $f:\mathbb{F}A \longrightarrow B$ to $f^{*}: \mathbb{F}A \longrightarrow \mathbb{F}B$ is called the \textit{Kleisli coextension}. 
\end{defn}
It is common in algebra for an algebraic structure to have an associated class of objects that the structure ``acts on''. For example, groups `act on' sets via group actions and rings `act on' modules. For comonads, the associated class of objects are \textit{coalgebras} and this class forms a category.
\begin{defn}
The \textit{category of coalgebras of $\mathbb{F}$} or \textit{Eilenberg-Moore category}, denoted $\mathcal{C}^{\mathbb{F}}$, is a category with:
\begin{itemize}
    \item objects a pair $(A,\alpha)$ with $A \in \mathcal{C}$ and $\mathcal{C}$-morphism $\alpha:A \longrightarrow \mathbb{F}A$ such that the following diagrams commute:
    \begin{equation}
        \bfig 
            \square[A`\mathbb{F}A`\mathbb{F}A`\mathbb{F}\mathbb{F}A;\alpha`\alpha`\delta_{A}`\mathbb{F}\alpha]
        \efig 
        \bfig
            \qtriangle[A`\mathbb{F}A`A;\alpha`\text{id}_{A}`\epsilon_{A}]
        \efig
        \label{eq:coalgebraLaw}
    \end{equation}
    \item morphisms from $(A,\alpha) \longrightarrow (B,\beta)$ given by $\mathcal{C}$-morphisms $h:A \longrightarrow B$ such that the following diagram commutes: 
    \begin{equation}
        \bfig
            \square[A`\mathbb{F}A`B`\mathbb{F}B;\alpha`h`\mathbb{F}h`\beta]
        \efig
    \end{equation}
\end{itemize}
The composition law is the same as in $\mathcal{C}$ and the identity on $(A,\alpha)$ is just the $\mathcal{C}$-morphism $\mathsf{id}_{A}:A \longrightarrow A$.
\end{defn}
\begin{exmpl}
For the fixed set comonad $F_{Z}$ in example (\ref{exmpl:fixedSetComonad}), the coalgebras correspond to functions $f:A \longrightarrow Z$. The colgebra $\alpha_{f}:A \longrightarrow Z \times A$, for a function $f:A \longrightarrow Z$, is given by $a \mapsto (f(a),a)$.  
\end{exmpl}
\begin{exmpl}
For the list comonad in example (\ref{exmpl:listComonad}), the coalgebras correspond to tree orders $\leq$ on $A$. The coalgebra $\alpha:A \longrightarrow A^{\omega}$, for a tree order $\leq$, is given by sending $a \in A$ to it's list of $\leq$-predecessers.  
\end{exmpl}
The diagrams (\ref{eq:coalgebraLaw}) are analogous to the presentation of group actions. Namely, the diagram on the right is analogous to the action commuting with the group operation. While the diagram on the left is analogous to the identity in the group acting trivially on a set.

\chapter{Ehrenfreucht-{\Fraisse} Comonad}
\section{Introduction}
The Ehrenfreucht-{\Fraisse} (EF) game was the first case of a two-player game used to prove equivalence between structures in fragments of first-order logic. In particular, the EF game is used to prove equivalence between structures in the $k$-rank fragments of first order logic. Other games, such as the pebbling game and bisimulation game, are essentially modifications of the EF game. Given two structures $A,B$, the Ehrenfreucht-{\Fraisse} game has two players, Spoiler and Duplicator. A $k$-round (symmetric) EF game, denoted $\efgameSym{k}{A}{B}$, is played as follows: for every $i \in [k]$, 
\begin{itemize}
\item Spoiler chooses an element in either structure $a_{i} \in A$ or $b_{i} \in B$. 
\item Duplicator chooses an element in the other structure $b_{i} \in B$ or $a_{i} \in A$  
\end{itemize}
At the end of the $k$-round game, $k$-tuples $(a_{1},\dots,a_{k})$ and $(b_{1},\dots,b_{k})$ have been chosen. Duplicator wins the $k$-round game if the map $\chi:a_{i} \longmapsto b_{i}$ is a partial $\sigma$-isomorphism from $A$ to $B$. Otherwise, Spoiler wins. The asymmetric (or existential positive) game from $A$ to $B$, denoted $\efgame{k}{A}{B}$, is the same game with the additional restriction that Spoiler must always play an element in $A$ and the map $\chi$ obtained is a partial $\sigma$-morphism. Hence, Duplicator must always respond in $B$. The following result is standard in any model theory text:
\begin{prop}
The following are equivalent:
\begin{itemize}
\item Duplicator has a winning strategy $\efgameSym{k}{A}{B}$
\item $\equivL{A}{B}{\mathcal{L}_{\omega,k}}$, i.e. for every sentence $\phi \in \mathcal{L}_{\omega,k}$, $A \vDash \phi \Leftrightarrow B \vDash \phi$
\end{itemize}
\end{prop}
The goal of the EF comonad is to construct a $\sigma$-structure $\efcomonad{k}{A}$ from a $\sigma$-structure $A$, that ``internalizes'' $\efgame{k}{A}{B}$ and $\efgameSym{k}{A}{B}$ into the category $\mathcal{R}(\sigma)$.  
\section{Comonad laws}
Let $A$ be a $\sigma$-structure over relational signature $\sigma$, then for every $k \in \omega$ we define a $\sigma$-structure $\efcomonad{k}{A}$. Intuitively, $\efcomonad{k}{A}$ is the structure of Spoiler's strategies in the Ehrenfreucht-{\Fraisse} $k$-round asymmetric game from $A$ to any $\sigma$-structure. A function $f:\efcomonad{k}{A} \longrightarrow B$, then represents Duplicator's strategy (i.e. responses) to Spoiler's plays in the $k$-round asymmetric game from $A$ to $B$. For every $i \in \omega$, let $A^{i}$ be the set of $i$-length sequences of elements in $A$. Let the domain of the structure be $|\efcomonad{k}{A}| = \bigcup_{i \leq k} A^{i}$. 
\begin{defn}
Define, for every $\sigma$-structure $A$, $\epsilon_{A}:\efcomonad{k}{A} \longrightarrow A$ by $[a_{1},\dots,a_{n}] \mapsto a_{n}$ (i.e. the last move of the play). 
\label{defn:epsilonEF}
\end{defn}
With this definitions in place, we can define a $\sigma$-structure on $\efcomonad{k}{A}$. Suppose $R \in \sigma$ is a $m$-ary relation, the we define the interpretation of $R^{\efcomonad{k}{A}}$ such that for every $s_{1},\dots,s_{m} \in |\efcomonad{k}{A}|$,
\begin{align}
(s_{1},\dots,s_{m}) \in R^{\efcomonad{k}{A}} &\Leftrightarrow \text{ for every $i,j \leq m$, } s_{i} \sqsubseteq s_{j} \text{ or } s_{j} \sqsubseteq s_{i} & \label{eq:R1st} \\ 
&\text{ and } (\epsilon_{A}(s_{1}),\dots,\epsilon_{A}(s_{m})) \in R^{A} & \label{eq:R2nd}
\end{align}
\begin{defn}
Given a morphism $f:A \longrightarrow B$, define the morphism $\efcomonad{k}{f}:\efcomonad{k}{A} \longrightarrow \efcomonad{k}{B}$ by $[a_{1},\dots,a_{n}] \mapsto [f(a_{1}),\dots,f(a_{n})]$
\label{defn:comonadMorphismEF}
\end{defn}
\begin{prop}
The definition (\ref{defn:comonadMorphismEF}) of $\efcomonad{k}{f}:\efcomonad{k}{A} \longrightarrow \efcomonad{k}{B}$ given above is indeed a morphism of $\sigma$-structures. 
\begin{proof}
Suppose $R \in \sigma$, then we want to show that if $(s_{1},\dots,s_{m}) \in R^{\efcomonad{k}{A}}$, then \linebreak $(\efcomonad{k}{f}(s_{1}),\dots,\efcomonad{k}{f}(s_{m})) \in R^{\efcomonad{k}{B}}$. For brevity, assume that $R$ is a binary relation (the proof for a general $m$-ary relation is a straightforward generalization). Suppose $s,s' \in \efcomonad{k}{A}$ such that $(s,s') \in R^{\efcomonad{k}{A}}$. Let $s = [a_{1},\dots,a_{n}]$ and $s' = [a_{1},\dots,a'_{m}]$. We aim to show that $(\efcomonad{k}{f}(s),\efcomonad{k}{f}(s')) \in R^{\efcomonad{k}{B}}$ \\
\begin{enumerate}
\item  Since $(s,s') \in R^{\efcomonad{k}{A}}$, by condition (\ref{eq:R1st}), $s \sqsubseteq s'$ or $s' \sqsubseteq s$. Without loss of generality, assume $s \sqsubseteq s'$. Since $s \sqsubseteq s'$.
$$s' = [a_{1},\dots,a_{n},a'_{n+1},\dots,a'_{m}]$$ 
(noting that for $i \leq n$, $a_{i} = a'_{i}$). Therefore $$\efcomonad{k}{f}(s) = [f(a_{1})),\dots,f(a_{n})]$$ 
$$\efcomonad{k}{f}(s') =[f(a_{1}),\dots,f(a_{n}),f(a'_{n+1}),\dots,f(a'_{m})]$$ 
Hence, $\efcomonad{k}{f}(s) \sqsubseteq \efcomonad{k}{f}(s')$ and (\ref{eq:R1st}) is satisfied. 
\item  By condition (\ref{eq:R2nd}) and $(s,s') \in R^{\efcomonad{k}{f}}$, $(\epsilon_{A}(s),\epsilon_{A}(s')) = (a_{n},a'_{m}) \in R^{A}$. Since $f:A \rightarrow B$ is a morphism of $\sigma$-structures, $(f(a_{n}),f(a'_{m})) \in R^{B}$. That is, $(\epsilon_{B}\circ \efcomonad{k}{f}(s),\epsilon_{B} \circ \efcomonad{k}{f}(s')) \in R^{B}$. Hence, (\ref{eq:R2nd}) is satisfied.
\end{enumerate}
Therefore, $(\efcomonad{k}{f}(s),\efcomonad{k}{f}(s')) \in R^{\efcomonad{k}{B}}$ and $\efcomonad{k}{f}$ is indeed a morphism of $\sigma$-structures. 
\end{proof}
\end{prop}
\begin{prop}
$\epsilon:\efcomonad{k}{} \longrightarrow 1_{\mathcal{R}(\sigma)}$ is a natural transformation.
\begin{proof}
For every $A,B \in \mathcal{R}(\sigma)$ we want to show that:
\begin{equation}
\bfig \square[\efcomonad{k}{A}`A`\efcomonad{k}{B}`B;\epsilon_{A}`\efcomonad{k}{f}`f`\epsilon_{B}] \efig
\label{eq:epsilonNEF}
\end{equation}
\begin{align*}
f \circ \epsilon_{A}([a_{1},\dots,a_{n})])  &= f(a_{n}) & \text{defn (\ref{defn:epsilonEF}) $\epsilon_{A}$}\\
&= \epsilon_{B}([f(a_{1}),\dots,f(a_{n})]) & \text{defn (\ref{defn:epsilonEF}) $\epsilon_{B}$}\\
&= \epsilon_{B} \circ \efcomonad{k}{f}([a_{1},\dots,a_{n}]) & \text{defn (\ref{defn:comonadMorphismEF}) $\efcomonad{k}{f}$}
\end{align*}
Hence, the above diagram (\ref{eq:epsilonNEF}) commutes.
\end{proof}
\label{prop:epsilonNEF}
\end{prop}
\begin{defn}
Suppose $s \in \efcomonad{k}{A}$, then $s = [a_{1},\dots,a_{n}]$ for some $n \in \omega$ and for every $i = 1,\dots, n$, $a_{i} \in A$. Let $s_{i} = [a_{1},\dots,a_{i}] \in \efcomonad{k}{A}$. Define, for every $\sigma$-structure $A$, $\delta_{A}:\efcomonad{k}{A} \longrightarrow \efcomonad{k}{\efcomonad{k}{A}}$ by $s \mapsto [s_{1},\dots,s_{n}]$.
\label{defn:deltaEF}
\end{defn}
\begin{prop}
$\delta:\efcomonad{k} \longrightarrow \efcomonad{k}{\efcomonad{k}{}}$ is a natural transformation.
\begin{proof}
For every $A,B \in \mathcal{R}(\sigma)$ we want to show that:
\begin{equation}
\bfig \square[\efcomonad{k}{A}`\efcomonad{k}{\efcomonad{k}{A}}`\efcomonad{k}{B}`\efcomonad{k}{\efcomonad{k}{B}};\delta_{A}`\efcomonad{k}{f}`\efcomonad{k}{\efcomonad{k}{f}}`\delta_{B}] \efig
\label{eq:deltaNEF}
\end{equation}
\begin{align*}
\efcomonad{k}{\efcomonad{k}{f}} \circ \delta_{A}([a_{1},\dots,a_{n}])   &= \efcomonad{k}{\efcomonad{k}{f}}([s_{1},\dots,s_{n}]) & \text{defn (\ref{defn:deltaEF}) $\delta_{A}$} \\
&= [\efcomonad{k}{f}(s_{1}),\dots,\efcomonad{k}{f}(s_{n})] & \text{defn (\ref{defn:comonadMorphismEF}) $\efcomonad{k}{\efcomonad{k}{f}}$}\\
&= [[f(a_{1})],\dots,[f(a_{1}),\dots,f(a_{n})]] & \text{defn (\ref{defn:comonadMorphismEF}) $\efcomonad{k}{f}$}\\
&= \delta_{B}([f(a_{1}),\dots,f(a_{n})]) & \text{defn (\ref{defn:deltaEF}) $\delta_{B}$} \\
&= \delta_{B} \circ \efcomonad{k}{f}([a_{1},\dots,a_{n}]) & \text{defn (\ref{defn:comonadMorphismEF}) $\efcomonad{k}{f}$}
\end{align*}
Hence, the above diagram (\ref{eq:deltaNEF}) commutes.
\end{proof}
\label{prop:deltaNEF}
\end{prop}
\begin{thm}
The triple $\langle \efcomonad{k}{},\delta,\epsilon \rangle$ is a comonad.
\begin{proof}
By proposition (\ref{prop:deltaNEF}) and (\ref{prop:epsilonNEF}), $\delta$ and $\epsilon$ are natural transformation. Hence, $\delta$ and $\epsilon$ are indeed the comultiplication and counit of $\efcomonad{k}{}$. The associative and identity laws remain to be shown. \\
For associativity, for every $A \in \mathcal{R}(\sigma)$, the following diagram commutes:  
\begin{equation}
\bfig \Square[\efcomonad{k}{A}`\efcomonad{k}{\efcomonad{k}{A}}`\efcomonad{k}{\efcomonad{k}{A}}`\efcomonad{k}{\efcomonad{k}{\efcomonad{k}{A}}};\delta_{A}`\delta_{A}`\delta_{\efcomonad{k}{A}}`\efcomonad{k}{\delta_{A}}] \efig 
\end{equation}
\begin{align*}
\delta_{\efcomonad{k}{A}} \circ \delta_{A}([a_{1},\dots,a_{n}])     &= \delta_{\efcomonad{k}{A}}([s_{1},\dots,s_{n}]) & \text{defn (\ref{defn:deltaEF}) $\delta_{A}$} \\
&= [[s_{1}],\dots,[s_{1},\dots,s_{n}]]  & \text{defn (\ref{defn:deltaEF}) $\delta_{\efcomonad{k}{A}}$}\\
&= [\delta_{A}(s_{1}),\dots,\delta_{A}(s_{n})] & \text{defn (\ref{defn:deltaEF}) $\delta_{A}$}  \\
&= \efcomonad{k}{\delta_{A}}([s_{1},\dots,s_{n}]) & \text{defn (\ref{defn:comonadMorphismEF}) $\efcomonad{k}{\delta_{A}}$}  \\
&= \efcomonad{k}{\delta_{A}} \circ \delta_{A}([a_{1},\dots,a_{n}]) & \text{defn (\ref{defn:deltaEF}) $\delta_{A}$}\\
\end{align*}
For identity, for every $A \in \mathcal{R}(\sigma)$, the following diagram commutes:  
\begin{equation}
\bfig 
    \square[\efcomonad{k}{A}`\efcomonad{k}{\efcomonad{k}{A}}`\efcomonad{k}{\efcomonad{k}{A}}`\efcomonad{k}{A};\delta_{A}`\delta_{A}`\efcomonad{k}{\epsilon_{A}}`\epsilon_{\efcomonad{k}{A}}] 
    \morphism(0,500)/=/<500,-500>[\efcomonad{k}{A}`\efcomonad{k}{A};]
\efig 
\end{equation}
\begin{align*}
\efcomonad{k}{\epsilon_{A}} \circ \delta_{A}([a_{1},\dots,a_{n}]) &= \efcomonad{k}{\epsilon_{A}}([s_{1},\dots,s_{n}]) & \text{defn (\ref{defn:deltaEF}) $\delta_{A}$}\\
&= [\epsilon_{A}(s_{1}),\dots,\epsilon_{A}(s_{n})]  & \text{defn (\ref{defn:comonadMorphismEF}) $\efcomonad{k}{\epsilon_{A}}$}  \\
&= [a_{1},\dots,a_{n}] & \text{defn (\ref{defn:epsilonEF}) $\epsilon_{A}$}\\
&= s_{n} & \text{defn (\ref{defn:deltaEF}) $s_{n}$}\\
&= \epsilon_{\efcomonad{k}{A}}([s_{1},\dots,s_{n}]) & \text{defn (\ref{defn:epsilonEF}) $\epsilon_{\efcomonad{k}{A}}$} \\
&= \epsilon_{\efcomonad{k}{A}} \circ \delta_{A}([a_{1},\dots,a_{n}]) & \text{defn (\ref{defn:deltaEF}) $\delta_{A}$}
\end{align*}
By definition, $\efcomonad{k}{}$ is a comonad.
\end{proof}
\end{thm}
For every $l,k \in \omega$ such that $l \leq k$ and $\sigma$-structure $A$, there exists an inclusion $i_{A}^{l,k}: \efcomonad{l}{A} \longrightarrow \efcomonad{k}{A}$. 
\begin{prop}
The inclusion maps form a natural transformation $i^{l,k}:\efcomonad{l}{} \longrightarrow \efcomonad{k}{}$. Further, each map preserves the counit and comultiplication (i.e. each map is a morphism of comonads). 
\end{prop}
\begin{proof}
\begin{equation}
\bfig \square[\efcomonad{l}{A}`\efcomonad{k}{A}`\efcomonad{l}{B}`\efcomonad{k}{B};i^{l,k}_{A}`\efcomonad{l}{f}`\efcomonad{k}{f}`i^{l,k}_{B}]\efig
\end{equation}
\begin{align*}
\efcomonad{k}{f} \circ i^{l,k}_{A}([a_{1},\dots,a_{n}])     &= \efcomonad{k}{f}([a_{1},\dots,,a_{n}]) & \text{inclusion}\\
&= [f(a_{1}),\dots,f(a_{n})] &\text{defn (\ref{defn:comonadMorphismEF}) $\efcomonad{k}{f}$} \\
&= i^{l,k}_{B}([f(a_{1}),\dots,f(a_{n})]) & \text{inclusion} \\ 
&= i^{l,k}_{B} \circ \efcomonad{l}{f}([a_{1},\dots,a_{n}]) & \text{defn (\ref{defn:comonadMorphismEF}) $\efcomonad{k}{f}$}
\end{align*}
\end{proof}
The grading given by these inclusion maps seem to suggest that there is a colimit object capturing the information of $\efcomonad{k}{A}$ for every $k \in \omega$. This is indeed the case. Consider the structure $\efcomonad{\omega}{A}$ with domain $|\efcomonad{\omega}{A}| = \bigcup_{k \in \omega} A^{k}$ where $A^{k}$ is the set of $k$-length sequences of elements in $A$. The structure on $\efcomonad{\omega}{A}$ is similar to the structure given to $\efcomonad{k}{A}$. 
\begin{prop}
Let $\omega$ be considered as a poset category under the usual order. The object $\efcomonad{\omega}{A}$ is the $\omega$-colimit of the family $\{\efcomonad{k}{A}\}_{k \in \omega}$ with the above inclusion maps. 
\begin{proof}
For every $k \in \omega$, define $i^{k}_{A}:\efcomonad{k}{A} \rightarrow \efcomonad{\omega}{A}$ as the inclusion (i.e. $[a_{1},\dots,a_{n}] \mapsto [a_{1},\dots,a_{n}]$ where $n \leq k$). Clearly, the following diagram commutes for all $l,k \in \omega$ with $l \leq k$
\begin{equation}
\bfig \Vtriangle[\efcomonad{l}{A}`\efcomonad{k}{A}`\efcomonad{\omega}{A};i^{l,k}_{A}`i^{l}_{A}`i^{k}_{A}]\efig
\label{eq:omegaColimitEF}
\end{equation}
Suppose that there exists a $\sigma$-structure $B$ and for every $l,k \in \omega$ with $l \leq k$, there exist morphisms $f^{l}:\efcomonad{l}{A} \longrightarrow B$, $f^{k}:\efcomonad{k}{A} \longrightarrow B$ such that $f^{l} = f^{k} \circ i^{l,k}$. Consider the morphism $u:\efcomonad{\omega}{A} \longrightarrow B$ given by $[a_{1},\dots,a_{n}] \mapsto f^{n}([a_{1},\dots,a_{n}])$. The following diagram commutes:
\begin{equation}
\bfig 
    \Vtriangle[\efcomonad{l}{A}`\efcomonad{k}{A}`\efcomonad{\omega}{A};i^{l,k}_{A}`i^{l}_{A}`i^{k}_{A}]
    \morphism(0,500)|l|/{@{>}@/^-7pt/}/<500,-1000>[\efcomonad{l}{A}`B;f^{l}]
    \morphism(1000,500)|r|/{@{>}@/^7pt/}/<-500,-1000>[\efcomonad{k}{A}`B;f^{k}]
    \morphism(500,0)|m|/.>/<0,-500>[\efcomonad{\omega}{A}`B;u]
\efig
\label{eq:omegaColimitUEF}
\end{equation}
Moreover, given the conditions on $f^{j}$ for all $j \in \omega$, $u$ is unique. Namely, suppose there exists a morphism $u':\efcomonad{\omega}{A} \longrightarrow B$ such that for all $j \in \omega$, $f^{j} = u' \circ i^{j}_{A}$. Suppose $s = [a_{1},\dots,a_{k}] \in \efcomonad{\omega}{A}$ then for all $j \geq k, s \in \efcomonad{j}{A}$.  
\begin{align*}
u(s)    &= f^{k}(s)  & \text{by defn of $u$}\\
        &= f^{j} \circ i^{k,j}_{A}(s) & \text{by (\ref{eq:omegaColimitUEF})} \\
        &= u' \circ i^{j}_{A} \circ i^{k,j}_{A}(s) & \text{by hypothesis on $u'$}  \\
        &= u' \circ i^{k}_{A}(s) & \text{by (\ref{eq:omegaColimitEF})} \\
        &= u'(s) & \text{by defn of inclusion} 
\end{align*}
Since $u(s) = u'(s)$ for all $s \in \efcomonad{\omega}{A}$, $u = u'$ so $u$ is unique as desired.  
\end{proof}
\end{prop}
\section{Positional Form and Equivalences}\label{sec:positionalFormEF}
In order to solidify the connection between the construction $\efcomonad{k}{}$ and the Ehrenfreucht-{\Fraisse} game, we have to encode the strategies in the game into ``positional form''. Intuitively, the positional form represents the set of positions of Spoiler and Duplicators' choices after a fixed number of rounds. This positional form is similar to the forth part of graded ``back-and-forth systems'' that are used in model theory texts to prove that the $k$-round symmetric EF game characterizes $\equivL{A}{B}{\mathcal{L}_{\omega,k}}$. \\~\\
Let $\Gamma_{k}(A,B) = (A \times B)^{\leq k}$ (i.e. sequences in pairs of elements in $A,B$ with length $\leq k$). Recall that for a $\sigma$-morphism $f:\efcomonad{k}{A} \longrightarrow B$, there exists the Kleisli coextension $f^{*}:\efcomonad{k}{A} \longrightarrow \efcomonad{k}{B}$. Define the set function $\theta_{f}:|\efcomonad{k}{A}| \longrightarrow \Gamma_{k}(A,B)$ by $s = [a_{1},\dots,a_{n}] \mapsto [(a_{1},b_{1}),\dots,(a_{n},b_{n})]$ where $f^{*}(s) = [b_{1},\dots,b_{n}]$. Rephrasing, $\theta_{f} = z \circ \langle \mathsf{id}_{\efcomonad{k}{A}},f^{*} \rangle \circ \Delta_{A}$ where $z$ is the `zipper' function sending two sequences of equal length to the natural sequence of pairs, and $\Delta_{A}:A \longrightarrow A \times A$ is the natural diagonal function.  
\begin{defn}
$S \subseteq \Gamma_{k}(A,B)$ is a \textit{strategy in positional form} if $S$ satisfies the following conditions:
\begin{enumerate}[label=(S\arabic*),ref=S\arabic*,start=0]
\item $\emptySeq \in S$ \label{eq:S1st}
\item For all $\chi \in S$ with $|\chi| < k$, $a \in A$, there exists a unique $b \in B$ such that $\chi[(a,b)] \in S$ \label{eq:S2nd}
\item $S$ is reachable: For all $\chi \in S$, there is a chain \label{eq:S3rd}
$$\chi_{0} \longrightarrow \dots \longrightarrow \chi_{n}$$
such that $\chi_{0} = \emptySeq$, $\chi_{n} = \chi$ and $\chi_{i} \in S$ with $|\chi_{i}| = i$ for all $i = 0,\dots,n$. 
\end{enumerate}
\end{defn}
\begin{prop}
If $f:\efcomonad{k}{A} \longrightarrow B$ is a $\sigma$-morphism, then there exists a strategy in positional form $S_{f}$.
\begin{proof}
Define
$$S_{f} = \{\theta_{f}(s) \mid s \in \efcomonad{k}{A}\} \cup \{\emptySeq\}$$
\begin{itemize}
\item $\emptySeq \in S_{f}$, so (\ref{eq:S1st}) is satisfied.
\item Suppose $\chi \in S_{f}$ with $|\chi| < k$ and $a \in A$. By definition of $S_{f}$ either $\chi = \emptySeq$ or $\chi = \theta_{f}(s)$ for some $s \in \efcomonad{k}{A}$. In the first case, consider $\theta_{f}([a]) = [(a,b)] \in S$ with $b = f([a])$. In the second case, consider $\chi' = \theta_{f}(s[a])$. Since $|\chi| < k$, the length $|s| = j$ and the length $|s[a]| = j+1 \leq k$. Hence, $s[a]$ is indeed in $\efcomonad{k}{A}$. Therefore, $\chi' \in S_{f}$ and $\chi' = \chi[(a,b)]$ where $b = f(s[a])$.  
\item Suppose $\chi \in S_{f}$, then $\chi = \theta_{f}(s)$ (the $\chi = \emptySeq$ case is trivial) for some $s = [a_{1},\dots,a_{n}] \in \efcomonad{k}{A}$. Let $s_{i} = [a_{1}.\dots,a_{i}]$ be the $i$-th element of $\delta_{A}(s) \in \efcomonad{k}{\efcomonad{k}{A}}$, then
$$ \emptySeq \longrightarrow \theta_{f}(s_{1}) \longrightarrow \dots \longrightarrow \theta_{f}(s_{n}) = \chi$$
Hence, $S_{f}$ is reachable
\end{itemize}
\end{proof}
\label{prop:fToPosFormEF}
\end{prop}
\begin{prop}
Conversely, for every strategy in positional form $S$ there exists a $f:\efcomonad{k}{A} \rightarrow B$ such that $S = S_{f}$
\begin{proof}
$S_{f} \subseteq S$ The approach is to construct an appropriate $f$. Define $\pi_{1}:\Gamma_{k}(A,B) \longrightarrow \efcomonad{k}{A}$ by $\pi_{1}:[(a_{1},b_{1}),\dots,(a_{n},b_{n})] \longrightarrow [a_{1},\dots,a_{n}]$ and $\pi_{2}:\Gamma_{k}(A,B) \longrightarrow \efcomonad{k}{B}$ by $\pi_{2}:[(a_{1},b_{1}),\dots,(a_{n},b_{n})] \longrightarrow [b_{1},\dots,b_{n}]$. We construct $f$ by recursion, up to $k$, on the length of a play $s \in \efcomonad{k}{A}$. \\ 
\textit{Base Case:} Suppose $s = [a]$ for $a \in A$. By (\ref{eq:S1st}) and (\ref{eq:S2nd}), there exists a unique $b$ such that $[a,b] \in S$. Let $f(s) = b$. \\    
\textit{Recursive Step:} Assume for the recursion, that $f(s)$ is defined for $|s| = n < k$ and that $\theta_{f}(s) \in S$. Consider $s' = s[a]$. By (\ref{eq:S2nd}), there exists a unique $b \in B$ such that $\chi' = \theta_{f}(s)[a,b]$. Let $f(s') = b$. \\      
$S \subseteq S_{f}$ For every $\chi$, we need to show that there exists an $s \in \efcomonad{k}{A}$ such that $\theta_{f}(s) = \chi$ for the $f$ constructed above. Consider $s = \pi_{1}(\chi)$. By construction, $\theta_{f}(s) = \theta_{f}(\pi_{1}(\chi)) = \chi$.
\end{proof}
\label{prop:posFormToFEF}
\end{prop}
\subsection{Equivalence $\exists^{+}\mathcal{L}_{\omega,k}$}
\begin{defn}
A position $\chi = [(a_{1},b_{1}),\dots,(a_{n},b_{n})] \in \Gamma_{k}(A,B)$ is \textit{winning for Duplicator in $\efgame{k}{A}{B}$} if the map $\chi:a_{i} \longmapsto b_{i}$ is a partial $\sigma$-morphism from $A$ to $B$.  
Naturally, we can extend the definition to say that a strategy in positional form $S \subseteq \Gamma_{k}(A,B)$ is \textit{winning for Duplicator in $\efgame{k}{A}{B}$} if for all $\chi \in S$, $\chi$ is winning for Duplicator. Function $f:\efcomonad{k}{A} \longrightarrow B$ is \textit{winning for Duplicator in $\efgame{k}{A}{B}$} if $S_{f}$ is winning for Duplicator in $\efgame{k}{A}{B}$.
\end{defn}
Consider the expanded signature $\sigma' = \sigma \cup \{I\}$ with $I$ a binary relation. In order to prove the theorem, it is necessary to lift $\sigma$-structures to $\sigma'$-structures. We say a $\sigma$-structure $A$ is a \textit{pure $\sigma'$-structure} if $I$ is interpreted as the identity relation on $A$. Note that if $A$ is a pure $\sigma'$-structure, then by the definition of $\efcomonad{k}{A}$ as a $\sigma'$-structure, $\efcomonad{k}{A}$ is not a pure $\sigma'$-structure. However, two prefix comparable plays in $\efcomonad{k}{A}$ are $I$-related if their last elements are the same. This ensures that $S_{f}$ contains well-defined partial functions.
\begin{thm}
If $A,B$ are pure $\sigma'$-structures and $f:\efcomonad{k}{A} \rightarrow B$ is a function, then 
\center{$f:\efcomonad{k}{A} \longrightarrow B$ is a $\sigma'$-morphism if and only if  $f$ is winning for Duplicator in $\efgame{k}{A}{B}$}
\begin{proof}
$\Rightarrow$ Suppose $\chi \in S_{f}$, then by definition of $S_{f}$, $\chi = \emptySeq$ or $\chi = \theta_{f}(s)$ for some $s \in \efcomonad{k}{A}$. If $\chi = \emptySeq$, then it is vacuously true that $\chi$ is a partial homomorphism (i.e. winning for Duplicator). If $\chi = \theta_{f}(s)$, then there exists some $s \in \efcomonad{k}{A}$ such that $s = [a_{1},\dots,a_{n}]$ where $f([a_{1},\dots,a_{i}]) = b_{i}$ and $\chi = [(a_{1},b_{1}),\dots,(a_{n},b_{n})]$ for all $i = 1,\dots,n$. For brevity, let $s_{i} = [a_{1},\dots,a_{i}] \sqsubseteq s$ (i.e. $s_{i}$ is the $i$-th component of $\delta_{A}(s)$). \\
Suppose $R \in \sigma$ is a $m$-ary relation symbol, and $i_{1},\dots,i_{m} \in \{1,\dots,n\}$ and $R^{A}(a_{i_{1}},\dots,a_{i_{m}})$, then:
\begin{align*}
(a_{i_{1}},\dots,a_{i_{m}}) \in R^{A} &\Rightarrow (\epsilon_{A}(s_{i_{1}}),\dots,\epsilon_{A}(s_{i_{m}})) \in R^{A} & \text{by defn of $\epsilon_{A}$} \\
\text{(and } s_{i_{\mu}} \sqsubseteq s_{i_{\nu}} \text{ or } s_{i_{\nu}} \sqsubseteq s_{i_{\mu}} &\text{ for all } \nu,\mu = 1,\dots,m\text{)} &\text{by each } s_{i_{*}} \sqsubseteq s \\
&\Rightarrow (s_{i_{1}},\dots,s_{i_{m}}) \in R^{\efcomonad{k}{A}} & \text{by interpretation of $R$ on $\efcomonad{k}{A}$} \\
&\Rightarrow (f(s_{i_{1}}),\dots,f(s_{i_{m}})) \in R^{B} & \text{by hypothesis $f$ is $\sigma$-morphism}\\
&\Rightarrow (b_{i_{1}},\dots,b_{i_{m}}) \in R^{B}
\end{align*}
Hence, $\chi$ preserves relations.
Moreover, for $a_{i}$ and $a_{j}$ appearing in $s$ with $a_{i} = a_{j}$, then:
\begin{align*}
a_{i} = a_{j}       &\Rightarrow (a_{i},a_{j}) \in I^{A} & \text{by $A$ a pure $\sigma'$-structure} \\
&\Rightarrow (\epsilon_{A}(s_{i}),\epsilon_{A}(s_{j})) \in I^{A} & \text{by defn of $\epsilon_{A}$} \\
\text{(and } s_{i} \sqsubseteq s_{j} &\text{ or } s_{j} \sqsubseteq s_{i}\text{)} &\text{by each } s_{*} \sqsubseteq s \\
&\Rightarrow (s_{i},s_{j}) \in I^{\efcomonad{k}{A}} & \text{by interpretation of $I$ on $\efcomonad{k}{A}$} \\
&\Rightarrow (f(s_{i}),f(s_{j})) \in I^{B} & \text{by hypothesis $f$ is a $\sigma'$-morphism} \\
&\Rightarrow (b_{i},b_{j}) \in I^{B}\\
&\Rightarrow b_{i} = b_{j} & \text{by $B$ a pure $\sigma'$-structure}
\end{align*}
Hence, $\chi$ is a well-defined partial function.
Therefore, $\chi$ is indeed a partial homomorphism. Hence, for all $\chi \in S_{f}$, $\chi$ is winning for Duplicator. Therefore, by definition, $S_{f}$ is winning for Duplicator.
\\~\\
$\Leftarrow$ Suppose $(s_{1},\dots,s_{m}) \in R^{\efcomonad{k}{A}}$ for $R \in \sigma'$ (including $I$), then by the interpretation of $R$ on $\efcomonad{k}{A}$, there exists some prefix order greatest element $s \in \{s_{1},\dots,s_{m}\}$ such that for all $i$, $s_{i} \sqsubseteq s$. Consider $\chi = \theta_{f}(s) \in S_{f}$. By $s_{i} \sqsubseteq s$ and definition of $\theta_{f}$, $(\epsilon_{A}(s_{1}),f(s_{1})),\dots, (\epsilon_{A}(s_{m}),f(s_{m}))$ appear in $\chi$. Since $(\epsilon_{A}(s_{1}),\dots,\epsilon_{A}(s_{m})) \in R^{A}$ and $\chi$ is a partial homomorphism (i.e. winning for Duplicator), $(f(s_{1}),\dots,f(s_{m})) \in R^{B}$. Hence, $f$ is a $\sigma'$-morphism.
\end{proof}
\label{thm:toPositionalFormEF}
\end{thm}
Once in terms of positional form, we can complete the proof that $\efcomonad{k}{}$ indeed captures equivalence in $\exists^{+}\mathcal{L}_{\omega,k}$. Essentially, using the positional form, we reproduce the forth part of the Ehrenfreucht-{\Fraisse} theorem conventional in many model-theoretic texts. In particular, the following proof is modification of the proof of \cite[Theorem 3.18]{Libkin2004}. 
\begin{prop}
For all $k \in \omega$, the following are equivalent:
\begin{enumerate}[label=(\arabic*)$_{k}$]
\item For every $\phi \in \exists^{+}\mathcal{L}_{\omega,k}$, $A \vDash \phi \Rightarrow B \vDash \phi$
\item There exists a strategy in positional form $S \subseteq \Gamma_{k}(A,B)$ that is winning for Duplicator 
\end{enumerate}
\begin{proof}
$\Rightarrow$ By recursion on $k$, we construct $S$. \\
\textit{Base Case:} If $k = 0$, then the let $S = \{\emptySeq\}$. The condition that $\emptySeq$ is winning for Duplicator, means $A$ and $B$ satisfy the same atomic sentences. \\
\textit{Inductive Step:} Assume for the inductive hypothesis that there exists a strategy in positional form $S_{k} \subseteq \Gamma_{k}(A,B)$ that is winning for Duplicator. By hypothesis, we may also assume that (1)$_{k+1}$ (i.e. that for all $\phi \in \exists^{+}\mathcal{L}_{\omega,k+1}$, $A \vDash \phi \Rightarrow B \vDash \phi$). For every $a \in A$, consider the set of formulas: 
$$\text{tp}^{k}_{1}(a) := \{\psi(x) \in \exists^{+}\mathcal{L}_{\omega,k}: A,a \vDash\psi(x) \} $$ 
In model theory literature, this set is called the (existential-positive) rank-$k$ $1$-type of $a \in A$. Since there are only finitely many equivalence classes of formulas of rank-$k$, this type is isolated. That is, there exists a formula $\alpha(x) \in \text{tp}^{k}_{1}(a)$ such that for all $\psi(x) \in \text{tp}^{k}_{1}(a)$, $A,a \vDash \alpha(x) \Rightarrow A,a \vDash \psi(x)$. Consider the sentence $\phi = \exists x \alpha(x)$, clearly $A \vDash \phi$ as $a$ witnesses $x$. Since $\alpha$ is a rank-$k$ type, the sentence $\phi \in \exists^{+}\mathcal{L}_{\omega,k+1}$. By hypothesis (i.e (1)$_{k+1}$) and $A \vDash \phi$, $B \vDash \phi$. Hence, there exists a $b \in B$ corresponding to $a$, such that $B,b \vDash \alpha(x)$.  Let $\chi'_{a} = \chi[a,b]$ for every $\chi \in S$, $a \in A$.  Define
$$S_{k+1} =  S_{k} \cup \{\chi'_{a} : \chi \in S_{k} \text{ and }  a \in A \} \subseteq \Gamma_{k+1}(A,B)$$
It is clear by construction and the inductive hypothesis, that $S_{k+1}$ satisfies (\ref{eq:S1st})-(\ref{eq:S3rd}), so $S_{k+1}$ is a strategy in positional form. By the definition of the $\alpha$, for every $a \in A$, there exist $b \in B$ corresponding to $a$ such that for every $\psi(x) \in \exists^{+}\mathcal{L}_{\omega,k}, A \vDash \psi(a) \Rightarrow B \vDash \psi(b)$. Hence, $S_{k+1}$ is indeed winning for Duplicator. \\~\\
$\Leftarrow$ By induction on complexity of formulas $\psi(\mathbf{x}) \in \exists^{+}\mathcal{L}_{\omega,k+1}$, we prove the stronger statement of (1)$_{k+1}$ where the sentence $\phi$ is replaced with the formula $\psi(\mathbf{x})$ with free variables among $\mathbf{x}$ and $n \leq k+1$. Let $S \subseteq \Gamma_{k+1}(A,B)$ be the strategy in positional form winning for Duplicator that exists by hypothesis (i.e. (2)$_{k+1}$). \\
\textit{Base Case:} Suppose $\psi(\mathbf{x})$ is an atomic formula with variables among $\mathbf{x} = (x_{1},\dots,x_{n})$ and $\mathbf{a} = (a_{1},\dots,a_{n})$, then there are two cases: \\
$x_{i} = x_{j}$: Suppose $A,\mathbf{a} \vDash x_{i} = x_{j}$. By (\ref{eq:S2nd}), there exists $\chi \in S$ with $\chi = [(a_{i},b_{i}),(a_{j},b_{j})]$. By $S$ winning for Duplicator, $a_{i} \mapsto b_{i}$ is a partial $\sigma$-morphism, so $B,\mathbf{b} \vDash x_{i} = x_{j}$\\
$R(x_{i_{1}},\dots,x_{i_{m}})$: Suppose $A,\mathbf{a} \vDash R(x_{i_{1}},\dots,x_{i_{n}})$, so $(a_{i_{1}},\dots,a_{i_{m}}) \in R^{A}$ By (\ref{eq:S2nd}), there exists $\chi \in S$ with $\chi = [(a_{1},b_{1}),\dots,(a_{n},b_{n})]$. By $S$ winning for Duplicator, $\chi:a_{i} \mapsto b_{i}$ is a partial $\sigma$-morphism. Hence, $(b_{i_{1}},\dots,b_{i_{m}}) \in R^{B}$, so $B,\mathbf{b} \vDash R(b_{i_{1}},\dots,b_{i_{n}})$. \\
\textit{Inductive Step:} Assume for the inductive hypothesis that given formulas $\psi_{1}(\mathbf{x})$, $\psi_{2}(\mathbf{x})$ in $\exists^{+}\mathcal{L}_{\omega,k+1}$ there exists $\mathbf{a} = (a_{1},\dots,a_{n}) \in A^{n}$ such that $A,\mathbf{a} \vDash \psi_{j}(\mathbf{x}) \Rightarrow B,\mathbf{b} \vDash \psi_{j}(\mathbf{x})$ for some $\chi = [(a_{1},b_{1}),\dots,(a_{n},b_{n})] = z((\mathbf{a},\mathbf{b})) \in S$. There are three cases. For the $\psi(\mathbf{x}) = \psi_{1}(\mathbf{x}) \wedge \psi_{2}(\mathbf{x})$ case: 
\begin{align*}
A,\mathbf{a} \vDash \psi(\mathbf{x}) &\Rightarrow A,\mathbf{a} \vDash \psi_{1}(\mathbf{x}) \wedge \psi_{2}(\mathbf{x}) \\
&\Rightarrow A,\mathbf{a} \vDash \psi_{1}(\mathbf{x}) \text{ and } A,\mathbf{a} \vDash \psi_{2}(\mathbf{x}) \\
&\Rightarrow B,\mathbf{b} \vDash \psi_{1}(\mathbf{x}) \text{ and } B,\mathbf{b} \vDash \psi_{2}(\mathbf{x}) & \text{by the inductive hypothesis}\\
&\Rightarrow B,\mathbf{b} \vDash \psi_{1}(\mathbf{x}) \wedge \psi_{2}(\mathbf{x}) \\
&\Rightarrow B,\mathbf{b} \vDash \psi(\mathbf{x}) 
\end{align*}
The $\vee$ case is handled similarly. For the $\exists$ case, suppose $\psi(\mathbf{x}) = \exists y \varphi(y,\mathbf{x})$ with $\varphi \in \exists^{+}\mathcal{L}_{\omega,k}$ with the same assumptions as $\psi_{1},\psi_{2}$ then:  
\begin{align*}
A,\mathbf{a} \vDash \psi(\mathbf{x}) &\Rightarrow A,\mathbf{a} \vDash \exists y \varphi(y,\mathbf{x}) \\ 
&\Rightarrow \text{ there is some } a' \in A \text{ such that } A,a'\mathbf{a} \vDash \varphi(y,\mathbf{x}) 
\end{align*}
By (\ref{eq:S2nd}) and $\chi \in S$, there exists $b' \in B$ such that $\chi' = \chi[a',b'] \in S$ winning for Duplicator (i.e a partial homomorphism). Hence,
\begin{align*}
A,\mathbf{a} \vDash \psi(\mathbf{x}) &\Rightarrow \text{ there is some } b' \in B \text{ such that } B,b'\mathbf{b} \vDash \varphi(y,\mathbf{x}) \\  
&\Rightarrow B,\mathbf{b} \vDash \exists y \varphi(y,\mathbf{x}) \\  
&\Rightarrow B,\mathbf{b} \vDash \psi(\mathbf{x})   
\end{align*}
Therefore, for all $\psi(\mathbf{x}) \in \exists^{+}\mathcal{L}_{\omega,k+1}$ with $n \leq k+1$, there exists $\mathbf{a} \in A^{n}$ and $\mathbf{b} \in B^{n}$ such that $A,\mathbf{a} \vDash \psi(\mathbf{x}) \Rightarrow B,\mathbf{b} \vDash \psi(\mathbf{x})$. In particular, for sentences $\phi \in \exists^{+}\mathcal{L}_{\omega,k+1}$, $A \vDash \phi \Rightarrow B \vDash \phi$. 
\end{proof}
\begin{cor}
There exists a $\sigma'$-morphism $f:\efcomonad{k}{A} \longrightarrow B$ if and only if for all $\phi \in \exists^{+}\mathcal{L}_{\omega,k}, A \vDash \phi \Rightarrow B \vDash \phi$
\begin{proof}
Straightforward from theorem (\ref{thm:toPositionalFormEF}) and proposition (\ref{prop:toSyntaxEF}).
\end{proof}
\label{cor:forthOneEF}
\end{cor}
\begin{cor}
There exists $\sigma'$-morphisms $f:\efcomonad{k}{A} \longrightarrow B$, $g:\efcomonad{k}{B} \longrightarrow B$ if and only if $\equivL{A}{B}{\exists^{+}\mathcal{L}_{\omega,k}}$. 
\begin{proof}
Recall definition (\ref{defn:equivLogic}) and apply above corollary (\ref{cor:forthOneEF}) to $f$ and $g$.  
\end{proof}
\label{cor:forthEF}
\end{cor}
\label{prop:toSyntaxEF}
\end{prop}
\subsection{Equivalence $\mathcal{L}_{\omega,k}$}
The comonad $\efcomonad{k}{}$ also captures equivalence $\equivL{A}{B}{\mathcal{L}_{\omega,k}}$ in the full $k$-rank fragment of first-order logic. Equivalence in this logic is given by a winning strategy for Duplicator in the symmetric $k$-round EF game $\efgameSym{k}{A}{B}$. To formalize this in terms of $\sigma$-morphisms, we need winning strategies for Duplicator in $\efgame{k}{A}{B}$ given by $\sigma'$-morphism $f:\efcomonad{k}{A} \longrightarrow B$, and $\efgame{k}{B}{A}$ given by $\sigma'$-morphism $g:\efcomonad{k}{B} \longrightarrow A$, that are `sufficiently compatible', in positional form.  That is, these morphisms must form a back-and-forth system of partial isomorphisms. More precisely, for every pair of pure $\sigma'$-structures $A,B$, the positional forms are posets $\Gamma_{k}(A,B)$ and $\Gamma_{k}(B,A)$ ordered by $\sqsubseteq$. There is a canonical `swap' monotone function $()^{-}:\Gamma_{k}(A,B) \longrightarrow \Gamma_{k}(B,A)$ given by $[(a_{i},b_{i})]^{-} = [(b_{i},a_{i})]$. The condition that $f$ and $g$ are compatible, in positional form, can be phrased as $S_{f}^{-} = S_{g}$. It follows that $S_{g}^{-} = S_{f}$. 
\begin{defn}
A strategy in positional form $S \subseteq \Gamma_{k}(A,B)$ is \textit{winning for Duplicator in $\efgameSym{k}{A}{B}$} if $S$ is winning for Duplicator in $\efgame{k}{A}{B}$ and $S^{-}$ is winning for Duplicator in $\efgame{k}{B}{A}$. Naturally, we can extend this definition to functions, $f:\efcomonad{k}{A} \longrightarrow B$ is \textit{winning for Duplicator in $\efgameSym{k}{A}{B}$} if $S_{f}$ is winning for Duplicator in $\efgameSym{k}{A}{B}$.
\end{defn}
It follows by (\ref{prop:posFormToFEF}), that if $f$ is winning for Duplicator in $\efgameSym{k}{A}{B}$, there exists a function $g:\efcomonad{k}{B} \longrightarrow A$ such that $S_{g}$ is winning for Duplicator in $\efgame{k}{B}{A}$ and $S_{f}^{-} = S_{g}$, so $g$ is winning for Duplicator in $\efgameSym{k}{A}{B}$. This is the right definition to characterize equivalence in $\mathcal{L}_{\omega,k}$. To show this result, we modify the proof for (\ref{prop:toSyntaxEF}).  
\begin{prop}
For all $k \in \omega$, the following are equivalent:
\begin{enumerate}[label=(\arabic*)$_{k}$]
\item $\equivL{A}{B}{\mathcal{L}_{\omega,k}}$, i.e. for every $\phi \in \mathcal{L}_{\omega,k}$, $A \vDash \phi \Leftrightarrow B \vDash \phi$
\item There exists a strategy in positional form $S \subseteq \Gamma_{k}(A,B)$ such that $S$ is winning for Duplicator in $\efgameSym{k}{A}{B}$ 
\end{enumerate}
\begin{proof}
$\Rightarrow$ Repeat the proof for the $\Rightarrow$-direction of proposition (\ref{prop:toSyntaxEF}), but consider the full rank-$k$ 1-type of $a \in A$:
$$\text{tp}^{k}_{1}(a) := \{\psi(x) \in \mathcal{L}_{\omega,k}: A,a \vDash \psi(x) \} $$ 
This type is still isolated by a formula $\alpha(x)$ and proceed with the proof using this $\alpha(x)$. \\
$\Leftarrow$ We repeat the proof for the $\Leftarrow$-direction of proposition (\ref{prop:toSyntaxEF}), but with the additional $\neg$ and $\forall$ cases and a modified base case. For the base case, by (\ref{eq:S2nd}) and $S,S^{-}$ winning for Duplicator there exists a partial isomorphism $\chi:a_{i} \mapsto b_{i}$ for $\mathbf{a} = (a_{1},\dots,a_{n})$ and $\mathbf{b} = (b_{1},\dots,b_{n})$. Suppose $R \in \sigma$, then: 
\begin{align*}
A,\mathbf{a} \vDash R(x_{i_{1}},\dots,x_{i_{m}}) &\Leftrightarrow (a_{i_{1}},\dots,a_{i_{m}}) \in R^{A} \\
&\Leftrightarrow (b_{i_{1}},\dots,b_{i_{m}}) \in R^{B} & \text{by $\chi$ a partial isomorphism} \\
&\Leftrightarrow B,\mathbf{b} \vDash R(x_{i_{1}},\dots,x_{i_{m}})
\end{align*}
The proof for equality $=$ is similar. For the $\neg$-case, assume for the inductive hypothesis that given formula $\psi(\mathbf{x})$ with free variables among $\mathbf{x}$ in $\mathcal{L}_{\omega,k+1}$, there exists $\mathbf{a} = (a_{1},\dots,a_{n}) \in A^{n}$ such that $A,\mathbf{a} \vDash \psi(\mathbf{x}) \Leftrightarrow B,\mathbf{b} \vDash \psi(\mathbf{x})$ for some $\chi = [(a_{1},b_{1}),\dots,(a_{n},b_{n})] \in S$ and $\mathbf{b} = (b_{1},\dots,b_{n})$, then:
\begin{align*}
A,\mathbf{a} \vDash \neg\psi(\mathbf{x}) &\Leftrightarrow A,\mathbf{a} \not\vDash \psi(\mathbf{x}) \\
&\Leftrightarrow B,\mathbf{b} \not\vDash \psi(\mathbf{x})  & \text{contrapositive of inductive hypothesis}\\
&\Leftrightarrow B,\mathbf{b} \vDash \neg\psi(\mathbf{x}) 
\end{align*}
For the $\forall$-case, suppose $b \in B$, then by (\ref{eq:S2nd}) and $\chi^{-} \in S^{-}$ winning for Duplicator, there exists an $a \in A$ such that $\chi:a \mapsto b$. 
\begin{align*}
A,\mathbf{a} \vDash \psi(\mathbf{x})  &\Rightarrow A,\mathbf{a} \vDash \forall y \varphi(y,\mathbf{x}) \\
&\Rightarrow \text{ for all $c \in A$, } A,c\mathbf{a} \vDash \varphi(y,\mathbf{x}) \\
\text{In particular, for $c = a$,} &\Rightarrow A,a\mathbf{a} \vDash \varphi(y,\mathbf{x})\\
&\Rightarrow B,b\mathbf{b} \vDash \varphi(y,\mathbf{x}) & \text{by inductive hypothesis}\\
\text{Since $b \in B$ was arbitrary,} &\Rightarrow \text{ for all $b \in B$ }, B,b\mathbf{b} \vDash \varphi(y,\mathbf{x}) \\
&\Rightarrow B,\mathbf{b} \vDash \forall y \varphi(y,\mathbf{x}) \\ 
&\Rightarrow B,\mathbf{b} \vDash \psi(\mathbf{x}) 
\end{align*}
The $\Leftarrow$-direction is similar. Therefore, for all $\psi(\mathbf{x}) \in \mathcal{L}_{\omega,k+1}$ with $n \leq k+1$, there exists $\mathbf{a} \in A^{n}$ and $\mathbf{b} \in B^{n}$ such that $A,\mathbf{a} \vDash \psi(\mathbf{x}) \Leftrightarrow B,\mathbf{b} \vDash \psi(\mathbf{x})$. In particular, for sentences $\phi \in \mathcal{L}_{\omega,k+1}$, $A \vDash \phi \Leftrightarrow B \vDash \phi$, so $\equivL{A}{B}{\mathcal{L}_{\omega,k+1}}$. 
\end{proof}
\begin{cor}
Suppose $A,B$ are pure $\sigma'$-structures, then the following are equivalent:
\begin{enumerate}[label=(\arabic*)]
\item $\equivL{A}{B}{\mathcal{L}_{\omega,k}}$ 
\item There exist $\sigma'$-morphisms $f:\efcomonad{k}{A} \longrightarrow B$, $g:\efcomonad{k}{B} \longrightarrow A$ such that $S_{f} = S_{g}^{-}$ 
\item There exists $f:\efcomonad{k}{A} \longrightarrow B$ winning for Duplicator in $\efgameSym{k}{A}{B}$ 
\end{enumerate}
\begin{proof}
$(1) \Rightarrow (2)$ By proposition (\ref{prop:backForthSyntaxEF}), there exists a strategy in positional form $S$ winning for Duplicator in $\efgameSym{k}{A}{B}$. Hence, $S$ winning for Duplicator in $\efgame{k}{A}{B}$ and $S^{-}$ winning for Duplicator in $\efgame{k}{B}{A}$. By theorem (\ref{thm:toPositionalFormEF}), there exists $\sigma'$-morphisms $f:\efcomonad{k}{A} \longrightarrow B$, $g:\efcomonad{k}{B} \longrightarrow A$ with $S_{f} = S$, and $S^{-} = S_{g}$. \\ 
$(2) \Rightarrow (3)$ Straightforward from definition of $f$ winning for Duplicator in $\efgameSym{k}{A}{B}$. \\ 
$(3) \Rightarrow (1)$ By proposition (\ref{prop:backForthSyntaxEF}) with $S_{f}$ witnessing $S$.  
\end{proof}
\label{cor:backForthEF}
\end{cor}
\label{prop:backForthSyntaxEF}
\end{prop}
\subsection{Equivalence $\mathcal{L}_{\infty,k}(\mathsf{Cnt})$} 
The $k$-round bijective Ehrenfreucht-{\Fraisse} game between two pure $\sigma'$-structures $A,B$, $\efgameBij{k}{A}{B}$ is played by Spoiler and Duplicator as follows:
\begin{itemize}
\item If cardinality of $A$ is not equal to the cardinality of $B$, then Spoiler wins the game.
\item If cardinality of $A$ is equal to the cardinality of $B$, then for each round $i = 1,\dots,k$, Duplicator selects a bijection $\psi_{i}:A \longrightarrow B$ and Spoiler selects a point $a_{i} \in A$. 
\item Duplicator wins if for $b_{i} = \psi_{i}(a_{i})$, the map $\chi:a_{i} \mapsto b_{i}$ is a partial isomorphism from $A$ to $B$. 
\end{itemize}
The game $\efgameBij{k}{A}{B}$ characterizes equivalence in the $k$-rank fragment of $\mathcal{L}_{\infty,\omega}(\mathsf{Cnt})$. Namely,
\begin{prop}[{\cite{Hella1996}}]
The following are equivalent: 
\begin{itemize}
\item Duplicator has a winning strategy in $\efgameBij{k}{A}{B}$ 
\item $\equivL{A}{B}{\mathcal{L}_{\infty,k}(\mathsf{Cnt})}$ 
\end{itemize}
\label{prop:bijToSyntaxEF}
\end{prop}
Our goal is to show that isomorphism in the coKleisli category internalizes $\efgameBij{k}{A}{B}$. Suppose $f:\efcomonad{k}{A} \longrightarrow B$ is a $\sigma$-morphism. For all $s \in \efcomonad{k}{A} \cup \{\emptySeq\}$, define the function $\psi_{s}:A \longrightarrow B$ by $a \mapsto f(s[a])$. Using these functions and the following definition we can show that isomorphism in the coKleisli category internalizes $\efgameBij{k}{A}{B}$ and characterizes equivalence in $\mathcal{L}_{\infty,k}(\mathsf{Cnt})$.
\begin{defn}
Given $f:\efcomonad{k}{A} \longrightarrow B$, $f$ is \textit{winning for Duplicator in $\efgameBij{k}{A}{B}$} if $\psi_{s}$ bijective for all $s \in \efcomonad{k}{A} \cup \{\emptySeq\}$ 
\end{defn}
\begin{prop}
$f:\efcomonad{k}{A} \longrightarrow B$ an isomorphism in $\mathcal{K}(\efcomonad{k}{})$ if and only if $f$ is winning for Duplicator in $\efgameBij{k}{A}{B}$  
\begin{proof}
$\Rightarrow$ If $f:\efcomonad{k}{A} \longrightarrow B$ is an isomorphism in $\mathcal{K}(\efcomonad{k}{})$, then there exists a $g:\efcomonad{k}{B} \longrightarrow A$ such that $f \circ_{\mathcal{K}} g = \epsilon_{B}$ and $g \circ_{\mathcal{K}} f = \epsilon_{A}$. Suppose $s \in \efcomonad{k}{A} \cup \{\emptySeq\}$, then we aim to show that $\psi_{s}:A \longrightarrow B$ is bijective. Let $t = f^{*}(s)$ (or $t = \emptySeq$ if $s = \emptySeq$) and define $\phi_{t}:B \longrightarrow A$ by $\phi_{t}(b) = g(t[b])$.   
\begin{align*}
\phi_{t} \circ \psi_{s}(a) &= \phi_{t}(f(s[a])) \\
&= g(t[f(s[a]])) \\
&= g(f^{*}(s)[f(s[a]])) \\
&= g(f^{*}(s[a])) \\
&= g \circ_{\mathcal{K}} f(s[a]) \\
&= \epsilon_{A}(s[a]) \\
&= a 
\end{align*}
The proof that $\psi_{s} \circ \phi_{t}(b) = b$ is similar. Therefore, $\psi_{s}$ is bijective. \\
$\Leftarrow$ Let $f:\efcomonad{k}{A} \longrightarrow B$ be the $\sigma$-morphism witnessing Duplicator winning in $\efgameBij{k}{A}{B}$. Define $g^{*}:\efcomonad{k}{B} \longrightarrow \efcomonad{k}{A}$ (i.e. the coextension of $g = \epsilon_{A} \circ g^{*}$) by recursion, up to $k$, on length of $t \in \efcomonad{k}{B}$. \\   
\textit{Base Case:} Suppose $t = [b]$ for $b \in B$, let $g^{*}([b]) = [\psi_{\emptySeq}^{-1}(b)]$.
\textit{Recursion Step:} Assume for $t \in B$ with length $|t| = n < k$ that $g^{*}(t)$ is already defined. Consider $t' = t[b]$ for $b \in B$. Let $g^{*}(t') = g^{*}(t)\psi_{s}^{-1}(b)$ where $s = g^{*}(t) \in \efcomonad{k}{A}$. To complete, the proof, we need to show that $f \circ_{\mathcal{K}} g = \epsilon_{B}$ and $g \circ_{\mathcal{K}} f = \epsilon_{A}$. Suppose $s' = s[a_{n}] = [a_{1},\dots,a_{n}] \in \efcomonad{k}{A}$ and $f(s') = b_{n}$ (i.e. it follows that $\psi_{s}(a_{n}) = b_{n}$) then: 
\begin{align*}
g \circ_{\mathcal{K}} f([a_{1},\dots,a_{n}]) &= g \circ \efcomonad{k}{f} \circ \delta_{A}([a_{1},\dots,a_{n}]) \\ 
&= g \circ \efcomonad{k}{f}([[a_{1}],\dots,[a_{1},\dots,a_{n}]]) \\
&= g(f([a_{1}]),\dots,f([a_{1},\dots,a_{n}])) \\
&= g([b_{1},\dots,b_{n}]) \\
&= \psi_{s}^{-1}(b_{n})  \\
&= a_{n} \\ 
&= \epsilon_{A}(s[a_{n}]) 
\end{align*}
The proof $f \circ_{\mathcal{K}} g = \epsilon_{B}$ is similar. By definition, $f: \efcomonad{k}{A} \longrightarrow B$ is an isomorphism in the coKleisli category.  
\end{proof}
\begin{cor}
$A,B$ are isomorphic in $\mathcal{K}(\efcomonad{k}{})$ if and only if $\equivL{A}{B}{\mathcal{L}_{\infty,k}(\mathsf{Cnt})}$.
\begin{proof}
Follows from (\ref{prop:bijToSyntaxEF}) and (\ref{prop:bijEF})
\end{proof}
\end{cor}
\label{prop:bijEF}
\end{prop}
\section{Coalgebras and Tree-Depth}
An advantage of the comonad perspective is exploring the structure of the category of coalgebras over $\efcomonad{k}{}$ to give a categorical characterization of combinatorial properties of structures. In particular, we use the coalgebra category of $\efcomonad{k}{}$ to give a categorical definition for tree-depth of a $\sigma$-structure $A$. Recall a coalgebra for $\efcomonad{k}{}$ is an object $A$ and morphism $\alpha:A \longrightarrow \efcomonad{k}{A}$ such that the following diagrams commute:
\begin{equation}
\bfig 
    \square[A`\efcomonad{k}{A}`\efcomonad{k}{A}`\efcomonad{k}{\efcomonad{k}{A}};\alpha`\alpha`\delta_{A}`\efcomonad{k}{\alpha}]
\efig 
\bfig
    \qtriangle[A`\efcomonad{k}{A}`A;\alpha`\mathsf{id}_{A}`\epsilon_{A}]
\efig
\label{eq:coalgebraLawEF}
\end{equation}
There are many equivalent ways to define the notion of tree-depth of a $\sigma$-structure $A$. We use the one given in \cite{Rossman2008} and is stated in definition (\ref{defn:treeDepth}). 
\begin{defn}
A \textit{forest} $\mathcal{F}$ is a disjoint union of finitely-many finite rooted trees. The \textit{height} of the forest is the longest path between two vertices in $\mathcal{F}$
\end{defn}
\begin{defn}
Given an undirected graph $G$, a \textit{forest cover of $G$} is a forest $\mathcal{F}$ such that $G$ is a subgraph of $\overline{\mathcal{F}}$ where $\overline{\mathcal{F}}$ is the transitive closure of $\mathcal{F}$
\end{defn}
Recall the definition (\ref{defn:gaifmanGraph}) of the Gaifman graph $\mathcal{G}(A)$ of a $\sigma$-structure $A$.
\begin{defn}
Given a $\sigma$-structure $A$, the \textit{tree-depth} of $A$, denoted $\mathsf{td}(A)$, is the least height of a forest cover $\mathcal{F}$ for $\mathcal{G}(A)$. 
\label{defn:treeDepth}
\end{defn}
\begin{prop}
Let $A$ be a finite $\sigma$-structure. There is a bijective correspondence between:
\begin{enumerate}[label=(\arabic*)]
\item coalgebras $\alpha:A \longrightarrow \efcomonad{k}{A}$
\item Forest covers $\mathcal{F}$ of $\mathcal{G}(A)$ with height $k$ 
\end{enumerate}
\begin{proof}
For the $(1) \Rightarrow (2)$ correspondence, suppose $\alpha:A \rightarrow \efcomonad{k}{A}$ is a coalgebra. Construct a forest $\mathcal{F}_{n}$ for every $n \leq k$ by recursion. Let $V(\mathcal{F})$ and $E(\mathcal{F})$ denote the vertices and edge relation of forest $\mathcal{F}$. \\
\textit{Base Case:} $$V(\mathcal{F}_{1}) := \{a \in A \mid \alpha(a) = [a]\}, E(\mathcal{F}_{1}) = \varnothing$$
\textit{Recursive Step:} Assume that $\mathcal{F}_{n}$ is a forest that has been constructed. From the counit coalgebra law (\ref{eq:coalgebraLaw} right), it must be the case that $\alpha(a) = s$ where the last move in $s$ is $a$ (i.e. $\epsilon_{A}(s) = a$). Let $a^{-}$ be the second-to-last move in $s$ (i.e $s = t[a^{-},a] = t[a^{-},\epsilon_{A}(s)]$ for some possibly empty sequence $t$). Define $\mathcal{F}_{n+1}$ as follows:  
$$V(\mathcal{F}_{n+1}) := V(\mathcal{F}_{n}) \cup \{a \in A \mid a^{-} \in V(\mathcal{F}_{n})\}$$
$$E(\mathcal{F}_{n+1}) := E(\mathcal{F}_{n}) \cup \{(a^{-},a) \mid a \in  V(\mathcal{F}_{n+1})\}$$ 
Let $\mathcal{F} = \mathcal{F}_{k}$ be the forest constructed at the end of the recursion up to $k$. Suppose $(a,a')$ is an edge in $\mathcal{G}(A)$, the by the definition of Gaifman graph there exists an $R \in \sigma$ such that $a$ and $a'$ appear in a tuple of $R^{A}$. By $\alpha$ a $\sigma$-morphism, $\alpha(a)$ and $\alpha(a')$ appear in a tuple of $R^{\efcomonad{k}{A}}$. By condition (1) in the definition $R^{\efcomonad{k}{A}}$, $\alpha(a) \sqsubseteq \alpha(a')$ or $\alpha(a') \sqsubseteq \alpha(a)$. Without loss of generality, assume $\alpha(a) \sqsubseteq \alpha(a')$, and let $t$ be the suffix of $\alpha(a)$ in $\alpha(a')$. From the construction of $\mathcal{F}$, $[a]t$ describes a path between $a$ and $a'$ in $\mathcal{F}$. Hence, $(a,a') \in \overline{\mathcal{F}}$. Since $(a,a')$ was arbitrary in $\mathcal{G}(A)$, $\mathcal{G}(A)$ is a subgraph of $\overline{\mathcal{F}}$. Therefore, $\mathcal{F}$ is a forest cover of $\mathcal{G}(A)$ as desired. \\~\\
For the $(2) \Rightarrow (1)$ correspondence, suppose $\mathcal{F}$ is a forest cover of $\mathcal{G}(A)$ of height $k$. Consider arbitrary $a \in A$. Since $\mathcal{F}$ is a cover of $\mathcal{G}(A)$, $a \in \mathcal{F}$. By definition of forest as a disjoint union finite trees, $a$ is in a unique tree $T \subseteq \mathcal{F}$. Let $a^{*}$ be the root of tree $T$. By $T$ being a tree, there exists a unique path $u = [a^{*},\dots,a]$ between $a^{*}$ and $a$. Let $\alpha(a) = u$ with $u$ considered as an element of $\efcomonad{k}{A}$. It is straightforward to see that $\alpha$ as defined satisfies the coalgebra laws (\ref{eq:coalgebraLaw}). 
\end{proof}
\label{prop:coalgebraForest}
\begin{defn}
Given a $\sigma$-structure $A$, define the \textit{$\mathbb{E}$-coalgebra number of $A$}, denoted $\kappa(A)$, as the least $k$ such that there exists a coalgebra $\alpha:A \longrightarrow \efcomonad{k}{A}$. 
\end{defn}
\begin{cor}
$\kappa(A) = \mathsf{td}(A)$
\begin{proof}
By definition of coalgebra number $\kappa(A)$ there exists a coalgebra $\alpha:A \longrightarrow \efcomonad{\kappa(A)}{A}$. By proposition (\ref{prop:coalgebraForest}), there exists a corresponding forest cover $\mathcal{F}$ of $\mathcal{G}(A)$. The height of $\mathcal{F}$ is $\kappa(A)$. Hence, $\mathsf{td}(A) \leq \kappa(A)$ by definition of tree-depth as the least height of forest covers. Similarly, by definition of tree-depth $\mathsf{td}(A)$, there exists a forest cover $\mathcal{F}$ of $\mathcal{G}(A)$ of height $k = \mathsf{td}(A)$. By proposition (\ref{prop:coalgebraForest}), there exists a corresponding coalgebra $\alpha:A \longrightarrow \efcomonad{k}{A}$. Hence, $\kappa(A) \leq k$ by definition of $\kappa(A)$. Therefore, $\mathsf{td}(A) \leq \kappa(A)$ and $\mathsf{td}(A) \geq \kappa(A)$, so $\kappa(A) = \mathsf{td}(A)$.      
\end{proof}
\label{cor:treeDepth}
\end{cor}
\end{prop}
Hence, as the above corollary (\ref{cor:treeDepth}) shows, the graded family of comonads $\{\efcomonad{k}\}_{k \in \omega}$ gives a purely categorical definition of tree-depth of a structure $A$. Namely, the tree-depth of $A$ is just the least $k$ such that a coalgebra $\alpha:A \longrightarrow \efcomonad{k}{A}$ exists. 

\chapter{Pebbling Comonad}
\section{Introduction}
The Pebbling game, similar in structure to the Ehrenfreuct-{\Fraisse} game, is used to prove equivalence between structures in the $k$-variable fragments of infinitary logic. Given two structures $A,B$, the $k$-pebble game is played with Spoiler and Duplicator both having a set of $k$-pebbles $[k] = \{1,\dots,k\}$.  
\begin{itemize} 
\item Spoiler places one of his pebbles $p \in [k]$ on an element in either strucure $a \in A$ or $b \in B$. If pebble $p$ is already placed on a different element, then Spoiler removes the pebble from this element and places the pebble on the new element he chose.  
\item Duplicator places her corresponding pebble $p \in [k]$ on an element in the other structure $b \in B$ or $a \in A$. Just as with Spoiler, Duplicator may have to move her pebble from a previously pebbled element.
\end{itemize} 
The game is played for $\omega$ many rounds. At the end of the game, $(a_{1},\dots,a_{k})$ and $(b_{1},\dots,b_{k})$ are the pebbled elements. Duplicator wins the $k$-pebble game if the map $\gamma:a_{p} \longmapsto b_{p}$ is a partial $\sigma$-isomorphism from $A$ to $B$. Otherwise, Spoiler wins. Just as with the case with EF games, the asymmetric (or existential positive) game from $A$ to $B$, is the same game with the additional restriction that Spoiler must always pebble elements in $A$ and that the map obtained is a partial $\sigma$-morphism. The following result holds:
\begin{prop}
The following are equivalent:
\begin{itemize}
\item Duplicator has a winning strategy in the symmetric $k$-pebble game 
\item $A \equiv^{\mathcal{L}^{k}_{\infty,\omega}} B$, i.e. for every $\phi \in \mathcal{L}^{k}_{\infty,\omega}$, $A \vDash \phi \Leftrightarrow B \vDash \phi$
\end{itemize}
\end{prop}
As was the goal with the EF game comonad, our goal is to construct a $\sigma$-structure $\pcomonad{k}{A}$ from a $\sigma$-structure $A$, that ``internalizes'' the asymmetric and symmetric $k$-pebble games in the category $\mathcal{R}(\sigma)$. The construction of $\pcomonad{k}{}$ and all of the results in this chapter were originally discovered in (\cite{Abramsky2017}). The paper \cite{Abramsky2017} was the inspiration to develop the other game comonads in this dissertation. We reproduce the proofs here to demonstrate the connections and differences with these other comonads.  
\section{Comonad laws}
Let $A$ be a $\sigma$-structure over relational signature $\sigma$, then for every $k \in \mathbb{N}$ we define a $\sigma$-structure $\pcomonad{k}{A}$. Intuitively, $\pcomonad{k}{A}$ is the structure of the finite $k$-pebblings on $A$. Let the domain of this structure be $|\pcomonad{k}{A}| = (\{1,\dots,k\} \times A)^{<\omega}$ (i.e. the set of finite sequence of elements in product $\{1,\dots,k\} \times A$). In order, to define the $\sigma$-structure on $\pcomonad{k}{A}$, a bit more notation is necessary:\\
\begin{defn}
Suppose $s,s' \in |\pcomonad{k}{A}|$ such that $s = [s_{1},\dots,s_{n}]$ and $s' = [s'_{1},\dots,s'_{m}]$, then define $ss' = [s_{1},
\dots,s_{n},s'_{1},\dots,s'_{m}]$ (i.e. the concatenation of these two sequences).
\end{defn}
\begin{defn}
For $s,t \in |\pcomonad{k}{A}|$, say $s \sqsubseteq t$ if there exists an $s' \in |\pcomonad{k}{A}|$ such that $ss' = t$; such an $s'$ is called the suffix of $s$ in $t$. $\sqsubseteq$ preorders $|\pcomonad{k}{A}|$. 
\end{defn}
\begin{defn}
Define, for every $\sigma$-structure $A$, $\epsilon_{A}:\pcomonad{k}{A} \longrightarrow A$ by $[(p_{1},a_{1}),\dots,(p_{n},a_{n})] \mapsto a_{n}$ (i.e. the element of the last move of the play). Further, define $\pi_{A}:\pcomonad{k}{A} \longrightarrow \{1,\dots,k\}$ by $[(p_{1},a_{1}),\dots,(p_{n},a_{n})] \mapsto p_{n}$ (i.e. the pebble index of the last move of the play).
\label{defn:epsilon}
\end{defn}
With these definitions in place, we can define the $\sigma$-structure on $\pcomonad{k}{A}$. Suppose $R \in \sigma$ is an $m$-ary relation, then we define the interpretation $R^{\pcomonad{k}{A}}$ such that for every $s_{1},\dots,s_{m} \in |\pcomonad{k}{A}|$, 
\begin{align}
(s_{1},\dots,s_{m}) \in R^{\pcomonad{k}{A}}  &\Leftrightarrow    \text{ for every $i,j$, } s_{i} \sqsubseteq s_{j} \text{ or } s_{j} \sqsubseteq s_{i} \text{(i.e. exists $\sqsubseteq$-greatest $s_{*}$)} & \label{eq:R1st}\\
&\text{ and for every $i$, } \pi_{A}(s_{i}) \text{ does not appear in suffix of $s_{i}$ in $s_{*}$ } & \label{eq:R2nd} \\
&\text{ and} R^{A}(\epsilon_{A}(s_{1}),\dots,\epsilon_{A}(s_{m})) & \label{eq:R3rd}
\end{align}
The definition of $\pcomonad{k}{A}$ can be extended to morphisms of $\sigma$-structures. 
\begin{defn}
Given a morphism $f:A \longrightarrow B$, define the morphism $\pcomonad{k}{f}:\pcomonad{k}{A} \longrightarrow \pcomonad{k}{B}$ by $[(p_{1},a_{1}),\dots,(p_{n},a_{n})] \mapsto [(p_{1},f(a_{1})),\dots,(p_{n},f(a_{n}))]$
\label{defn:comonadMorphism}
\end{defn}
\begin{prop}
The definition (\ref{defn:comonadMorphism}) of $\pcomonad{k}{f}:\pcomonad{k}{A} \longrightarrow \pcomonad{k}{B}$ given above is indeed a morphism of $\sigma$-structures. 
\begin{proof}
Suppose $R \in \sigma$, then we want to show that if $(s_{1},\dots,s_{m}) \in R^{\pcomonad{k}{A}}$, then \linebreak $(\pcomonad{k}{f}(s_{1}),\dots,\pcomonad{k}{f}(s_{m})) \in R^{\pcomonad{k}{B}}$. For brevity, assume that $R$ is a binary relation (the proof for a general $m$-ary relation is a straightforward generalization). Suppose $s,s' \in \pcomonad{k}{A}$ such that $(s,s') \in R^{\pcomonad{k}{A}}$. Let $s = [(p_{1},a_{1}),\dots,(p_{n},a_{n})]$ and $s' = [(q_{1},a_{1}),\dots,(q_{m},a'_{m})]$. We aim to show that $(\pcomonad{k}{f}(s),\pcomonad{k}{f}(s')) \in R^{\pcomonad{k}{B}}$ \\
\begin{enumerate}
\item  Since $(s,s') \in R^{\pcomonad{k}{A}}$, by condition (\ref{eq:R1st}), $s \sqsubseteq s'$ or $s' \sqsubseteq s$. Without loss of generality, assume $s \sqsubseteq s'$. Since $s \sqsubseteq s'$.
$$s' = [(p_{1},a_{1}),\dots,(p_{n},a_{n}),(q_{n+1},a'_{n+1}),\dots,(q_{m},a'_{m})]$$ 
(noting that for $i \leq n$, $p_{i} = q_{i}$ and $a_{i} = a'_{i}$). Therefore $$\pcomonad{k}{f}(s) = [(p_{1},f(a_{1})),\dots,(p_{n},f(a_{n}))]$$ 
and 
$$\pcomonad{k}{f}(s') =[(p_{1},f(a_{1})),\dots,(p_{n},f(a_{n})),(q_{n+1},f(a'_{n+1})), \dots,(q_{m},f(a'_{m}))]$$ 
Hence, $\pcomonad{k}{f}(s) \sqsubseteq \pcomonad{k}{f}(s')$ and (\ref{eq:R1st}) is satisfied. 
\item By condition (\ref{eq:R2nd}) and $(s,s') \in R^{\pcomonad{k}{f}}$, for $n < i \leq m$, $p_{n} \not= q_{i}$. Since $\pcomonad{k}{f}$ does not affect pebble indices, (\ref{eq:R2nd}) is satisfied.
\item  By condition (\ref{eq:R3rd}) and $(s,s') \in R^{\pcomonad{k}{f}}$, $(\epsilon_{A}(s),\epsilon_{A}(s')) = (a_{n},a'_{m}) \in R^{A}$. Since $f:A \rightarrow B$ is a morphism of $\sigma$-structures, $(f(a_{n}),f(a'_{m})) \in R^{B}$. That is, $(\epsilon_{B}\circ \pcomonad{k}{f}(s),\epsilon_{B} \circ \pcomonad{k}{f}(s')) \in R^{B}$. Hence, (\ref{eq:R3rd}) is satisfied.
\end{enumerate}
Therefore, $(\pcomonad{k}{f}(s),\pcomonad{k}{f}(s')) \in R^{\pcomonad{k}{B}}$ and $\pcomonad{k}{f}$ is indeed a morphism of $\sigma$-structures. 
\end{proof}
\end{prop}
\begin{prop}
$\epsilon:\pcomonad{k}{} \longrightarrow 1_{\mathcal{R}(\sigma)}$ is a natural transformation.
\begin{proof}
For every $A,B \in \mathcal{R}(\sigma)$ we want to show that:
\begin{equation}
\bfig \square[\pcomonad{k}{A}`A`\pcomonad{k}{B}`B;\epsilon_{A}`\pcomonad{k}{f}`f`\epsilon_{B}] \efig
\label{eq:epsilonN}
\end{equation}
\begin{align*}
f \circ \epsilon_{A}([(p_{1},a_{1}),\dots,(p_{n},a_{n})])   &= f(a_{n}) & \text{by defn (\ref{defn:epsilon}) of $\epsilon_{A}$}\\
&= \epsilon_{B}([(p_{1},f(a_{1})),\dots,(p_{n},f(a_{n}))]) & \text{by defn (\ref{defn:epsilon}) of $\epsilon_{B}$}\\
&= \epsilon_{B} \circ \pcomonad{k}{f}([(p_{1},a_{1}),\dots,(p_{n},a_{n})]) & \text{by defn (\ref{defn:comonadMorphism}) of $\pcomonad{k}{f}$}
\end{align*}
Hence, the above diagram (\ref{eq:epsilonN}) commutes.
\end{proof}
\label{prop:epsilonN}
\end{prop}
\begin{defn}
Suppose $s \in \pcomonad{k}{A}$, then $s = [(p_{1},a_{1}),\dots,(p_{n},a_{n})]$ for some $n \in \omega$ and for every $i = 1,\dots, n$, $p_{i} \in \{1,\dots,k\}$, $a_{i} \in A$. Let $s_{i} = [(p_{1},a_{1}),\dots,(p_{i},a_{i})] \in \pcomonad{k}{A}$. Define, for every $\sigma$-structure $A$, $\delta_{A}:\pcomonad{k}{A} \longrightarrow \pcomonad{k}{\pcomonad{k}{A}}$ by $s \mapsto [(p_{1},s_{1}),\dots,(p_{n},s_{n})]$.
\label{defn:delta}
\end{defn}
\begin{prop}
$\delta:\pcomonad{k} \longrightarrow \pcomonad{k}{\pcomonad{k}{}}$ is a natural transformation.
\begin{proof}
For every $A,B \in \mathcal{R}(\sigma)$ we want to show that:
\begin{equation}
\bfig \square[\pcomonad{k}{A}`\pcomonad{k}{\pcomonad{k}{A}}`\pcomonad{k}{B}`\pcomonad{k}{\pcomonad{k}{B}};\delta_{A}`\pcomonad{k}{f}`\pcomonad{k}{\pcomonad{k}{f}}`\delta_{B}] \efig
\label{eq:deltaN}
\end{equation}
\begin{alignat*}{2}
\pcomonad{k}{\pcomonad{k}{f}} \circ \delta_{A}([(p_{1},a_{1}),\dots,(p_{n},a_{n})])   &= \pcomonad{k}{\pcomonad{k}{f}}([(p_{1},s_{1}),\dots,(p_{n},s_{n})]) & \text{by defn (\ref{defn:delta}) of $\delta_{A}$} \\
&= [(p_{1},\pcomonad{k}{f}(s_{1})),\dots,(p_{n},\pcomonad{k}{f}(s_{n}))] & \text{by defn (\ref{defn:comonadMorphism}) of $\pcomonad{k}{\pcomonad{k}{f}}$}\\
&= [(p_{1},[(p_{1},f(a_{1}))]),\dots,(p_{n},[(p_{1},f(a_{1})),\dots,(p_{n},f(a_{n}))])] & \text{by defn (\ref{defn:comonadMorphism}) of $\pcomonad{k}{f}$}\\
&= \delta_{B}([(p_{1},f(a_{1})),\dots,(p_{n},f(a_{n}))]) & \text{by defn (\ref{defn:delta}) of $\delta_{B}$} \\
&= \delta_{B} \circ \pcomonad{k}{f}([(p_{1},a_{1}),\dots,(p_{n},a_{n})]) & \text{by defn (\ref{defn:comonadMorphism}) of $\pcomonad{k}{f}$}
\end{alignat*}
Hence, the above diagram (\ref{eq:deltaN}) commutes.
\end{proof}
\label{prop:deltaN}
\end{prop}
\begin{thm}
The triple $\langle \pcomonad{k}{},\delta,\epsilon \rangle$ is a comonad.
\begin{proof}
By proposition (\ref{prop:deltaN}) and (\ref{prop:epsilonN}), $\delta$ and $\epsilon$ are natural transformation. Hence, $\delta$ and $\epsilon$ are indeed the comultiplication and counit of $\pcomonad{k}{}$. The associative and identity laws remain to be shown. \\
For associativity, for every $A \in \mathcal{R}(\sigma)$, the following diagram commutes:  
\begin{equation}
\bfig \Square[\pcomonad{k}{A}`\pcomonad{k}{\pcomonad{k}{A}}`\pcomonad{k}{\pcomonad{k}{A}}`\pcomonad{k}{\pcomonad{k}{\pcomonad{k}{A}}};\delta_{A}`\delta_{A}`\delta_{\pcomonad{k}{A}}`\pcomonad{k}{\delta_{A}}] \efig 
\end{equation}
\begin{center}
\begin{alignat*}{2}
\delta_{\pcomonad{k}{A}} \circ \delta_{A}([(p_{1},a_{1}),\dots,(p_{n},a_{n})])   &= \delta_{\pcomonad{k}{A}}([(p_{1},s_{1}),\dots,(p_{n},s_{n})]) & \text{by defn (\ref{defn:delta}) of $\delta_{A}$} \\
&= [(p_{1},[(p_{1},s_{1})]),\dots,(p_{n},[(p_{1},s_{1}),\dots,(p_{n},s_{n})])]  & \text{by defn (\ref{defn:delta}) of $\delta_{\pcomonad{k}{A}}$}\\
&= [(p_{1},\delta_{A}(s_{1})),\dots,(p_{n},\delta_{A}(s_{n}))] & \text{by defn (\ref{defn:delta}) of $\delta_{A}$}  \\
&= \pcomonad{k}{\delta_{A}}([(p_{1},s_{1}),\dots,(p_{n},s_{n})]) & \text{by defn (\ref{defn:comonadMorphism}) of $\pcomonad{k}{\delta_{A}}$}  \\
&= \pcomonad{k}{\delta_{A}} \circ \delta_{A}([(p_{1},a_{1}),\dots,(p_{n},a_{n})]) & \text{by defn (\ref{defn:delta}) of $\delta_{A}$}\\
\end{alignat*}
\end{center}
For identity, for every $A \in \mathcal{R}(\sigma)$, the following diagram commutes:  
\begin{equation}
\bfig 
    \square[\pcomonad{k}{A}`\pcomonad{k}{\pcomonad{k}{A}}`\pcomonad{k}{\pcomonad{k}{A}}`\pcomonad{k}{A};\delta_{A}`\delta_{A}`\pcomonad{k}{\epsilon_{A}}`\epsilon_{\pcomonad{k}{A}}] 
    \morphism(0,500)/=/<500,-500>[\pcomonad{k}{A}`\pcomonad{k}{A};]
\efig 
\end{equation}
\begin{align*}
\pcomonad{k}{\epsilon_{A}} \circ \delta_{A}([(p_{1},a_{1}),\dots,(p_{n},a_{n})]) &= \pcomonad{k}{\epsilon_{A}}([(p_{1},s_{1}),\dots,(p_{n},s_{n})]) & \text{by defn (\ref{defn:delta}) of $\delta_{A}$}\\
&= [(p_{1},\epsilon_{A}(s_{1})),\dots,(p_{n},\epsilon_{A}(s_{n}))]  & \text{by defn (\ref{defn:comonadMorphism}) of $\pcomonad{k}{\epsilon_{A}}$}  \\
&= [(p_{1},a_{1}),\dots,(p_{n},a_{n})] & \text{by defn (\ref{defn:epsilon}) of $\epsilon_{A}$}\\
\hspace{1pt}\\
&= s_{n} & \text{by defn (\ref{defn:delta}) of $s_{n}$}\\
&= \epsilon_{\pcomonad{k}{A}}([(p_{1},s_{1}),\dots,(p_{n},s_{n})]) & \text{by defn (\ref{defn:epsilon}) of $\epsilon_{\pcomonad{k}{A}}$} \\
&= \epsilon_{\pcomonad{k}{A}} \circ \delta_{A}([(p_{1},a_{1}),\dots,(p_{n},a_{n})]) & \text{by defn (\ref{defn:delta}) of $\delta_{A}$}
\end{align*}
By definition, $\pcomonad{k}{}$ is a comonad.
\end{proof}
\end{thm}
For every $l,k \in \omega$ such that $l \leq k$ and $\sigma$-structure $A$, there exists an inclusion $i_{A}^{l,k}: \pcomonad{l}{A} \longrightarrow \pcomonad{k}{A}$. 
\begin{prop}
The inclusion maps form a natural transformation $i^{l,k}:\pcomonad{l}{} \longrightarrow \pcomonad{k}{}$. Further, each map preserves the counit and comultiplication (i.e. each map is a morphism of comonads). 
\end{prop}
\begin{proof}
\begin{equation}
\bfig \square[\pcomonad{l}{A}`\pcomonad{k}{A}`\pcomonad{l}{B}`\pcomonad{k}{B};i^{l,k}_{A}`\pcomonad{l}{f}`\pcomonad{k}{f}`i^{l,k}_{B}]\efig
\end{equation}
\begin{align*}
\pcomonad{k}{f} \circ i^{l,k}_{A}([(p_{1},a_{1}),\dots,(p_{n},a_{n})])   &= \pcomonad{k}{f}([(p_{1},a_{1}),\dots,(p_{n},a_{n})]) & p_{i} \in \{1,\dots, l\} \subseteq \{1,\dots,k\}\\
&= [(p_{1},f(a_{1})),\dots,(p_{n},f(a_{n}))] &\text{by defn (\ref{defn:comonadMorphism}) of $\pcomonad{k}{f}$} \\
&= i^{l,k}_{B}([(p_{1},f(a_{1})),\dots,(p_{n},f(a_{n}))]) & p_{i} \in \{1,\dots, l\} \subseteq \{1,\dots,k\}\\
&= i^{l,k}_{B} \circ \pcomonad{l}{f}([(p_{1},a_{1}),\dots,(p_{n},a_{n})]) & \text{by defn (\ref{defn:comonadMorphism}) of $\pcomonad{k}{f}$}\\
\end{align*}
\end{proof}
The grading given by these inclusion maps seem to suggest that there is a colimit object capturing the information of $\pcomonad{k}{A}$ for every $k \in \omega$. This is indeed the case. Consider the structure $\pcomonad{\omega}{A}$ with domain $|\pcomonad{\omega}{A}| = (\omega \times A)^{\omega}$. The structure on $\pcomonad{\omega}{A}$ is similar to the structure given to $\pcomonad{k}{A}$. 
\begin{prop}
Let $\omega$ be considered as a poset category under the usual order. The object $\pcomonad{\omega}{A}$ is the $\omega$-colimit of the family $\{\pcomonad{k}{A}\}_{k \in \omega}$ with the above inclusion maps. 
\begin{proof}
For every $k \in \omega$, define $i^{k}_{A}:\pcomonad{k}{A} \rightarrow \pcomonad{\omega}{A}$ as the inclusion (i.e. $[(p_{1},a_{1}),\dots,(p_{n},a_{n})] \mapsto [(p_{1},a_{1}),\dots,(p_{n},a_{n})]$). Clearly, the following diagram commutes for all $l,k \in \omega$ with $l \leq k$
\begin{equation}
\bfig \Vtriangle[\pcomonad{l}{A}`\pcomonad{k}{A}`\pcomonad{\omega}{A};i^{l,k}_{A}`i^{l}_{A}`i^{k}_{A}]\efig
\label{eq:omegaColimit}
\end{equation}
Suppose that there exists a $\sigma$-structure $B$ and for every $l,k \in \omega$ with $l \leq k$, there exist morphisms $f^{l}:\pcomonad{l}{A} \longrightarrow B$, $f^{k}:\pcomonad{k}{A} \longrightarrow B$ such that $f^{l} = f^{k} \circ i^{l,k}$. Consider the morphism $u:\pcomonad{\omega}{A} \longrightarrow B$ given by $[(p_{1},a_{1}),\dots,(p_{n},a_{n})] \mapsto f^{j}([(p_{1},a_{1}),\dots,(p_{n},a_{n})])$ where $j = \max(p_{1},\dots,p_{n})$ 
\begin{equation}
\bfig 
    \Vtriangle[\pcomonad{l}{A}`\pcomonad{k}{A}`\pcomonad{\omega}{A};i^{l,k}_{A}`i^{l}_{A}`i^{k}_{A}]
    \morphism(0,500)|l|/{@{>}@/^-7pt/}/<500,-1000>[\pcomonad{l}{A}`B;f^{l}]
    \morphism(1000,500)|r|/{@{>}@/^7pt/}/<-500,-1000>[\pcomonad{k}{A}`B;f^{k}]
    \morphism(500,0)|m|/.>/<0,-500>[\pcomonad{\omega}{A}`B;u]
\efig
\label{eq:omegaColimitU}
\end{equation}
Moreover, given the conditions on $f^{j}$ for all $j \in \omega$, $u$ is unique. Namely, suppose there exists a morphism $u':\pcomonad{\omega}{A} \longrightarrow B$ such that for all $j \in \omega$, $f^{j} = u' \circ i^{j}_{A}$. Suppose $s = [(p_{1},a_{1}),\dots,(p_{n},a_{n})] \in \pcomonad{\omega}{A}$ and $k = \max(p_{1},\dots,p_{n})$, then for all $j \geq k, s \in \pcomonad{j}{A}$.  
\begin{align*}
u(s)    &= f^{k}(s) & \text{by defn of $u$} \\
        &= f^{j} \circ i^{k,j}_{A}(s) & \text{by (\ref{eq:omegaColimitU})} \\
        &= u' \circ i^{j}_{A} \circ i^{k,j}_{A}(s) & \text{by hypothesis on $u'$} \\
        &= u' \circ i^{k}_{A}(s) & \text{by (\ref{eq:omegaColimit})}\\
        &= u'(s) & \text{by defn of inclusion} 
\end{align*}
Since $u(s) = u'(s)$ for all $s \in \pcomonad{\omega}{A}$, $u = u'$ so $u$ is unique as desired.  
\end{proof}
\end{prop}
\section{Positional Form and Equivalences}
\subsection{Equivalence $\exists^{+}\mathcal{L}^{k}_{\infty,\omega}$}
\subsection{Equivalence $\mathcal{L}^{k}_{\infty,\omega}$}
\subsection{Equivalence $\mathcal{L}^{k}_{\infty,\omega}(Cnt)$}
\section{Coalgebras and Tree-Width}

\chapter{Modal Unfolding Comonad}
\section{Introduction}
Two transitions systems, or Kripke structures, $A,B$ are bisimilar if every transition in $A$ can be matched with a transition in $B$ and vice-versa. To show two structures are bisimilar, we need to construct a bisimulation $Z \subseteq A \times B$ matching transitions in $A$ with transitions in $B$ (and vice-versa). Just as the EF game and pebbling game are used to obtain partial homormorphisms between structures, a bisimulation game can be used to obtain a bisimulation. In terms of logic, the $k$-round bisimulation game is used to prove equivalence between two Kripke structures in the $k$-depth modal fragment of first order logic. Given two Kripke structures $A,B$, the $k$-round bisimulation game is played with Spoiler and Duplicator as follows: 
\begin{itemize} 
\item In the first round, Spoiler picks an arbitrary element $a_{1} \in A$ or $b_{1} \in B$
\item Duplicator responds by picking an arbitrary element in the other structure $b_{1} \in B$ or $a_{1} \in A$.
\item In a subsequent $i$-th round, Spoiler chooses a binary relation $R_{z} \in \sigma$ and $a_{i} \in A$ or $b_{i} \in B$ such that $(a_{i-1},a_{i}) \in R_{z}^{A}$ or $(b_{i-1},b_{i}) \in R_{z}^{B}$
\item Duplicator responds by choosing an element in the other structure $b_{i} \in B$ or $a_{i} \in A$. If $(a_{i-1},a_{i}) \in R_{z}^{A} \Leftrightarrow (b_{i-1},b_{i}) \in R_{z}^{B}$ for the $R_{z} \in \sigma$ that was picked by Spoiler, then Duplicator wins the $i$-th round.  
\end{itemize}
At the end of the $k$-round game, $k$-tuples $(a_{1},\dots,a_{k})$ and $(b_{1},\dots,b_{k}$ have be chosen. Duplicator wins the $k$-round game if Duplicator won every $i$-th round for $i \in [k]$ and for all $j \in [k]$ and predicates $P \in \sigma$, $a_{j} \in P^{A} \Leftrightarrow b_{j} \in P^{B}$. Otherwise, Spoiler wins. The simulation game (or $\lozenge$-game) from $A$ to $B$, is the same game with the additional restriction that Spoiler must always play an element in $A$. Hence, Duplicator must always respond in $B$ and the winning conditions alter to one-sided implications. The following result is standard in any modal logic text: 
\begin{prop}
The following are equivalent:
\begin{itemize}
\item Duplicator has a winning strategy in the $k$-round bisimulation game 
\item $\equivL{A}{B}{\mathcal{M}_{\omega,k}}$, i.e. for every sentence $\phi \in \mathcal{M}_{\omega,k}, A \vDash \phi \Leftrightarrow B \vDash \phi$
\end{itemize}
\end{prop}
As was the goal with the EF game comonad, our goal is to construct a $\sigma$-structure $\mcomonad{k}{A}$ that ``internalizes'' the $k$-round simulation and bisimulation games in the category $\mathcal{R}(\sigma)$. The tree unfolding construction of depth $k$ for pointed transition systems, detailed in \cite{Gradel2014}, turns out to be the correct candidate for $\mcomonad{k}{A}$. This construction is typically empolyed to show the tree-model property which yields a proof of the decidability of modal logic. We use this construction, instead, to show that it captures the modal simulation and bisimulation game (i.e. analogous to corollaries \ref{cor:forthEF} and \ref{cor:backForthEF}) as a comonad.
\section{Comonad laws}
\section{Positional Form and Equivalences}\label{sec:positionalFormM}
The positional form representation for the simulation and bisimulation games involving $A,B$ are, just as with the EF game, sequences of pairs of elements in $A,B$ of length $\leq k$. However, the choice of each pair must be local (i.e. only one transition away from the previous pair). Our definitions will reflect this modification. Recall, from \ref{sec:positionalFormEF}, that $\Gamma_{k}(A,B) = (A \times B)^{\leq k}$ and for a $\sigma$-morphism $f:\mcomonad{k}{A} \longrightarrow B$ there exists the Kleisli coextension $f^{*}:\mcomonad{k}{A} \longrightarrow \mcomonad{k}{B}$. Define the set function $\theta_{f}:|\mcomonad{k}{A}| \longrightarrow \Gamma_{k}(A,B)$ by $s = [a_{1},i_{1},\dots,i_{n-1},a_{n}] \mapsto [(a_{1},b_{1}),\dots,(a_{n},b_{n})]$ where $f^{*}(s) = [b_{1},i_{1},\dots,i_{n-1},b_{n}]$. 
\begin{defn}
$S \subseteq \Gamma_{k}(A,B)$ is a strategy in positional form if $S$ satisfies the following conditions:
\begin{enumerate}[label=(S\arabic*),ref=S\arabic*,start=0]
\item For every $a \in A$, there exists a unique $b \in B$ such that $[(a,b)] \in S$ \label{eq:S1st}
\item For all $\zeta \in S$ with last position $(a,b)$ and $i \in [m]$ with $(a,a') \in R_{i}^{A}$, there exists a unique $b' \in B$ such that $\zeta' = \zeta[(a',b')] \in S$. Denote this update as $\zeta \xlongrightarrow{(i,a):b} \zeta'$ \label{eq:S2nd}
\item $S$ is reachable: For all $\zeta \in S$, there is a chain \label{eq:S3rd}
$$\zeta_{1} \longrightarrow \dots \longrightarrow \zeta_{n}$$
such that $\zeta_{n} = \zeta$ and $\zeta_{i} \in S$. 
\end{enumerate}
\end{defn}
\begin{prop}
If $f:\mcomonad{k}{A} \longrightarrow B$ is a $\sigma$-morphism, then there exists a strategy in positional form $S_{f}$.
\begin{proof}
Define: 
$$S_{f} = \{\theta_{f}(w) \mid w \in \mcomonad{k}{A}\}$$
\begin{itemize}
\item Suppose $a \in A$, then there exists a unique $b$ such that $f([a]) = b$. By definition of $\theta_{f}$, $[(a,b)] \in S_{f}$. 
\item Suppose $\zeta \in S_{f}$ with $|\zeta| < k$ and last position $(a,b)$, then there exists some $w \in \mcomonad{k}{A}$ such that $\zeta = \theta_{f}(w)$. Consider arbitrary $i \in [m]$ with $(a,a') \in  R_{i}^{A}$, then there exists a unique $b' \in B$ such that $f(w[i,a']) = b'$. Hence, $\theta_{f}(w[i,a']) = \zeta[(a',b')] \in S_{f}$.
\item Suppose $\zeta \in S_{f}$, then $\zeta = \theta_{f}(w)$ for some $w \in \mcomonad{k}{A}$. Let $w_{i}$ be the $i$-th element in the sequence $\delta_{A}(w) \in \mcomonad{k}{\mcomonad{k}{A}}$, then:  
$$\theta_{f}(w_{1}) \longrightarrow \dots \longrightarrow \theta_{f}(w_{n}) = \zeta$$ 
Hence, $S_{f}$ is reachable.
\end{itemize}
\end{proof}
\label{prop:fToPosFormM}
\end{prop}
\begin{prop}
Conversely, for every strategy in positional form $S$ there exists a $f:\efcomonad{k}{A} \rightarrow B$ such that $S = S_{f}$
\begin{proof}
$S_{f} \subseteq S$ The approach is to construct an appropriate $f$. We construct $f$ by recursion, up to $k$, on the length of a play $w \in \efcomonad{k}{A}$. \\ 
\textit{Base Case:} Suppose $w = [a]$ for $a \in A$. By (\ref{eq:S1st}), there exists a unique $b$ such that $[(a,b)] \in S$. Let $f(w) = b$. \\
\textit{Recursive Step:} Assume for the recursion, that $f(w)$ is defined for $|w| = n < 2k+1$ and that $\theta_{f}(w) \in S$. Consider $w' = w[i,a]$ for $(\epsilon_{A}(w),a) \in R_{i}$. By (\ref{eq:S2nd}), there exists a unique $b \in B$ such that $\zeta' = \theta_{f}(w)[(a,b)]$. Let $f(w') = b$. \\
$S \subseteq S_{f}$ We must show that for every $\zeta \in S$, there exists a $w \in \mcomonad{k}{A}$ such that $\zeta = \theta_{f}(w)$. By induction on reachability sequence of $\zeta$ (obtained via \ref{eq:S3rd}). \\
\textit{Base Case:} Suppose $\zeta = [(a,b)]$. By the construction of $f$ above, $f([a]) = b$, so $\zeta = \theta_{f}([a])$. \\
\textit{Inductive Step:} Assume for the inductive hypothesis that $\zeta = \theta_{f}(w)$ for some $w \in \mcomonad{k}{A}$. Suppose $\zeta \xlongrightarrow{(i,a):b} \zeta'$. Consider $w' = w[i,a]$, then $\theta_{f}(w') = \zeta'$.
\end{proof}
\label{prop:posFormToFM}
\end{prop}
\subsection{Equivalence $\exists^{+}\mathcal{M}_{\omega,k}$}
\begin{defn}
A position $\zeta = [(a_{1},b_{1}),\dots,(a_{n},b_{n})] \in \Gamma_{k}(A,B)$ is \textit{winning for Duplicator in $\mgame{k}{A}{B}$} if $\zeta$ a is simulation. That is, for all $i = 2,\dots,n$, $(a_{i-1},a_{i}) \in R_{z}^{A} \Rightarrow (b_{i-1},b_{i}) \in R_{z}^{B}$ for some $R_{z} \in \sigma$ binary and for all $P \in \sigma$ unary $a_{i} \in P_{j} \Rightarrow b_{i} \in P_{j}$. Naturally, we can extend the definition to say that a strategy in positional form $S \subseteq \Gamma_{k}(A,B)$ is \textit{winning for Duplicator in $\mgame{k}{A}{B}$} if for all $\zeta \in S$, $\zeta$ is winning for Duplicator $\mgame{k}{A}{B}$. Function $f:\mcomonad{k}{A} \longrightarrow B$ is \textit{winning for Duplicator $\mgame{k}{A}{B}$} if $S_{f}$ is winning for Duplicator in $\mgame{k}{A}{B}$ 
\end{defn}
Intuitively, condition (\ref{eq:S2nd}) and winning for Duplicator together form the forth requirement in the back-and-forth definition of bisimulation. With the definition of positional form for the modal fragment in place, the main theorem relating the comonad and equivalence in $\mathcal{M}_{\infty,k}$ can be proved.  
\begin{thm}
If $A,B$ are Kripke structures and $f:\mcomonad{k}{A} \rightarrow B$ is a function, then 
\center{$f:\mcomonad{k}{A} \longrightarrow B$ is a $\sigma$-morphism if and only if $f$ is winning for Duplicator $\mgame{k}{A}{B}$}
\begin{proof}
$\Leftarrow$ Suppose $\zeta \in S_{f}$, the by definition of $S_{f}$, $\zeta = \theta_{f}(w)$ for some $w \in \mcomonad{k}{A}$. If $w = [a]$, then it is trivally true that $\zeta = [(a,b)]$ is a simulation (i.e. winning for Duplicator). If $w = v[i,a']$ (i.e. $\zeta = \theta_{f}(v)[(a',b')]$) and suppose the last position of $\theta_{f}(v)$ is $(a,b)$, then by interpretation of $R_{i}$ on $\mcomonad{k}{A}$, $(v,w) \in R_{i}^{\mcomonad{k}{A}}$. By $f$ a $\sigma$-morphism, $(f(v),f(w)) \in R_{i}^{B}$. By the definition of $\theta_{f}$, $f(v) = b$ and $f(w) = b'$, so $(b,b') \in R_{i}^{B}$. Moreover, for predicate $P \in \sigma$, if $a' \in P^{A}$, then $w \in P^{\mcomonad{k}{A}}$ by $a$ being the last element of $w$, so $f(w) = b' \in P^{B}$ by $f$ a $\sigma$-morphism. Therefore, $\zeta$ is a simulation and winning for Duplicator. Hence, $S_{f}$ is winning for Duplicator.\\
$\Rightarrow$ Suppose $(v,w) \in R_{i}^{\mcomonad{k}{A}}$ for $R_{i} \in \sigma$ binary relation, then by intepretation of $R_{i}$ on $\mcomonad{k}{A}$, $v = w[i,a]$ or $w = v[i,a]$. Without loss of generality, assume $w = v[i,a]$. Consider $\zeta \in \theta_{f}(w) \in S_{f}$. By $w = v[i,a']$ and definition of $\theta_{f}$, $\theta_{f}(w) = \theta_{f}(v)[(a',f(w))]$ where $\theta_{f}(v) = u[(\epsilon_{A}(v),f(v))]$ for some (possibly empty) sequence $u$. Since $\zeta$ is a simulation (i.e. winning for Duplicator), $(f(v),f(w)) \in R^{B}$. Hence, $f$ is a $\sigma$-morphism.
\end{proof}
\label{thm:toPositionalFormM}
\end{thm}
\begin{prop}
For all $k \in \omega$, the following are equivalent:
\begin{enumerate}[label=(\arabic*)$_{k}$]
\item For every $\phi \in \exists^{+}\mathcal{M}_{\omega,k}$, $A \vDash \phi \Rightarrow B \vDash \phi$
\item There exists a strategy in positional form $S \subseteq \Gamma_{k}(A,B)$ that is winning for Duplicator 
\end{enumerate}
\begin{cor}
There exists a $\sigma$-morphism $f:\mcomonad{k}{A} \longrightarrow B$ if and only if for all $\phi \in \exists^{+}\mathcal{M}_{\omega,k}, A \vDash \phi \Rightarrow B \vDash \phi$
\begin{proof}
Straightforward from theorem (\ref{thm:toPositionalFormM}) and proposition (\ref{prop:toSyntaxM}).
\end{proof}
\label{cor:forthOneM}
\end{cor}
\begin{cor}
There exists $\sigma$-morphisms $f:\mcomonad{k}{A} \longrightarrow B$, $g:\mcomonad{k}{B} \longrightarrow B$ if and only if $\equivL{A}{B}{\exists^{+}\mathcal{M}_{\omega,k}}$. 
\begin{proof}
Recall definition (\ref{defn:equivLogic}) and apply above corollary (\ref{cor:forthOneM}) to $f$ and $g$.  
\end{proof}
\label{cor:forthM}
\end{cor}
\label{prop:toSyntaxM}
\end{prop}
\subsection{Equivalence $\mathcal{M}_{\omega,k}$}
Just as with the $\efcomonad{k}{}$ comonad, $\mcomonad{k}{}$ can also internalize winning strategies in the bisimulation game $\mgameSym{k}{A}{B}$. That is, given two Kripke $\sigma$-structures $A,B$, with $\sigma$-morphisms $f:\mcomonad{k}{A} \longrightarrow B$ and $g:\mcomonad{k}{B} \longrightarrow A$, Duplicator has a winning strategy in $\mgameSym{k}{A}{B}$ if $f$ and $g$ are compatible in positional form. Namely, given the canonical swap function, if $S_{f} = S_{g}^{-}$ and $S_{g} = S_{f}^{-}$, then there exists a bisimulation between $A$ and $B$. To see this $f:\mcomonad{k}{A} \longrightarrow A$ implies that $S_{f}$ is winning $\mgame{k}{A}{B}$ and the forth condition in the simulation from $A$ to $B$ is satisfied. Likewise, $g:\mcomonad{k}{B} \longrightarrow A$ implies that $S_{g}$ is winning in $\mgame{k}{B}{A}$ and the forth condition in the simulation from $B$ to $A$ is satisfied. The condition that $S_{f} = S_{g}^{-}$ means that the forth condition in the simulation from $B$ to $A$ is equivalent to the back condition in the bisimulation from $A$ to $B$. 
\begin{prop}
For all $k \in \omega$, the following are equivalent:
\begin{enumerate}[label=(\arabic*)$_{k}$]
\item $\equivL{A}{B}{\mathcal{M}_{\omega,k}}$ 
\item There exists a winning strategy in positional form $S \subseteq \Gamma_{k}(A,B)$ such that $S$ is winning for Duplicator and $S
^{-}$ is winning for Duplicator. 
\end{enumerate}
\end{prop}
\section{Guarded Unfolding}

\chapter{Generalizations and Relationships}
\section{Arrow-Theoretic from $\exists^{+}\mathcal{L}$ to $\mathcal{L}$}
\section{Relationship between $\pcomonad{k}{}$ and $\efcomonad{k}{}$}
\subsection{$\pcomonad{\omega}{}$ and $\efcomonad{k}{}$}
\subsection{Grading by both $\bcomonad{k}{n}{}$}
\section{New proofs for Common Results}

\bibliographystyle{plainnat} 
\bibliography{main}
\end{document}
