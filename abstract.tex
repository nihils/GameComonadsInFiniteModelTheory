\begin{abstract}
Back-and-Forth or Ehrenfreucht-{\Fraisse} style games are an important tool in finite model theory, complexity theory, and database theory. Monads and comonads are constructions of category theory that appear often in the semantics of computation and functional programming languages. A recent paper by Abramsky, Dawar, and Wang \cite{Abramsky2017} showed that existential $k$-pebble games have a natural comonadic formulation. The purpose of this dissertation is to modify and generalize the results in \cite{Abramsky2017} to give a comonadic formulation of existential positive Ehrenfreuct-{\Fraisse} games and modal simulation games. The results presented here lay the basis for a new research programme that connect two, previously disjoint areas of logic in computer science: (1) finite and algorithmic model theory and (2) categorical structures of computation. This is a step towards the goal of giving purely categorical definitions of logics over finite structures and complexity classes.\\~\\ 
In Chapter 2, we provide the necessary notations and background for the concepts in finite model theory and category theory we will need. \\~\\
In Chapter 3, we construct a comonad $\efcomonad{k}{}$ that internalizes Ehrenfreucht-{\Fraisse} games. The Ehrenfreucht-{\Fraisse} game captures equivalence in finite quantifier rank fragments of first-order logic. The second section defines and proves the construction is indeed a comonad. The third section shows that $\efcomonad{k}{}$ indeed captures the game and equivalence in finite rank logic. The final section studies the category of coalgebras of $\efcomonad{k}{}$ and proves the suprising result that these coalgebras provide a definition of tree-depth.  \\~\\
In Chapter 4, we construct the comonad $\pcomonad{k}{}$, discovered in \cite{Abramsky2017}, that internalizes pebbling games. The pebbling game captures equivalence in finite variable fragments of first-order and infinitary logic. We reproduce the proofs for the results analogous to the ones in chapter 2. A novel proof is given for the result \cite[]{Abramsky2017} that gives a definition of tree-width in terms of the category of coalgebras of $\pcomonad{k}{}$. \\~\\ 
In Chapter 5, we construct a comonad $\mcomonad{k}{}$ which internalizes simulations and bisimulations for the modal fragments of infinitary logic. We prove results analogous to the ones in chapter 3 and 4 for the $\mcomonad{k}{}$ comonad. \\~\\ 
In Chapter 6, we explain some results that generalize and relate the $\efcomonad{k}{}$, $\pcomonad{k}{}$, and $\mcomonad{k}{}$ comonads. The first section gives a general arrow-theoretic development for transferring winning strategies in an asymmetric game to winning strategies in a symmetric game. The second section relates the $\efcomonad{k}{}$ and $\pcomonad{k}{}$ comonads. The third section demonstrates how these comonads yield new proofs of inclusions among fragments of first-order logic and inequalities among combinatorial parameters of relational structures. \\~\\ 
In Chapter 7, we summarize the main results. We also point towards further application of this work to functional programming with Idris code samples. We dicuss possible ways of creating comonads for extensions (e.g. generalized quantifiers, branching quantifiers) of first-order logic. \\~\\
Our novel contribution is:
\begin{itemize}
\item The construction of $\efcomonad{k}{}$ and results showing that it provides a comonadic formulation of asymmetric, symmetric, and bijective Ehrenfreucht-{\Fraisse} games. The charcterization of tree-depth of structures in terms of the coalgebras of $\efcomonad{k}{}$
\item A novel proof of the characterization of tree-width in terms of coalgebras of $\pcomonad{k}{}$ 
\item Showing that the $\mcomonad{k}{}$ unravelling construction in \cite{Gradel2014} provides a comonadic formulation of simulation and bisimulation games in the modal fragment of first-order logic.
\item The general arrow-theoretic development relating asymmetric games and symmetric games.
\item A natural transformation from $\efcomonad{k}{}$ to $\pcomonad{k}{}$ gives a novel proof of the result that $k$-rank logic is contained within $k$-variable logic. The same natural transformation gives a novel proof that tree-depth of a structure is strictly greater than its tree-width. 
\item A natural transformation between $\mcomonad{\omega}{}$ and $\pcomonad{2}{}$ gives a novel proof that modal logic is contained within $2$-variable logic. 
\end{itemize}
\end{abstract}
