\chapter{Modal Unfolding Comonad}
\section{Introduction}
Two transitions systems, or Kripke structures, $A,B$ are bisimilar if every transition in $A$ can be matched with a transition in $B$ and vice-versa. To show two structures are bisimilar, we need to construct a bisimulation $Z \subseteq A \times B$ matching transitions in $A$ with transitions in $B$ (and vice-versa). Just as the EF game and pebbling game are used to obtain partial homormorphisms between structures, a bisimulation game can be used to obtain a bisimulation. In terms of logic, the $k$-round bisimulation game is used to prove equivalence between two Kripke structures in the $k$-depth modal fragment of first order logic. Given two Kripke structures $A,B$, the $k$-round bisimulation game is played with Spoiler and Duplicator as follows: 
\begin{itemize} 
\item In the first round, Spoiler picks an arbitrary element $a_{1} \in A$ or $b_{1} \in B$
\item Duplicator responds by picking an arbitrary element in the other structure $b_{1} \in B$ or $a_{1} \in A$.
\item In a subsequent $i$-th round, Spoiler chooses a binary relation $R_{z} \in \sigma$ and $a_{i} \in A$ or $b_{i} \in B$ such that $(a_{i-1},a_{i}) \in R_{z}^{A}$ or $(b_{i-1},b_{i}) \in R_{z}^{B}$
\item Duplicator responds by choosing an element in the other structure $b_{i} \in B$ or $a_{i} \in A$. If $(a_{i-1},a_{i}) \in R_{z}^{A} \Leftrightarrow (b_{i-1},b_{i}) \in R_{z}^{B}$ for the $R_{z} \in \sigma$ that was picked by Spoiler, then Duplicator wins the $i$-th round.  
\end{itemize}
At the end of the $k$-round game, $k$-tuples $(a_{1},\dots,a_{k})$ and $(b_{1},\dots,b_{k}$ have be chosen. Duplicator wins the $k$-round game if Duplicator won every $i$-th round for $i \in [k]$ and for all $j \in [k]$ and predicates $P \in \sigma$, $a_{j} \in P^{A} \Leftrightarrow b_{j} \in P^{B}$. Otherwise, Spoiler wins. The simulation game (or $\lozenge$-game) from $A$ to $B$, is the same game with the additional restriction that Spoiler must always play an element in $A$. Hence, Duplicator must always respond in $B$ and the winning conditions alter to one-sided implications. The following result is standard in any modal logic text: 
\begin{prop}
The following are equivalent:
\begin{itemize}
\item Duplicator has a winning strategy in the $k$-round bisimulation game 
\item $\equivL{A}{B}{\mathcal{M}_{\omega,k}}$, i.e. for every sentence $\phi \in \mathcal{M}_{\omega,k}, A \vDash \phi \Leftrightarrow B \vDash \phi$
\end{itemize}
\end{prop}
As was the goal with the EF game comonad, our goal is to construct a $\sigma$-structure $\mcomonad{k}{A}$ that ``internalizes'' the $k$-round simulation and bisimulation games in the category $\mathcal{R}(\sigma)$. The tree unfolding construction of depth $k$ for pointed transition systems, detailed in \cite{Gradel2014}, turns out to be the correct candidate for $\mcomonad{k}{A}$. This construction is typically empolyed to show the tree-model property which yields a proof of the decidability of modal logic. We use this construction, instead, to show that it captures the modal simulation and bisimulation game (i.e. analogous to corollaries \ref{cor:forthEF} and \ref{cor:backForthEF}) as a comonad.
\section{Comonad laws}
A Kripke structure or transition system is a relational structure with a signature $\sigma = <R_{1},\dots,R_{m},P_{1},\dots,P_{n}>$ where every $R_{i}$ is a binary relation and every $P_{j}$ is a unary relation. Let $A$ be a $\sigma$-structure, then for every $k \in \omega$ we define a  $\sigma$-structure $\mcomonad{k}{A}$. Intuitively, $\mcomonad{k}{A}$ is the structure of the finite Spoiler plays in the bisimulation game of modal depth $k$. The purpose of studying such a structure is that it captures, in a manner analogous to the pebbling and Ehrenfeucht-{\Fraisse} comonads, bisimulation equivalence for models of modal propositional logic. This logic is equivalent to a fragment of first-order logic with quantifiers guarded by the binary relations $R_{1},\dots,R_{m}$. This development will be illustrative for introducing the more general guarded fragment of first-order logic where we consider sentences guarded by general $n$-ary relations. Let $|\mcomonad{k}{A}|$ be the set of sequences of the form $(a_{1},i_{1},a_{2},\dots,a_{n-1},i_{n-1},a_{n})$ where $n \leq k$, $a_{j} \in A, i_{j} \in \{1,\dots,m\}$, and for every $i = 1,\dots,k$, $(a_{j},a_{j+1}) \in R^{A}_{i_{j}}$.   
\begin{defn}
Suppose $w,w' \in |\mcomonad{k}{A}|$ such that $w = [a_{1},i_{1},\dots,i_{n-1},a_{n}]$, $s' = [a'_{1},i'_{1},\dots,i'_{m-1},a'_{m}]$, $m+n \leq k$ and $(a_{n},a'_{1}) \in R^{A}_{\nu}$, then define $ss' = [a_{1},i_{1}
\dots,a_{n},\nu,a'_{1},i'_{1},\dots,i'_{m-1},a'_{m}]$ (i.e. the concatenation of these two tuples).
\end{defn}
\begin{defn}
For $w,z \in |\mcomonad{k}{A}|$, say $w \sqsubseteq z$ if there exists an $w' \in |\mcomonad{k}{A}|$ such that $ww' = z$; such an $w'$ is called the suffix of $w$ in $z$. $\sqsubseteq$ preorders $|\mcomonad{k}{A}|$.
\end{defn}
\begin{defn}
Define, for every $\sigma$-structure $A$, $\epsilon_{A}:\mcomonad{k}{A} \longrightarrow A$ by $[a_{1},i_{1},\dots,i_{n-1},a_{n}] \mapsto a_{n}$ (i.e. the last move of the play). 
\label{defn:epsilon}
\end{defn}
With these definitions in place, we can define a $\sigma$-structure on $\mcomonad{k}{A}$. Suppose $R_{j}$ is a binary relation in $\sigma$, the we define the interpretation of $R^{\mcomonad{k}{A}}$ such that for every $w,w' \in |\mcomonad{k}{A}|$,
\begin{equation}
(w,w') \in R^{\mcomonad{k}{A}}_{j} \Leftrightarrow w' = w[(j,\epsilon_{A}(w'))] \label{eq:R}
\end{equation}
By the definition of the set $|\mcomonad{k}{A}|$, the above condition is equivalent to $(\epsilon_{A}(w),\epsilon_{A}(w')) \in R^{A}_{j}$. \\
Similarly, for $P_{i}$ a unary relation in $\sigma$, we define the interpretation $P^{\mcomonad{k}{A}}(w) \Leftrightarrow P^{A}_{i}(\epsilon_{A}(w))$
\begin{defn}
Given a morphism $f:A \longrightarrow B$, define the morphism $\mcomonad{k}{f}:\mcomonad{k}{A} \longrightarrow \mcomonad{k}{B}$ by $[a_{1},i_{1}\dots,i_{n-1},a_{n}] \mapsto [f(a_{1}),i_{1},\dots,i_{n-1},f(a_{n})]$
\label{defn:mComonadMorphism}
\end{defn}
\begin{prop}
The definition (\ref{defn:mComonadMorphism}) of $\mcomonad{k}{f}:\mcomonad{k}{A} \longrightarrow \mcomonad{k}{B}$ given above is indeed a morphism of $\sigma$-structures. 
\begin{proof}
Suppose $w,w' \in \mcomonad{k}{A}$ such that $(w,w') \in R_{j}^{\mcomonad{k}{A}}$. Let $w = [a_{1},i_{1},\dots,i_{p-1},a_{p}]$ and $w' = [a'_{1},i_{1},\dots,i_{q-1},a'_{q}]$. We aim to show that $(\mcomonad{k}{f}(w),\mcomonad{k}{f}(w')) \in R_{j}^{\mcomonad{k}{B}}$ \\
\begin{align*}
(w,w') \in R_{j}^{\mcomonad{k}{A}} &\Leftrightarrow w' = w[(j,\epsilon_{A}(w'))] & \text{by (\ref{eq:R}) above} \\
&\Leftrightarrow R_{j}^{A}(\epsilon_{A}(w),\epsilon_{A}(w')) & \text{by defn of } |\mcomonad{k}{A}| \\
&\Rightarrow R_{j}^{B}(f(\epsilon_{A}(w)),f(\epsilon_{A}(w')) & \text{by hypothesis $f$ a $\sigma$-morphism} \\
&\Rightarrow \mcomonad{k}{f}(w') = \mcomonad{k}{f}(w)[(j,f \circ \epsilon_{A}(w'))] & \text{by defn of } |\mcomonad{k}{A}| \\
&\Rightarrow \mcomonad{k}{f}(w') = \mcomonad{k}{f}(w)[(j,\epsilon_{B} \circ \mcomonad{k}{f}(w'))] & \text{by proposition (\ref{prop:epsilonN})}\\
&\Rightarrow (\mcomonad{k}{f}(w),\mcomonad{k}{f}(w')) \in R_{j}^{\mcomonad{k}{B}} & \text{by (\ref{eq:R}) above}
\end{align*}
Therefore, $(\mcomonad{k}{f}(w),\mcomonad{k}{f}(w')) \in R_{j}^{\mcomonad{k}{B}}$ and $\mcomonad{k}{f}$ is indeed a morphism of $\sigma$-structures. 
\end{proof}
\end{prop}
\begin{prop}
$\epsilon:\mcomonad{k}{} \longrightarrow 1_{\mathcal{R}(\sigma)}$ is a natural transformation.
\begin{proof}
For every $A,B \in \mathcal{R}(\sigma)$ we want to show that:
\begin{equation}
\bfig \square[\mcomonad{k}{A}`A`\mcomonad{k}{B}`B;\epsilon_{A}`\mcomonad{k}{f}`f`\epsilon_{B}] \efig
\label{eq:epsilonN}
\end{equation}
\begin{align*}
f \circ \epsilon_{A}([a_{1},i_{1},\dots,i_{n-1},a_{n})])    &= f(a_{n}) & \text{by defn (\ref{defn:epsilon}) of $\epsilon_{A}$}\\
&= \epsilon_{B}([f(a_{1}),i_{1},\dots,i_{n-1},f(a_{n})]) & \text{by defn (\ref{defn:epsilon}) of $\epsilon_{B}$}\\
&= \epsilon_{B} \circ \mcomonad{k}{f}([a_{1},i_{1},\dots,i_{n-1},a_{n}]) & \text{by defn (\ref{defn:mComonadMorphism}) of $\mcomonad{k}{f}$}
\end{align*}
Hence, the above diagram (\ref{eq:epsilonN}) commutes.
\end{proof}
\label{prop:epsilonN}
\end{prop}
\begin{defn}
Suppose $w \in \mcomonad{k}{A}$, then $w = [a_{1},i_{1},\dots,i_{n-1},a_{n}]$ for some $n \in \omega$ and for every $i = 1,\dots,n$, $a_{i} \in A$. Let $w_{i} = [a_{1},i_{1},\dots,i_{i-1},a_{i}] \in \mcomonad{k}{A}$. Define, for every $\sigma$-structure $A$, $\delta_{A}:\mcomonad{k}{A} \longrightarrow \mcomonad{k}{\mcomonad{k}{A}}$ by $w \mapsto [w_{1},i_{1},\dots,i_{n-1},w_{n}]$.
\label{defn:delta}
\end{defn}
\begin{prop}
$\delta:\mcomonad{k} \longrightarrow \mcomonad{k}{\mcomonad{k}{}}$ is a natural transformation.
\begin{proof}
For every $A,B \in \mathcal{R}(\sigma)$ we want to show that:
\begin{equation}
\bfig \square[\mcomonad{k}{A}`\mcomonad{k}{\mcomonad{k}{A}}`\mcomonad{k}{B}`\mcomonad{k}{\mcomonad{k}{B}};\delta_{A}`\mcomonad{k}{f}`\mcomonad{k}{\mcomonad{k}{f}}`\delta_{B}] \efig
\label{eq:deltaN}
\end{equation}
\begin{align*}
\mcomonad{k}{\mcomonad{k}{f}} \circ \delta_{A}([a_{1},i_{1},\dots,i_{n-1},a_{n}])   &= \mcomonad{k}{\mcomonad{k}{f}}([w_{1},i_{1},\dots,i_{n-1},w_{n}]) & \text{by defn (\ref{defn:delta}) of $\delta_{A}$} \\
&= [\mcomonad{k}{f}(w_{1}),i_{1},\dots,i_{n-1},\mcomonad{k}{f}(w_{n})] & \text{by defn (\ref{defn:mComonadMorphism}) of $\mcomonad{k}{\mcomonad{k}{f}}$}\\
&= [[f(a_{1})],i_{1},\dots,i_{n-1},[f(a_{1}),i_{1},\dots,i_{n-1},f(a_{n})]] & \text{by defn (\ref{defn:mComonadMorphism}) of $\mcomonad{k}{f}$}\\
&= \delta_{B}([f(a_{1}),i_{1},\dots,i_{n-1},f(a_{n})]) & \text{by defn (\ref{defn:delta}) of $\delta_{B}$} \\
&= \delta_{B} \circ \mcomonad{k}{f}([a_{1},i_{1},\dots,i_{n-1},a_{n}]) & \text{by defn (\ref{defn:mComonadMorphism}) of $\mcomonad{k}{f}$}
\end{align*}
Hence, the above diagram (\ref{eq:deltaN}) commutes.
\end{proof}
\label{prop:deltaN}
\end{prop}
\begin{thm}
The triple $\langle \mcomonad{k}{},\delta,\epsilon \rangle$ is a comonad.
\begin{proof}
By proposition (\ref{prop:deltaN}) and (\ref{prop:epsilonN}), $\delta$ and $\epsilon$ are natural transformation. Hence, $\delta$ and $\epsilon$ are indeed the comultiplication and counit of $\mcomonad{k}{}$. The associative and identity laws remain to be shown. \\
For associativity, for every $A \in \mathcal{R}(\sigma)$, the following diagram commutes:  
\begin{equation}
\bfig \Square[\mcomonad{k}{A}`\mcomonad{k}{\mcomonad{k}{A}}`\mcomonad{k}{\mcomonad{k}{A}}`\mcomonad{k}{\mcomonad{k}{\mcomonad{k}{A}}};\delta_{A}`\delta_{A}`\delta_{\mcomonad{k}{A}}`\mcomonad{k}{\delta_{A}}] \efig 
\end{equation}
\begin{align*}
\delta_{\mcomonad{k}{A}} \circ \delta_{A}([a_{1},i_{1},\dots,i_{n-1},a_{n}])    &= \delta_{\mcomonad{k}{A}}([w_{1},i_{1},\dots,i_{n-1},w_{n}]) & \text{by defn (\ref{defn:delta}) of $\delta_{A}$} \\
&= [[w_{1}],i_{1},\dots,i_{n-1},[w_{1},i_{1},\dots,i_{n-1},w_{n}]]  & \text{by defn (\ref{defn:delta}) of $\delta_{\mcomonad{k}{A}}$}\\
&= [\delta_{A}(w_{1}),i_{1},\dots,i_{n-1},\delta_{A}(w_{n})] & \text{by defn (\ref{defn:delta}) of $\delta_{A}$}  \\
&= \mcomonad{k}{\delta_{A}}([w_{1},i_{1},\dots,i_{n-1},w_{n}]) & \text{by defn (\ref{defn:mComonadMorphism}) of $\mcomonad{k}{\delta_{A}}$}  \\
&= \mcomonad{k}{\delta_{A}} \circ \delta_{A}([a_{1},i_{1},\dots,i_{n-1},a_{n}]) & \text{by defn (\ref{defn:delta}) of $\delta_{A}$}\\
\end{align*}
For identity, for every $A \in \mathcal{R}(\sigma)$, the following diagram commutes:  
\begin{equation}
\bfig 
    \square[\mcomonad{k}{A}`\mcomonad{k}{\mcomonad{k}{A}}`\mcomonad{k}{\mcomonad{k}{A}}`\mcomonad{k}{A};\delta_{A}`\delta_{A}`\mcomonad{k}{\epsilon_{A}}`\epsilon_{\mcomonad{k}{A}}] 
    \morphism(0,500)/=/<500,-500>[\mcomonad{k}{A}`\mcomonad{k}{A};]
\efig 
\end{equation}
\begin{align*}
\mcomonad{k}{\epsilon_{A}} \circ \delta_{A}([a_{1},i_{1},\dots,i_{n-1},a_{n}]) &= \mcomonad{k}{\epsilon_{A}}([w_{1},i_{1},\dots,i_{n-1},s_{n}]) & \text{by defn (\ref{defn:delta}) of $\delta_{A}$}\\
&= [\epsilon_{A}(w_{1}),\dots,\epsilon_{A}(w_{n})]  & \text{by defn (\ref{defn:mComonadMorphism}) of $\mcomonad{k}{\epsilon_{A}}$}  \\
&= [a_{1},i_{1},\dots,i_{n-1},a_{n}] & \text{by defn (\ref{defn:epsilon}) of $\epsilon_{A}$}\\
&= w_{n} & \text{by defn (\ref{defn:delta}) of $s_{n}$}\\
&= \epsilon_{\mcomonad{k}{A}}([w_{1},i_{1},\dots,i_{n-1},w_{n}]) & \text{by defn (\ref{defn:epsilon}) of $\epsilon_{\mcomonad{k}{A}}$} \\
&= \epsilon_{\mcomonad{k}{A}} \circ \delta_{A}([a_{1},i_{1},\dots,i_{n-1},a_{n}]) & \text{by defn (\ref{defn:delta}) of $\delta_{A}$}
\end{align*}
By definition, $\mcomonad{k}{}$ is a comonad.
\end{proof}
\end{thm}
For every $l,k \in \omega$ such that $l \leq k$ and $\sigma$-structure $A$, there exists an inclusion $i_{A}^{l,k}: \mcomonad{l}{A} \longrightarrow \mcomonad{k}{A}$. 
\begin{prop}
The inclusion maps form a natural transformation $i^{l,k}:\mcomonad{l}{} \longrightarrow \mcomonad{k}{}$. Further, each map preserves the counit and comultiplication (i.e. each map is a morphism of comonads). 
\end{prop}
\begin{proof}
\begin{equation}
\bfig \square[\mcomonad{l}{A}`\mcomonad{k}{A}`\mcomonad{l}{B}`\mcomonad{k}{B};i^{l,k}_{A}`\mcomonad{l}{f}`\mcomonad{k}{f}`i^{l,k}_{B}]\efig
\end{equation}
\begin{align*}
\mcomonad{k}{f} \circ i^{l,k}_{A}([a_{1},i_{1},\dots,i_{n-1},a_{n}])    &= \mcomonad{k}{f}([a_{1},\dots,,a_{n}]) & \text{by defn of inclusion}\\
&= [f(a_{1}),i_{1},\dots,i_{n-1},f(a_{n})] &\text{by defn (\ref{defn:mComonadMorphism}) of $\mcomonad{k}{f}$} \\
&= i^{l,k}_{B}([f(a_{1}),i_{1},\dots,i_{n-1},f(a_{n})]) & \text{by defn of inclusion} \\ 
&= i^{l,k}_{B} \circ \mcomonad{l}{f}([a_{1},i_{1},\dots,i_{n-1},a_{n}]) & \text{by defn (\ref{defn:mComonadMorphism}) of $\mcomonad{k}{f}$}\\
\end{align*}
\end{proof}
The grading given by these inclusion maps seem to suggest that there is a colimit object capturing the information of $\mcomonad{k}{A}$ for every $k \in \omega$. This is indeed the case. Consider the structure $\mcomonad{\omega}{A}$ with domain $|\mcomonad{\omega}{A}|$ the set of sequences of the form $(a_{1},i_{1},\dots,i_{n-1},a_{n})$ for any $n \in \omega$. The structure on $\mcomonad{\omega}{A}$ is similar to the structure given to $\mcomonad{k}{A}$. 
\begin{prop}
Let $\omega$ be considered as a poset category under the usual order. The object $\mcomonad{\omega}{A}$ is the $\omega$-colimit of the family $\{\mcomonad{k}{A}\}_{k \in \omega}$ with the above inclusion maps. 
\begin{proof}
For every $k \in \omega$, define $i^{k}_{A}:\mcomonad{k}{A} \rightarrow \mcomonad{\omega}{A}$ as the inclusion (i.e. $[a_{1},i_{1},\dots,i_{n-1},a_{n}] \mapsto [a_{1},i_{1},\dots,i_{n-1},a_{n}]$ where $n \leq k$). Clearly, the following diagram commutes for all $l,k \in \omega$ with $l \leq k$
\begin{equation}
\bfig \Vtriangle[\mcomonad{l}{A}`\mcomonad{k}{A}`\mcomonad{\omega}{A};i^{l,k}_{A}`i^{l}_{A}`i^{k}_{A}]\efig
\label{eq:omegaColimit}
\end{equation}
Suppose that there exists a $\sigma$-structure $B$ and for every $l,k \in \omega$ with $l \leq k$, there exist morphisms $f^{l}:\mcomonad{l}{A} \longrightarrow B$, $f^{k}:\mcomonad{k}{A} \longrightarrow B$ such that $f^{l} = f^{k} \circ i^{l,k}$. Consider the morphism $u:\mcomonad{\omega}{A} \longrightarrow B$ given by $[a_{1},i_{1},\dots,i_{n-1},a_{n}] \mapsto f^{n}([a_{1},i_{n},\dots,i_{n-1},a_{n}])$. The following diagram commutes:
\begin{equation}
\bfig 
    \Vtriangle[\mcomonad{l}{A}`\mcomonad{k}{A}`\mcomonad{\omega}{A};i^{l,k}_{A}`i^{l}_{A}`i^{k}_{A}]
    \morphism(0,500)|l|/{@{>}@/^-7pt/}/<500,-1000>[\mcomonad{l}{A}`B;f^{l}]
    \morphism(1000,500)|r|/{@{>}@/^7pt/}/<-500,-1000>[\mcomonad{k}{A}`B;f^{k}]
    \morphism(500,0)|m|/.>/<0,-500>[\mcomonad{\omega}{A}`B;u]
\efig
\label{eq:omegaColimitU}
\end{equation}
Moreover, given the conditions on $f^{j}$ for all $j \in \omega$, $u$ is unique. Namely, suppose there exists a morphism $u':\mcomonad{\omega}{A} \longrightarrow B$ such that for all $j \in \omega$, $f^{j} = u' \circ i^{j}_{A}$. Suppose $w = [a_{1},i_{1},\dots,i_{k-1},a_{k}] \in \mcomonad{\omega}{A}$ then for all $j \geq k, w \in \mcomonad{j}{A}$.  
\begin{align*}
u(w)    &= f^{k}(w)  & \text{by defn of $u$}\\
        &= f^{j} \circ i^{k,j}_{A}(w) & \text{by (\ref{eq:omegaColimitU})} \\
        &= u' \circ i^{j}_{A} \circ i^{k,j}_{A}(w) & \text{by hypothesis on $u'$}  \\
        &= u' \circ i^{k}_{A}(w) & \text{by (\ref{eq:omegaColimit})} \\
        &= u'(w) & \text{by defn of inclusion} 
\end{align*}
Since $u(w) = u'(w)$ for all $w \in \mcomonad{\omega}{A}$, $u = u'$ so $u$ is unique as desired.  
\end{proof}
\end{prop}
\section{Positional Form and Equivalences}\label{sec:positionalFormM}
The positional form representation for the simulation and bisimulation games involving $A,B$ are, just as with the EF game, sequences of pairs of elements in $A,B$ of length $\leq k$. However, the choice of each pair must be local (i.e. only one transition away from the previous pair). Our definitions will reflect this modification. Recall, from \ref{sec:positionalFormEF}, that $\Gamma_{k}(A,B) = (A \times B)^{\leq k}$ and for a $\sigma$-morphism $f:\mcomonad{k}{A} \longrightarrow B$ there exists the Kleisli coextension $f^{*}:\mcomonad{k}{A} \longrightarrow \mcomonad{k}{B}$. Define the set function $\theta_{f}:|\mcomonad{k}{A}| \longrightarrow \Gamma_{k}(A,B)$ by $s = [a_{1},i_{1},\dots,i_{n-1},a_{n}] \mapsto [(a_{1},b_{1}),\dots,(a_{n},b_{n})]$ where $f^{*}(s) = [b_{1},i_{1},\dots,i_{n-1},b_{n}]$. 
\begin{defn}
$S \subseteq \Gamma_{k}(A,B)$ is a strategy in positional form if $S$ satisfies the following conditions:
\begin{enumerate}[label=(S\arabic*),ref=S\arabic*,start=0]
\item For every $a \in A$, there exists a unique $b \in B$ such that $[(a,b)] \in S$ \label{eq:S1st}
\item For all $\zeta \in S$ with last position $(a,b)$ and $i \in [m]$ with $(a,a') \in R_{i}^{A}$, there exists a unique $b' \in B$ such that $\zeta' = \zeta[(a',b')] \in S$. Denote this update as $\zeta \xlongrightarrow{(i,a):b} \zeta'$ \label{eq:S2nd}
\item $S$ is reachable: For all $\zeta \in S$, there is a chain \label{eq:S3rd}
$$\zeta_{1} \longrightarrow \dots \longrightarrow \zeta_{n}$$
such that $\zeta_{n} = \zeta$ and $\zeta_{i} \in S$. 
\end{enumerate}
\end{defn}
\begin{prop}
If $f:\mcomonad{k}{A} \longrightarrow B$ is a $\sigma$-morphism, then there exists a strategy in positional form $S_{f}$.
\begin{proof}
Define: 
$$S_{f} = \{\theta_{f}(w) \mid w \in \mcomonad{k}{A}\}$$
\begin{itemize}
\item Suppose $a \in A$, then there exists a unique $b$ such that $f([a]) = b$. By definition of $\theta_{f}$, $[(a,b)] \in S_{f}$. 
\item Suppose $\zeta \in S_{f}$ with $|\zeta| < k$ and last position $(a,b)$, then there exists some $w \in \mcomonad{k}{A}$ such that $\zeta = \theta_{f}(w)$. Consider arbitrary $i \in [m]$ with $(a,a') \in  R_{i}^{A}$, then there exists a unique $b' \in B$ such that $f(w[i,a']) = b'$. Hence, $\theta_{f}(w[i,a']) = \zeta[(a',b')] \in S_{f}$.
\item Suppose $\zeta \in S_{f}$, then $\zeta = \theta_{f}(w)$ for some $w \in \mcomonad{k}{A}$. Let $w_{i}$ be the $i$-th element in the sequence $\delta_{A}(w) \in \mcomonad{k}{\mcomonad{k}{A}}$, then:  
$$\theta_{f}(w_{1}) \longrightarrow \dots \longrightarrow \theta_{f}(w_{n}) = \zeta$$ 
Hence, $S_{f}$ is reachable.
\end{itemize}
\end{proof}
\label{prop:fToPosFormM}
\end{prop}
\begin{prop}
Conversely, for every strategy in positional form $S$ there exists a $f:\efcomonad{k}{A} \rightarrow B$ such that $S = S_{f}$
\begin{proof}
$S_{f} \subseteq S$ The approach is to construct an appropriate $f$. We construct $f$ by recursion, up to $k$, on the length of a play $w \in \efcomonad{k}{A}$. \\ 
\textit{Base Case:} Suppose $w = [a]$ for $a \in A$. By (\ref{eq:S1st}), there exists a unique $b$ such that $[(a,b)] \in S$. Let $f(w) = b$. \\
\textit{Recursive Step:} Assume for the recursion, that $f(w)$ is defined for $|w| = n < 2k+1$ and that $\theta_{f}(w) \in S$. Consider $w' = w[i,a]$ for $(\epsilon_{A}(w),a) \in R_{i}$. By (\ref{eq:S2nd}), there exists a unique $b \in B$ such that $\zeta' = \theta_{f}(w)[(a,b)]$. Let $f(w') = b$. \\
$S \subseteq S_{f}$ We must show that for every $\zeta \in S$, there exists a $w \in \mcomonad{k}{A}$ such that $\zeta = \theta_{f}(w)$. By induction on reachability sequence of $\zeta$ (obtained via \ref{eq:S3rd}). \\
\textit{Base Case:} Suppose $\zeta = [(a,b)]$. By the construction of $f$ above, $f([a]) = b$, so $\zeta = \theta_{f}([a])$. \\
\textit{Inductive Step:} Assume for the inductive hypothesis that $\zeta = \theta_{f}(w)$ for some $w \in \mcomonad{k}{A}$. Suppose $\zeta \xlongrightarrow{(i,a):b} \zeta'$. Consider $w' = w[i,a]$, then $\theta_{f}(w') = \zeta'$.
\end{proof}
\label{prop:posFormToFM}
\end{prop}
\subsection{Equivalence $\exists^{+}\mathcal{M}_{\omega,k}$}
\begin{defn}
A position $\zeta = [(a_{1},b_{1}),\dots,(a_{n},b_{n})] \in \Gamma_{k}(A,B)$ is \textit{winning for Duplicator in $\mgame{k}{A}{B}$} if $\zeta$ a is simulation. That is, for all $i = 2,\dots,n$, $(a_{i-1},a_{i}) \in R_{z}^{A} \Rightarrow (b_{i-1},b_{i}) \in R_{z}^{B}$ for some $R_{z} \in \sigma$ binary and for all $P \in \sigma$ unary $a_{i} \in P_{j} \Rightarrow b_{i} \in P_{j}$. Naturally, we can extend the definition to say that a strategy in positional form $S \subseteq \Gamma_{k}(A,B)$ is \textit{winning for Duplicator in $\mgame{k}{A}{B}$} if for all $\zeta \in S$, $\zeta$ is winning for Duplicator $\mgame{k}{A}{B}$. Function $f:\mcomonad{k}{A} \longrightarrow B$ is \textit{winning for Duplicator $\mgame{k}{A}{B}$} if $S_{f}$ is winning for Duplicator in $\mgame{k}{A}{B}$ 
\end{defn}
Intuitively, condition (\ref{eq:S2nd}) and winning for Duplicator together form the forth requirement in the back-and-forth definition of bisimulation. With the definition of positional form for the modal fragment in place, the main theorem relating the comonad and equivalence in $\mathcal{M}_{\infty,k}$ can be proved.  
\begin{thm}
If $A,B$ are Kripke structures and $f:\mcomonad{k}{A} \rightarrow B$ is a function, then 
\center{$f:\mcomonad{k}{A} \longrightarrow B$ is a $\sigma$-morphism if and only if $f$ is winning for Duplicator $\mgame{k}{A}{B}$}
\begin{proof}
$\Leftarrow$ Suppose $\zeta \in S_{f}$, the by definition of $S_{f}$, $\zeta = \theta_{f}(w)$ for some $w \in \mcomonad{k}{A}$. If $w = [a]$, then it is trivally true that $\zeta = [(a,b)]$ is a simulation (i.e. winning for Duplicator). If $w = v[i,a']$ (i.e. $\zeta = \theta_{f}(v)[(a',b')]$) and suppose the last position of $\theta_{f}(v)$ is $(a,b)$, then by interpretation of $R_{i}$ on $\mcomonad{k}{A}$, $(v,w) \in R_{i}^{\mcomonad{k}{A}}$. By $f$ a $\sigma$-morphism, $(f(v),f(w)) \in R_{i}^{B}$. By the definition of $\theta_{f}$, $f(v) = b$ and $f(w) = b'$, so $(b,b') \in R_{i}^{B}$. Moreover, for predicate $P \in \sigma$, if $a' \in P^{A}$, then $w \in P^{\mcomonad{k}{A}}$ by $a$ being the last element of $w$, so $f(w) = b' \in P^{B}$ by $f$ a $\sigma$-morphism. Therefore, $\zeta$ is a simulation and winning for Duplicator. Hence, $S_{f}$ is winning for Duplicator.\\
$\Rightarrow$ Suppose $(v,w) \in R_{i}^{\mcomonad{k}{A}}$ for $R_{i} \in \sigma$ binary relation, then by intepretation of $R_{i}$ on $\mcomonad{k}{A}$, $v = w[i,a]$ or $w = v[i,a]$. Without loss of generality, assume $w = v[i,a]$. Consider $\zeta \in \theta_{f}(w) \in S_{f}$. By $w = v[i,a']$ and definition of $\theta_{f}$, $\theta_{f}(w) = \theta_{f}(v)[(a',f(w))]$ where $\theta_{f}(v) = u[(\epsilon_{A}(v),f(v))]$ for some (possibly empty) sequence $u$. Since $\zeta$ is a simulation (i.e. winning for Duplicator), $(f(v),f(w)) \in R^{B}$. Hence, $f$ is a $\sigma$-morphism.
\end{proof}
\label{thm:toPositionalFormM}
\end{thm}
\begin{prop}
For all $k \in \omega$, the following are equivalent:
\begin{enumerate}[label=(\arabic*)$_{k}$]
\item For every $\phi \in \exists^{+}\mathcal{M}_{\omega,k}$, $A \vDash \phi \Rightarrow B \vDash \phi$
\item There exists a strategy in positional form $S \subseteq \Gamma_{k}(A,B)$ that is winning for Duplicator 
\end{enumerate}
\begin{cor}
There exists a $\sigma$-morphism $f:\mcomonad{k}{A} \longrightarrow B$ if and only if for all $\phi \in \exists^{+}\mathcal{M}_{\omega,k}, A \vDash \phi \Rightarrow B \vDash \phi$
\begin{proof}
Straightforward from theorem (\ref{thm:toPositionalFormM}) and proposition (\ref{prop:toSyntaxM}).
\end{proof}
\label{cor:forthOneM}
\end{cor}
\begin{cor}
There exists $\sigma$-morphisms $f:\mcomonad{k}{A} \longrightarrow B$, $g:\mcomonad{k}{B} \longrightarrow B$ if and only if $\equivL{A}{B}{\exists^{+}\mathcal{M}_{\omega,k}}$. 
\begin{proof}
Recall definition (\ref{defn:equivLogic}) and apply above corollary (\ref{cor:forthOneM}) to $f$ and $g$.  
\end{proof}
\label{cor:forthM}
\end{cor}
\label{prop:toSyntaxM}
\end{prop}
\subsection{Equivalence $\mathcal{M}_{\omega,k}$}
Just as with the $\efcomonad{k}{}$ comonad, $\mcomonad{k}{}$ can also internalize winning strategies in the bisimulation game $\mgameSym{k}{A}{B}$. That is, given two Kripke $\sigma$-structures $A,B$, with $\sigma$-morphisms $f:\mcomonad{k}{A} \longrightarrow B$ and $g:\mcomonad{k}{B} \longrightarrow A$, Duplicator has a winning strategy in $\mgameSym{k}{A}{B}$ if $f$ and $g$ are compatible in positional form. Namely, given the canonical swap function, if $S_{f} = S_{g}^{-}$ and $S_{g} = S_{f}^{-}$, then there exists a bisimulation between $A$ and $B$. To see this $f:\mcomonad{k}{A} \longrightarrow A$ implies that $S_{f}$ is winning $\mgame{k}{A}{B}$ and the forth condition in the simulation from $A$ to $B$ is satisfied. Likewise, $g:\mcomonad{k}{B} \longrightarrow A$ implies that $S_{g}$ is winning in $\mgame{k}{B}{A}$ and the forth condition in the simulation from $B$ to $A$ is satisfied. The condition that $S_{f} = S_{g}^{-}$ means that the forth condition in the simulation from $B$ to $A$ is equivalent to the back condition in the bisimulation from $A$ to $B$. 
\begin{prop}
For all $k \in \omega$, the following are equivalent:
\begin{enumerate}[label=(\arabic*)$_{k}$]
\item $\equivL{A}{B}{\mathcal{M}_{\omega,k}}$ 
\item There exists a winning strategy in positional form $S \subseteq \Gamma_{k}(A,B)$ such that $S$ is winning for Duplicator and $S
^{-}$ is winning for Duplicator. 
\end{enumerate}
\end{prop}
\section{Guarded Unfolding}
