\chapter{Introduction} 
Given a mathematical theory $T$, model theory studies the relationship between the formal sentences needed to express $T$ and the mathematical structures that satisfy $T$. Category theory, on the other hand, studies the structure-preserving relationships between two or more structures that satisfy $T$. The fact that both of these fields abstract notions that are employed in every mathematical field, mean they occupy a high-level, even philosophical\cite{Mancosu2010}\cite{Landry2017}, place in mathematics. Despite their abstract nature, both model theory and category theory manifest concretely in the field of logic applied to computer science. In the case of model theory, restricting attention to finite mathematical structures, i.e. finite model theory, provides new avenues in complexity theory \cite{Immerman1998} and database theory \cite{Abiteboul1995}. In the case of category theory, many constructions, in particular monads and comonads, are used in the semantics of programming languages \cite{BrookesGeva1992}. This dissertation draws a bridge between the applications of these two fields. The bridge consists of capturing games, e.g. Ehrenfreucht-{\Fraisse}, pebbling, bisimulation, which are used to prove when two structures are indistinguishable by a logic as comonadic constructions. This categorical reformulation ``internalizes'' the games between two structures as a constructions within the category of structures itself. Not only does this internalization of games provide easy proofs of common results, the category of coalgebras give a categorical definition of combinatorial parameters, like tree-depth and tree-width. The inspiration for this dissertation was the paper \cite{Abramsky2017} which constructed a family of comonads $\{\pcomonad{k}{}\}$ that internalizes pebble games. With this dissertation we construct and prove analogous results for a family of comonads $\{\efcomonad{k}{}\}$ that internalizes Ehrenfreucht-{\Fraisse} games. We also show that the unfolding comonad for modal $\{\mcomonad{k}{}\}$ logics detailed in \cite{Gradel2014}, share similar properties and internalize bisimulation games.  
\\~\\
\noindent All results are proven by author, unless explicitly cited or stated otherwise. 
\section{History}
\subsection{Finite Model Theory and Games}
Model theory, unlike most mathematical fields which develop from other mathematics, developed from philosophical preoccupations with the language and concepts employed by mathematicians \cite{Mancosu2010}. Early results and constructions such as G{\"o}del's Completeness Theorem, L{\"o}wenheim-Skolem Theorems, and Robinson's hyperreals explicitly address the role of logic and infinity in the language of mathematics. Moreover, the definition of model-theoretic satisfaction used today is argued to have stemmed form Tarski's deflationary definitions of truth \cite{Mancosu2010}. Nevertheless, in the last several decades, model theory has grown into a mature mathematical field generalizing and providing insight into algebraic geometry,  number theory, and computer science. For a long time, model theory developed with no restriction on the cardinality of the structures in question and with no particular interest in the theory of finite structures. However, Fagin's Theorem discovered in Ronald Fagin's 1973 doctoral thesis demonstrated that existential second-order logic on finite structures captured the complexity class \textbf{NP} \cite{Fagin1974}. This spawned a new interest in the model theory of finite and recursive structures.  Another reason why finite model theory has developed as its own independent study, rather than occupying a section in (unrestricted) model theory, is the different techniques used. Many of the classical results in model theory (e.g. Compactness Theorem, Lyndon's Theorem, L{\"o}wenheim-Skolem theorems, L{\'o}s-Vaught test) only work on or apply to infinite models. Moreover, the common constructions (e.g. Henkin models, ultraproducts) produce infinite structures. One technique that works both on finite and infinite structures are back-and-forth games. This technique, phrased as systems for partial isomorphisms, was developed in {\Fraisse}'s 1953 thesis. Arguably, this work originates from generalizing Cantor's back-and-forth argument which showed two countable dense linear orders are isomorphic. Ehrenfreucht, in 1961, phrased these back-and-forth systems, perspicaciously, as a two-player game. Since then, there have been many such games developed such as pebble games \cite{Immerman1982} for finite-variable logics, bisimulation games \cite{Gradel2014} for modal and guarded logics, and even fragments of second-order logics like the bijection game \cite{Hella1996} for counting-quantifier logic and extended pebble game \cite{Libkin2004} for monadic second-order logic.
\subsection{Category Theory and Comonads}
Category theory finds its origin in the 1945 paper \textit{General Theory of Natural Equivalences} by Eilenberg and MacLane. This paper develops the general theory of functors and natural transformations. The paper shows how notions of functor, isomorphism, and natural isomorphism are used for classifying and distinguishing structures through different notions of equivalence. Eilenberg and MacLane's paper had emerged out of generalizing work in homology and cohomology; group structures that are used to distinguish spaces. Comonads had first appeared in the work of Roger Godemont, in 1958, in a book on sheaf theory. Sheaves were integral to the modern turn of algebraic geometry and found their origins in studying manifolds. Sheaves are related to capturing the local data of a space. From this history, it becomes clear that category theory is a powerful language for (1) distinguishing structures through notions of equivalence and (2) capturing local information about a mathematical object. Considering that Ehrenfreucht-{\Fraisse} style games capture equivalence of structures in a ``local' way (e.g. finite rank, finite variable fragments), it seems natural to reformulate these games in category theory. Moreover, the seminal paper \cite{Moggi1991} Eugenio Moggi, sparked the use of monads and comonads in computer science. In this paper, Moggi provides a categorical semantics of computation. Essentially, Moggi represents types in a program as objects in a category, and computational effects as a functor (in fact, a monad) on this category. This abstraction made it possible to capture and formalize many ideas in modern functional programming languages.  
\section{General research program}
In finite model theory, through the discovery of Fagin's theorem, the field of descriptive complexity was born. The field is often praised for giving a more intrinsic definition of computational complexity classes. Namely, descriptive complexity measures complexity, not through steps or tapes in machine models, but rather through the amount of expressive resources needed \textit{describe} the problem. On the other hand, through Moggi's work on monads in computation, a very abstract definition of computation is given. By formulating games, which are integral to inexpressiblity proofs, in terms of comonads, we make step towards giving intrinsic categorical, rather than model-theoretic, definition of complexity classes. This step is part of a larger research program. \\~\\
Namely, in the conclusion of \cite{Abramsky2017}, the authors mention questions related to a larger research program linking finite model theory in the analysis of computational complexity, and the categorical semantics of programs. More specifically, the conclusion asks whether there are comonad formulations, similar to one given for the pebbling game, for the Ehrenfreucht-{\Fraisse} game. It is also asks whether the unravelling comonad in \cite{Gradel2014} plays a similar role. Moreover, in light of the result linking tree-width to coalgebras of the pebbling comonad, it is asks whether other combinatorial parameters have similar definitions. All of these questions are answered in the affirmative. One long range goal of the program is to analyze and generalize Rossman's seminal theorem on homomorphism preservation using category theory. Given that much of the work on the Ehrenfreucht-{\Fraisse} game in this dissertation was inspired by Rossman's paper, we provide further evidence for the success of such an endeavour. Furthermore, the colimit constructions we investigate may also shed light on the comments made in \cite{Abramsky2017} on the paper \cite{Nesetril2013} relating different limits of graphs with bounded tree-depth. 
