\chapter{Introduction} 
Given a new mathematical theory $T$, model theory studies the relationship between the formal sentences needed to express $T$ and the mathematical structures that satisfy $T$. Category theory, on the other hand, studies the structure-preserving relationships between two or more structures that satisfy $T$. The fact that both of these fields abstract notions that are employed in every mathematical field, mean they occupy a high-level, even philosophical\cite{Mancosu2010}\cite{Landry2017}, place in mathematics. Despite their abstract nature, both model theory and category theory manifest concretely in the field of logic applied to computer science. In the case of model theory, restricting attention to finite mathematical structures, i.e. finite model theory, provides new avenues in complexity theory \cite{Immerman1998} and database theory \cite{Abiteboul1995}. In the case of category theory, many constuctions, in particular monads and comonads, are used in the semantics of programming languages \cite{BrookesGeva1992}. This dissertation draws a bridge between the applications of these two fields. The bridge consists of capturing games, e.g. Ehrenfreuct-{\Fraisse}, pebbling, bisimulation, which are used to prove when two structures are indistinguisable by a logic as comonadic constructions. This categorical reformulation ``internalizes'' the games between two structures as a constructions within the category of structures itself. Not only does this internalization of games provide easy proofs of common results, the category of coalgebras give a categorical definition of combinatorial parameters, like tree-depth and tree-width. The inspiration for this dissertation was the paper \cite{Abramsky2017} which constructed a family of comonads $\{\pcomonad{k}{}\}$ that internalizes pebble games. With this dissertation we construct and prove analogous results for a family of comonads $\{\efcomonad{k}{}\}$ that internalizes Ehrenfreucht-{\Fraisse} games. We also show that the unfolding comonad for modal $\{\mcomonad{k}{}\}$ and guarded $\{\gcomonad{k}{}\}$ logics detailed in \cite{Gradel2014}, share similar properties and internalize bisimulation games.  
\\~\\
\noindent All results are proven by author, unless explictly cited or stated otherwise. 
\section{History}
\subsection{Finite Model Theory and Games}
Model theory, unlike most mathematical fields which develop from other mathematics, developed from philosophical preoccupations with the langauge and concepts employed by mathematicians. Early results and constructions such as G{\"o}del's Completeness Theorem, L{\"o}wenheim-Skolem Theorems, and Robinson's hyperreals explicitly address the role of logic and infinity in the language of mathematics. Moreover, the definition of model-theoretic satisfaction used today is argued to have stemmed form Tarski's deflationary definitions of truth \cite[]{Mancosu2010}. Nevertheless, in the last several decades, model theory has grown into a mature mathematical field generalizing and providing insight into algebraic geometry,  number theory, and computer science . For a long time, model theory developed with no restriction on the cardinality of the structures in question and with no particular interest in the theory of finite structures. However, Fagin's Theorem discovered in Ronald Fagin's 1973 doctoral thesis demonstrated that existential second-order logic on finite structures captured the complexity class \textbf{NP}. This spawned a new interest in the model theory of finite and recursive structures.  Another reason why finite model theory has developed as its own independent study, rather than occupy a section in (unrestricted) model theory, is the different techniques used. Many of the classical results in model theory (e.g. Compactness Theorem, Lyndon's Theorem, L{\"o}wenheim-Skolem theorems, L{\'o}s-Vaught test) only work on or apply to infinite models. Moreover, the common constructions (e.g. Henkin models, ultraproducts) produce infinite structures. One technique that works both on finite and infinite structures are back-and-forth games. This technique, phrased as systems for partial isomorphisms, was developed in {\Fraisse}'s 1953 thesis. Arguably, this work originates from generalizing Cantor's back-and-forth argument which showed two countable dense linear orders are isomorphic. Ehrenfreucht, in 1961, phrased these back-and-forth systems, perspeciously, as a two-player game. Since then, there have been many such games developed such as pebble games \cite{Immerman1982} for finite-variable logics, bisimulation games \cite{Gradel2014} for modal and guarded logics, and even fragments of second-order logics like the bijection game \cite{Hella1996} for counting-quantifier logic and extended pebble game \cite{Libkin2004} for monadic second-order logic.
\subsection{Category Theory and Comonads}
Category theory, finds its origin in the 1945 paper \textit{General Theory of Natural Equivalences} by Eilenberg and MacLane. This paper develops the general theory of functors and natural transformations. The paper shows how notions of functor, isomorphism, and natural isomorphism are used algebraic topology for purpose of classifying and distinguishing structures. These notions extend to many different mathematical contextets. Given that the back-and-forth games used in model theory are equivalent to distinguishability in a logic, category theory is natural setting to formalize these games. Moreover, the algebraic constructions of monad and comonad are used to provide semantics for functional programming languages \cite{Moggi1991} \cite{BrookesGeva1992}. This paper is part of the general class of research programs that seek to ``categorify'' the notions of equivalence and classification in a wide-range of mathematical fields.   
\section{General resarch programme}

